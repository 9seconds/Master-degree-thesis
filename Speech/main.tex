% coding: utf-8
% --------------------------------------------------------------------------------------------------
% "Синтез оптимального стохастического управления", 2011 год
% : рецензия и отзывы
% --------------------------------------------------------------------------------------------------



\documentclass[12pt,a4paper,legalpaper,pdftex]{letter}
\usepackage{cmap}

\usepackage[T2A]{fontenc}
\usepackage[utf8x]{inputenc}
\usepackage[russian]{babel}
\selectlanguage{russian}

\usepackage[breaklinks,pdftex,unicode,bookmarks,pdfborder={0 0 0}]{hyperref}

\usepackage{multirow}
\usepackage{ragged2e}
\usepackage[pdftex]{graphicx}

\usepackage{microtype}
\SetProtrusion{encoding=T2A,family=cmr}
{
« =  {1000,     },
» =  {    , 1000},
„ =  {1000,     },
“ =  {    , 1000},
( =  {1000,     },
) =  {    , 1000},
! =  {    , 1000},
? =  {    , 1000},
: =  {    , 1000},
; =  {    , 1000},
. =  {    , 1000},
- =  {    , 1000},
{,}= {    , 1000}
}
\DeclareMicrotypeSet{t2atext}{encoding=T2A}
\UseMicrotypeSet{t2atext}



\frenchspacing
\fussy
\raggedbottom



\setlength{\leftmargin}{20mm}
\setlength{\rightmargin}{10mm}
\righthyphenmin=2
\emergencystretch=6pt




\begin{document}



\newcommand{\header}[1]{    \begin{center}{\fontsize{26pt}{26pt}\selectfont \textbf{#1}}\end{center}  }
\newcommand{\subheader}[1]{ \vspace{20pt} {\fontsize{16pt}{10pt}\selectfont Слайд #1} \vspace{8pt} \\ }



\pagestyle{empty}



\header{Финальная речь}



\subheader{1}
Здравствуйте, уважаемый председатель и члены государственной аттестационной комиссии. Вашему вниманию предлагается выпускная квалификационная ра­бота на тему <<Синтез оптимального регулятора в стохастических системах>> студента Архипова Сергея Валерьевича, чей научный руководитель~--- Пакшин Павел Владимирович.



\subheader{2}
В работе рассматривается проблема синтеза оптимального управления в системах, в которых случайные воздействия~--- например со стороны внешней среды~--- оказывают значительную роль на ее динамику. Рассмотрим сначала случай детерминированных систем. Предположим, что система задается уравнением (1) и требуется найти такое оптимальное управление, которое бы доставляло минимум некоторому заранее заданному критерию качества.



\subheader{3}
Если критерий качества можно представить в виде некоторого интеграла, то необходимое условие оптимальности можно записать с помощью системы уравнений Беллмана (4).



\subheader{4}
Наиболее распространенным на практике критерием качества является квадратичный критерий качества. Используя уравнения Беллмана, было показано, что оптимальное управление может быть синтезировано с помощью уравнения (6), для решения которого необходимо решение дифференциального уравнения Риккати.



\subheader{5}
Если же рассматривается бесконечный интервал времени, то в случае полностью управляемой стационарной системы, для синтеза оптимального уравнения достаточно решить нелинейное матричное алгебраическое уравнение Риккати.



\subheader{6}
Как было установлено, эти результаты могут быть обобщены и на случай стохастических систем, чьи уравнения динамики задаются в виде процесса Ито, а критерий качества может быть представлен в виде математического ожидания.



\subheader{7}
В случае квадратичного критерия качества было установлено, что оптимальное управление может быть синтезировано способами, аналогичными подходам, применяемым для детерминированных систем. Это справедливо как для конечного интервала времени, так и для бесконечного в случае полностью управляемой системы.



\subheader{8}
Рассмотрим решение задачи наведения баллистической ракеты на заданную траекторию в условиях турбулентной атмосферы. Динамическую систему для этой задачи можно представить в виде системы (16) и определить критерий качества (17).



\subheader{9}
Рассмотрим пример задачи при некоторых заранее заданных коэффициентах, отражающих полет ракеты на высоте километра от земли в течение 250 секунд.



\subheader{10}
Для численного моделирования задачи и проведения вычислительных экспериментов, был разработан программный комплекс на языке \texttt{Java}, предназначенный для использования в UNIX-подобных операционных системах, который помимо непосредственного моделирования, реализует несколько алгоритмов синтеза оптимального управления, о которых и пойдет речь дальше.



\subheader{11}
Первый алгоритм основывается на численном решении дифференциальных уравнений Риккати. После синтеза решения, полученные результаты используются для непосредственного вычисления управления на каждом шаге.



\subheader{12}
Результаты решения алгоритмом в условиях типичных показателей турбулентности на заданной высоте приведены на слайде.



\subheader{13}
Следующий алгоритм основывается на решении алгебраического уравнения Риккати и предназначен исключительно для стационарных полностью управляемых динамических систем.



\subheader{14}
Результаты решения с помощью этого алгоритмы приведены на слайде.



\subheader{15}
Следующие алгоритмы являются адаптивными, для их корректного определения требуется введение понятия вторичного критерия качества, который характеризует отклонение значения критерия качества на каждом интервале от полностью детерминированного, идеального случая.



\subheader{16}
Первый адаптивный алгоритм основывается на том предположении, что решения уравнений Риккати следует пересчитывать только тогда, когда значение вторичного критерия качества превысит некоторый порог.



\subheader{17}
Этот алгоритм показал превосходные результаты, позволил определнным образом снизить отклонения, возникающие вследствие зашумленности атмосферы. Его недостаток заключается в том, что он очень тяжел с вычислительной точки зрения.



\subheader{18}
Последний алгоритм основывается на предположении, что пересчет требуется на каждом подынтервале времени.



\subheader{19}
Этот алгоритм показал свою полную неэффективность, лишь усугубив отклонения.



\subheader{20}
Итак, была исследована задача синтеза оптимального регулятора в стохастических системах, и на основе полученных результатов были разработан ряд алгоритмов. С помощью программного комплекса была исследована и проанализирована их эффективность.



\end{document}

