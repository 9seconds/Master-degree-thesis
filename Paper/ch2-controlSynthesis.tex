\chapter{Синтез оптимального управления}
% ==============================================================================================
\newcommand{\optF}{  \optimum{\calf{F}} } % оптимальная функция F
\newcommand{\optU}{  \optimum{u}        } % оптимальное управление u
\newcommand{\optX}{  \optimum{x}        } % оптимальная траектория x
\newcommand{\funcL}{ \sff{L}            } % функция Лагранжа
% ==============================================================================================




% **********************************************************************************************
\section{Принцип Беллмана}
% **********************************************************************************************



Рассмотрим следующую задачу Лагранжа. Для системы $\dot{x} = f(x, u, t)$, начинающей движение из начального состояния $x(t_1) = x_1$, найдем управление $u(t)$, ограниченное некоторым допустимым классом функций $\Omega(t)$ и определенное на интервале времени $[t_1, t_2]$, которое минимизирует показатель качества

\beq{eq:2/1/1}
    \calf{F} = \int\limits_{t_1}^{t_2} \funcL(x, u, t)\,dt \mbox{,}
\eeq

где функция $\funcL(x, u, t)$ предполагается непрерывной по $t$. Конечное состояние $x(t_2) = x_2$ не задано. Все функции, принадлежащие $\Omega(t)$ ограничены по амплитуде в любой момент времени $t$: $|u_j(t)| \leqslant V_j$. Множество точек $|u_j(t)| \leqslant V_j$ ($j = 1, 2, \ldots, n$) будем обозначать как $\Upsilon$.

Показатель качества $\calf{F}$ для фиксированного значения $t_2$ зависит от переменных $u(t)$, $x_1$ и $t_1$. Однако оптимальное значение функционала $\calf{F}$ зависит лишь от начального состояния $x_1$ и момента времени $t_1$. Обозначим оптимальное значение этого функционала как $\optF(x_1, t_1)$. Если поставлена задача оптимального управления, то для каждой точки $x_1$ ставится в соответствие некоторое значение критерия оптимальности. Функция $\optF(x_1, t_1)$ определяет гиперповерхность в $n+1$-мерном пространстве. Эта гиперповерхность, конечно, в явном виде не задана, но некоторые ее свойства позволяют вывести ряд необходимых условий, которым должна удовлетворять функция оптимального управления $\optU(t)$.

Обозначим через $x \bigl( u(t), t \bigr)$ траекторию, которая получается в результате воздействия управления $u(t)$ на систему $\dot{x} = f(x, u, t)$ с начальным состоянием $x_1$ в момент времени $t = t_1$. При этом $\optF$ определяется выражением

\beq{eq:2/1/2}
    \optF(x_1, t_1) = \underset{\substack{u(t) \in \Omega(t) \\ t_1 \leqslant t \leqslant t_2}}{\min} \biggl\{ \int\limits_{t_1}^{t_2} \funcL\bigl(x(u, t), u, t\bigr)\,dt \biggr\} \mbox{.}
\eeq

Для некоторого момента времени $t'$ из интервала между $t_1$ и $t_2$ это выражение можно написать в таком виде:

\beq{eq:2/1/3}
    \optF(x_1, t_1) = \underset{\substack{u(t) \in \Omega(t) \\ t_1 \leqslant t \leqslant t_2}}{\min} \biggl\{ \int\limits_{t_1}^{t'} \funcL\bigl(x(u, t), u, t\bigr)\,dt + \int\limits_{t'}^{t_2} \funcL\bigl(x(u, t), u, t\bigr)\,dt \biggr\} \mbox{.}
\eeq

Уравнение~\ref{eq:2/1/3} позволяет применить для его решения принцип оптимальности. Для случая непрерывных систем принцип оптимальности можно сформулировать следующим образом.

\begin{statement}\lbl{statement:1}
	Оптимальное управление $\optU(t)$ на интервале времени $[t_1, t_2]$ имеет следующее свойство: для любого $t'$, заключенного в интервале $t_1 < t' < t_2$ независимо от значений, которые управление $\optU(t)$ принимало на интервале времени $[t_1, t']$, и, следовательно, независимо от значения $\optX(t')$ оно должно оставаться оптимальным управлением относительно состояния $\optX(t')$ на интервале времени $(t', t_2]$.
\end{statement}

Применяя этот принцип, уравнение~\ref{eq:2/1/3} можно преобразовать к следующему виду

\beq{eq:2/1/4}
    \optF(x_1, t_1) = \underset{\substack{u(t) \in \Omega(t) \\ t_1 \leqslant t \leqslant t'}}{\min} \biggl\{ \int\limits_{t_1}^{t'} \funcL\bigl(x(u, t), u, t\bigr)\,dt + \optF\bigl(x(t'), t'\bigr) \biggr\} \mbox{,}
\eeq

где $x(t')$~--- конечное состояние, которое является результатом действия управления $u(t)$ на интервале времени $[t_1, t']$\footnote{ Если предположить, что функция $\calf{F}$ имеет непрерывные частные производные по $x$ и $t$, можно легко прийти к уравнению Беллмана (см TODO). Однако, как будет показано ниже, это предположение не выполняется для большого класса задач оптимального управления. В приводимом здесь доказательстве, основанном на работе TODO, это предположение не используется }.

При оптимальном управлении $u(t) = \optU(t)$ на интервале $[t_1, t']$ имеем

\beq{eq:2/1/5}
    \optF(x_1, t_1) = \int\limits_{t_1}^{t'} \funcL\bigl(x(\optU, t), \optU(t), t\bigr)\,dt + \optF\bigl(x(t'), t'\bigr) \mbox{.}
\eeq

Перенося члены и деля на $t' - t_1$ получим

\beq{eq:2/1/6}
    -\frac{\optF\bigl( x(t'), t' \bigr) - \optF( x_1, t_1)}{t' - t_1} = \frac{1}{t' - t_1} \int\limits_{t_1}^{t'} \funcL\bigl(x(\optU, t), \optU(t), t\bigr)\,dt \mbox{.}
\eeq

При $t' \to t_1$ уравнение~\ref{eq:2/1/6} принимает вид

\beq{eq:2/1/7}
    \lim_{t' \to t_1} \biggr[ -\frac{\optF\bigl( x(t'), t' \bigr) - \optF( x_1, t_1)}{t' - t_1} \biggr] = \funcL\bigl(x_1, \optU(t_1), t_1\bigr) \mbox{.}
\eeq

Заметим, что для получения правой части выражения~\ref{eq:2/1/7} как предела правой части уравнения~\ref{eq:2/1/6} используется теорема о среднем значении (TODO). Если предел левой части уравнения~\ref{eq:2/1/7} существует, то можно определить величину

\beq{eq:2/1/8}
    \biggl[ \genericdiff{\optF}{t} \biggr]_{\optU, t_1} \eqdef \lim_{t' \to t_1} \biggr[ -\frac{\optF\bigl( x(t'), t' \bigr) - \optF( x_1, t_1)}{t' - t_1} \biggr] \mbox{.}
\eeq

Величина $[ \genericlinediff{\optF}{t} ]_{\optU, t_1}$ есть производная функции $\optF$ по времени, вычисленная в момент времени $t_1$. Анализируя правую часть выражения~\ref{eq:2/1/8}, можно видеть, что производную следует вычислять вдоль траектории, обусловленной управлением $\optU$ и начинающейся в $x_1$. Иными словами, пусть для некоторой задачи функция $\optF(x, t)$ определена для любого начального состояния $x$ и начального момента времени $t$. Пусть $x\bigl( u(t), t \bigr)$ определяет траекторию, обусловленную управлением $u(t)$; тогда вдоль любой из этих траекторий величина $\optF(x, t)$ будет изменяться во времени со скоростью, определяемой величиной $[ \genericlinediff{\optF}{t} ]_{u, t}$. В частности, вдоль оптимальной траектории она будет изменяться со скоростью $[ \genericlinediff{\optF}{t} ]_{\optU, t}$.

Таким образом

\beq{eq:2/1/9}
    \biggl[ \genericdiff{\optF}{t} \biggr]_{\optU, t_1} + \funcL\bigl(x_1, \optU(t_1), t_1\bigr) = 0 \mbox{.}
\eeq

Заметим, что при $u(t) \neq \optU(t)$ для интеграла~\ref{eq:2/1/5} в соответствии с определением должно выполняться неравенство

\beq{eq:2/1/10}
    \optF(x_1, t_1) \leqslant \int\limits_{t_1}^{t'} \funcL\bigl(x(\optU, t), \optU(t), t\bigr)\,dt + \optF\bigl(x(t'), t'\bigr) \mbox{.}
\eeq

Откуда следует, что

\beq{eq:2/1/11}
    \biggl[ \genericdiff{\optF}{t} \biggr]_{u, t_1} + \funcL\bigl(x_1, u(t_1), t_1\bigr) \geqslant 0 \mbox{.}
\eeq

Используя выражения~\vref{eq:2/1/9} и~\ref{eq:2/1/11}, получим

\beq{eq:2/1/12}
    \biggl[ \genericdiff{\optF}{t} \biggr]_{u, t_1} + \funcL\bigl(x_1, u(t_1), t_1\bigr) \geqslant \biggl[ \genericdiff{\optF}{t} \biggr]_{\optU, t_1} + \funcL\bigl(x_1, \optU(t_1), t_1\bigr) \mbox{,}
\eeq

или

\beq{eq:2/1/13}
    \underset{u(t_1) \in \Upsilon}{\min} \biggl\{ \biggl[ \genericdiff{\optF}{t} \biggr]_{u, t_1} + \funcL\bigl(x_1, u(t_1), t_1\bigr) \biggr\} = 0 \mbox{.}
\eeq

Уравнение~\ref{eq:2/1/13} справедливо в любой момент времени из интервала $[t_1, t_2]$, так что для любого момента времени $t$ и любого состояния $x$ на траектории $\optX(t)$, принимая его за начальное, можно написать

\beq{eq:2/1/14}
    \underset{u(t) \in \Upsilon}{\min} \biggl\{ \biggl[ \genericdiff{\optF}{t} \biggr]_{u, t} + \funcL\bigl(x(u, t), u, t\bigr) \biggr\} = 0 \mbox{.}
\eeq

Уравнение~\ref{eq:2/1/14} и есть функциональное уравнение Беллмана в общей форме, выражающее необходимое условие оптимальности.

Для любого состояния $x$ и момента времени $t$, когда функция $\genericlinediff{\optF}{t}$, $\genericlinediff{\optF}{x}$ и $f(x, u, t)$ непрерывны по $x$ и $t$, полную производную $[\genericlinediff{\optF}{t}]_{u, t}$ можно записать так:

\beq{eq:2/1/15}
    \biggl[ \genericdiff{\optF}{t} \biggr]_{u, t} = \biggl(\partdiff{\optF}{x}\biggr)^T f(x, u, t) + \partdiff{\optF}{t} \mbox{.}
\eeq

Так как частная производная $\partlinediff{\optF}{t}$ не зависит от $u$, уравнение \ref{eq:2/1/14} можно представить в виде

\beq{eq:2/1/16}
    - \partdiff{\optF}{t} = \underset{u(t) \in \Upsilon}{\min} \Biggl\{ \funcL\bigl(x(u, t), u, t\bigr) + \biggl(\partdiff{\optF}{x}\biggr)^T f(x, u, t) \Biggr\} \mbox{.}
\eeq

В тех случаях, когда это уравнение применимо, оно обеспечивает необходимое условие оптимальности. Отметим, что оно является необычной формой дифференциального уравнения в частных производных, которое включает операцию минимизации. Это, в общем случае нелинейное, дифференциальное уравнение в частных производных первого порядка относительно одной переменной $\optF$. Это уравнение определяет одноточечную краевую задачу с граничным условием вида

\beq{eq:2/1/17}
    \lim_{t_1 \to t_2} \optF(x_1, t_1) = \lim_{t_1 \to t_2} \int\limits_{t_1}^{t_2} \funcL(\optX, \optU, t)\,dt = 0 \mbox{.}
\eeq

Очевидно, это условие соответствует двум.

\benum
    \item
        Оно указывает, как следует вести поиск оптимальной $u(t)$: в любой момент времени $t$ поиск должен быть таким, чтобы минимизировать величину, заключенную в квадратные скобки в уравнении~\vref{eq:2/1/16}.

    \item
        Если функция оптимального управления $\optU(t)$ найдена, то~\ref{eq:2/1/16} сводится к уравнению без операции минимизации
        \beq{eq:2/1/18}
            - \partdiff{\optF}{t} = \funcL\bigl(x(\optU(t), t), \optU, t\bigr) + \biggl(\partdiff{\optF}{x}\biggr)^T f(x(\optU(t), t), \optU, t) \mbox{,}
        \eeq 
        которому удовлетворяет функция $\optF(x, t)$ для всех значений $t$ в интервале $[t_1, t_2]$.
\eenum

При отсутствии ограничений на величину $u$ и при условии, что функции $\funcL$ и $f$ имеют частные производные по $u$, оптимальное управление $\optU(t)$ можно найти путем дифференцирования выражения, заключенного в квадратные скобки в уравнении~\ref{eq:2/1/16} и приравнивания полученного результата к нулю. Это дает условие

\beq{eq:2/1/19}
    \biggl[ \partdiff{\funcL}{u} + \sum\limits_{i=1}^n \partdiff{\optF}{x_i} \centerdot \partdiff{f_i}{u} \biggr] = 0 \mbox{ для всех } t \in [t_1, t_2] \mbox{.}
\eeq

Если показатель качества соответствует задаче Майера, а именно $\calf{F} = P\bigl( x(t_2), t_2 \bigr)$, то, придерживаясь тех же рассуждений, получим соответствующее уравнение Беллмана

\beq{eq:2/1/20}
    \underset{u(t) \in \Upsilon}{\min} \Biggl\{ \biggl[ \genericdiff{\optF}{t} \biggr]_{u, t} \Biggr\} = 0
\eeq

с граничным условием

\beq{eq:2/1/21}
    \lim_{t_1 \to t_2} \optF\bigl(x(t_1), t_1\bigr) = P\bigl(x(t_2), t_2\bigr) \mbox{.}
\eeq

В тех же случаях, когда функции $\partlinediff{\optF}{t}$, $\partlinediff{\optF}{x}$ и $f(x, u, t)$ непрерывны по $x$ и $t$, имеем

\beq{eq:2/1/22}
    - \partdiff{\optF}{t} = \underset{u(t) \in \Upsilon}{\min} \Biggl\{ \biggl(\partdiff{\optF}{x}\biggr)^T f(x, u, t) \Biggr\}
\eeq

с тем же граничным условием~\ref{eq:2/1/21}.

Отметим, что тип граничного условия для конечного состояния не играет роли при выводе функционального уравнения. Таким образом, уравнение Беллмана~\ref{eq:2/1/14} или~\vref{eq:2/1/16} справедливо для задачи, даже если конечное состояние $x_2$ задано, а конечный момент времени $t_2$ нет. Условие~\vref{eq:2/1/17} по-прежнему справедливо, и нет необходимости вносить какие-либо изменения при выводе уравнения Беллмана.



% **********************************************************************************************
\section{Синтез оптимального управления в системах с квадратичным критерием качества}
% **********************************************************************************************



% **********************************************************************************************
\section{Связь с принципом максимума}
% **********************************************************************************************



% **********************************************************************************************
\section{Достаточное условие оптимальности}
% **********************************************************************************************
