% coding: utf-8
% --------------------------------------------------------------------------------------------------
% "Синтез оптимального стохастического управления", 2011 год
% : рецензия и отзывы
% --------------------------------------------------------------------------------------------------



\documentclass[12pt,a4paper,legalpaper,pdftex]{letter}
\usepackage{cmap}

\usepackage[T2A]{fontenc}
\usepackage[utf8x]{inputenc}
\usepackage[russian]{babel}
\selectlanguage{russian}

\usepackage[breaklinks,pdftex,unicode,bookmarks,pdfborder={0 0 0}]{hyperref}

\usepackage{multirow}
\usepackage{ragged2e}

\usepackage{microtype}
\SetProtrusion{encoding=T2A,family=cmr}
{
« =  {1000,     },
» =  {    , 1000},
„ =  {1000,     },
“ =  {    , 1000},
( =  {1000,     },
) =  {    , 1000},
! =  {    , 1000},
? =  {    , 1000},
: =  {    , 1000},
; =  {    , 1000},
. =  {    , 1000},
- =  {    , 1000},
{,}= {    , 1000}
}
\DeclareMicrotypeSet{t2atext}{encoding=T2A}
\UseMicrotypeSet{t2atext}



\frenchspacing
\fussy
\raggedbottom



\setlength{\leftmargin}{20mm}
\setlength{\rightmargin}{10mm}
\righthyphenmin=2
\emergencystretch=6pt




\begin{document}

\newcommand{\underscore}[1]{\underline{\hspace*{#1cm}}}

\newcommand{\header}[1]{\begin{center}{\fontsize{20.74pt}{10pt}\selectfont #1}\end{center}}
\newcommand{\subheader}[1]{\begin{center}{\fontsize{14pt}{12pt}\selectfont\center #1}\end{center}}
\newcommand{\wtitle}[1]{\begin{center}{\fontsize{17.28pt}{14pt}\selectfont <<\textbf{#1}>>}\end{center}}

\pagestyle{empty}


\header{Отзыв}
\subheader{на магистерскую диссертацию студента группы 86М1 факультета ВМК}
\subheader{Архипова Сергея Валерьевича}

\wtitle{Синтез оптимального регулятора в стохастических системах}

Магистерская диссертация Архипова С.\,В. посвящена задаче создания оптимального по квадратичному критерию управления в многомерных динамических стохастических системах. Исследование вопросов синтеза такого управления является одной из важнейших задач современной теории управления.

Задача синтеза такого управления была разделена на два этапа: сначала был исследован синтез оптимального регулятора в случае детерминированных систем, затем результаты были обобщены на случай стохастических систем. В работе рассматриваются как классические модели, основанные на работах Беллмана, Калмана и Мэрриэма, так и их обобщения на класс стохастических систем. Основой для создания моделей служат системы линейных дифференциальных уравнений. Проведен подробный анализ моделей, получены методы синтеза оптимального управления, а также установлены рамки и ограничения, в которых подобные методы реализуемы.

На примере задачи наведения ракеты на заданную траекторию в условиях турбулентной атмосферы, были разработаны алгоритмы, которые лежат в основе блока автоматического управления полетом. Эти алгоритмы основываются как на реализации классических результатов, так и на их расширениях: были предложены адаптивные алгоритмы, активно учитывающие специфику случайных воздействий со стороны внешней среды, исследованы области их применимости.

Результаты аналитического исследования подтверждены численными расчетами в программном комплексе, реализованном с помощью языков программирования \texttt{Java}, \texttt{Python}, \texttt{sed} и \texttt{awk}.
При выполнении работы Архипов Сергей Валерьевич зарекомендовал себя как высококвалифицированный специалист, владеющий современными математическими методами и средствами вычислительной техники; проявил способность самостоятельно решать поставленные задачи на современном уровне.

Считаю, что магистерская диссертация Архипова Сергея Валерьевича соответствует требованиям образовательного стандарта, предъявляемым к выпускным квалификационным работам по специальности <<Прикладная математика и информатика>>, и заслуживает оценки <<\textbf{отлично}>>.

\vspace{18pt}
\begin{flushright}
    \begin{tabular}[t]{p{5cm}r}
        Научный руководитель & \\
        профессор, д.\,ф--м.\,н & \underscore{4}~Пакшин П.\,В. \\
        & <<\underscore{0.5}>>~\underscore{2}~2011 г.
    \end{tabular}
\end{flushright}



\newpage



\header{Рецензия}
\subheader{на магистерскую диссертацию студента группы 86М1 факультета ВМК}
\subheader{Архипова Сергея Валерьевича}

\wtitle{Синтез оптимального регулятора в стохастических системах}

Синтез оптимального регулятора с обратной связью в многомерных стохастических системах является актуальной задачей, постоянно возникающей при проектировании различных систем и устройств, предназначенных для автоматизации процесса управления. Этому вопросу посвящена магистерская диссертация Архипова Сергея Валерьевича.

На примере задачи наведения ракеты на заданную траекторию в условиях турбулентной атмосферы, были предложены оригинальные адаптивные алгоритмы, позволившие снизить влияние шума на систему.

Вопрос синтеза регулятора в общем случае был исследован сведением начальной задачи к уравнениям Беллмана. При рассмотрении частного случая квадратичного критерия качества были получены результаты, позволившие сформулировать алгоритмы для решения поставленной задачи. Основными результатами являются:

\begin{itemize}
    \item Исследование задачи синтеза оптимального регулятора в случае детерминированных систем;
    \item Обобщение результатов на случай стохастических систем;
    \item Формулирование алгоритмов для решения поставленной задачи;
    \item Программная реализация решения на языке \texttt{Java};
    \item Анализ результатов, полученных в ходе вычислительных экспериментов.
\end{itemize}

Представленная на рецензию магистерская диссертация является завершенной работой, достаточно полно отражающей постановку исходной задачи.

Считаю, что работа Архипова Сергея Валерьевича удовлетворяет всем требованиям, предъявляемым к магистерским диссертациям по специальности «Прикладная математика и информатика» и заслуживает оценки «\textbf{отлично}».

\vspace{18pt}
\begin{flushright}
    \begin{tabular}[t]{p{5cm}r}
        Рецензент & \\
        профессор, д.\,ф--м.\,н & \underscore{4}~Баландин Д.\,В. \\
        & <<\underscore{0.5}>>~\underscore{2}~2011 г.
    \end{tabular}
\end{flushright}

\end{document}

