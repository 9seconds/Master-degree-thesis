% coding: utf-8
% --------------------------------------------------------------------------------------------------
% "Синтез оптимального стохастического управления", 2011 год
% --------------------------------------------------------------------------------------------------
\documentclass[12pt, a4paper, titlepage, openbib, oneside]{report}

\usepackage[utf8]{inputenc}
\usepackage[russian]{babel}
%\usepackage{indentfirst}
\usepackage{amsmath}
\usepackage{amssymb}
\usepackage{theorem}
\usepackage[russian]{varioref}
\usepackage[pdftex]{graphicx}
\usepackage{listings}
\usepackage[font=it,labelsep=period,format=hang]{caption}


\frenchspacing
\fussy
\raggedbottom

\setlength{\leftmargin}{20mm}
\setlength{\rightmargin}{10mm}
\righthyphenmin=2
\emergencystretch=6pt


\newtheorem{teo}{Теорема}
\newtheorem{statement}{Утверждение}
\newtheorem{lemma}{Лемма}
\newtheorem{alg}{Алгоритм}
\newtheorem{df}{Определение}


\DeclareMathOperator{\diag}{\mathrm{diag}}
\DeclareMathOperator{\tr}{\mathrm{tr}}




%\includeonly{ch2-controlSynthesis,ch4-solutionExample}




\begin{document}

\renewcommand{\bibname}{Список использованной литературы}
\renewcommand{\chaptername}{Раздел}




% --------------------------------------------------------------------------------------------------
% команды
% --------------------------------------------------------------------------------------------------



% перевод строки
\newcommand{\br}{ \vspace{12pt} }

% равно по определению
\newcommand{\eqdef}{ \stackrel{ \mathrm{def} }{=} }

% пространство R в степени x
\newcommand{\Rv}[1]{ \mathbb{R}^{#1} }

% матричное пространство R в степени x на y
\newcommand{\Rs}[2]{ \mathbb{R}^{{#1} \times {#2}} }

% математическое ожидание
\newcommand{\E}{ \mathbf{E} }

% вероятность
\newcommand{\prob}{ \mathbf{P} }

% эпсилон
\newcommand{\eps}{ \varepsilon }

% сделать текст полужирным
\newcommand{\strong}[1]{{ \fontseries{bx}\selectfont {#1}} }

% простая производная
\newcommand{\genericdiff}[2]{ \frac{d #1}{d #2} }

% простая строчная производная
\newcommand{\genericlinediff}[2]{ d #1 / d #2 }

% частная производная
\newcommand{\partdiff}[2]{ \frac{\partial {#1}}{\partial {#2}} }

% частная строчная производная
\newcommand{\partlinediff}[2]{ \partial {#1} / \partial {#2} }

% вторая частная производная
\newcommand{\partdiffsecond}[3]{ \frac{\partial^2 {#1}}{\partial {#2}\,\partial {#3}} }

% квадратик qed
\newcommand{\qed}{ \vspace{12pt}\hfill \mbox{\raggedright \rule{.1in}{.1in}} }

% каллиграфическое начертание
\newcommand{\calf}[1]{ \mathcal{#1} }

% прямое начертание
\newcommand{\sff}[1]{ \mathsf{#1} }

% полужирная капитель
\newcommand{\bff}[1]{ \mathbf{#1} }

% код
\newcommand{\code}[1]{\texttt{#1}}

% вектор или матрица
\newcommand{\m}[1]{ \bff{#1} }

% признак оптимальности
\newcommand{\optimum}[1]{ #1^{*} }

% сокращение для лейбла
\newcommand{\lbl}[1]{ \label{#1} }

% начало уравнения
\newcommand{\beq}[1]{ \begin{equation}\lbl{#1} }

% конец уравнения
\newcommand{\eeq}{ \end{equation} }

% начало уравнения без номера
\newcommand{\beqn}{ \begin{equation*} }

% начало блока уравнений
\newcommand{\beqarr}{ \begin{eqnarray} }

% конец блока уравнений
\newcommand{\eeqarr}{ \end{eqnarray} }

% конец уравнения без номера
\newcommand{\eeqn}{ \end{equation*} }

% начало нумерованного списка
\newcommand{\benum}{ \begin{enumerate} }

% конец нумерованного списка
\newcommand{\eenum}{ \end{enumerate} }

% начало теоремы
\newcommand{\bteo}[1]{ \begin{teo}\lbl{#1} }

% конец теоремы
\newcommand{\eteo}{ \end{teo} }

% начало алгоритма
\newcommand{\balgo}[1]{ \begin{alg}\lbl{#1}~ }

% конец алгоритма
\newcommand{\ealgo}{ \end{alg} }

% начало рисунка
\newcommand{\bfig}{ \begin{figure}\center }

% конец рисунка
\newcommand{\efig}{ \end{figure} }

% вставка рисунка
\newcommand{\addfigure}[3]{ \bfig \input{fig_#2} \caption{#3} \lbl{#1} \efig }

% центральная точка
\newcommand{\emptyarg}{ \,\cdotp }

% система уравнений
\newcommand{\eqsystem}[1]{ \begin{cases} #1 \end{cases} }

% матрица
\newcommand{\matr}[1]{ \begin{bmatrix} #1 \end{bmatrix} }

% вставка командной строки
\newcommand{\cli}[1]{ \begin{center} \code{#1} \end{center} }


% --------------------------------------------------------------------------------------------------
% страницы и главы
% --------------------------------------------------------------------------------------------------



% титульная страница
% (готово)
% coding: utf-8
% --------------------------------------------------------------------------------------------------
% "Численные методы аппроксимации гибридных диффузионных процессов", 2010
% : титульный лист
% --------------------------------------------------------------------------------------------------

\begin{titlepage}
	\begin{center}
		\par Федеральное агентство по образованию\vspace{0.7mm}
		\par Государственное образовательное учреждение\vspace{0.7mm}
		\par высшего профессионального образования\vspace{1.5mm}
		\par «Нижегородский Государственный университет им.~Н.\,И.\,Лобачевского»
		\vspace{1cm}
		{\fontsize{18pt}{9mm} \fontseries{bx} \selectfont
            \par О проблемах синтеза оптимального регулятора с обратной связью в многомерных стохастических системах
        }
        \vspace{2cm}
        \par {\large Магистерская диссертация}
    \end{center}
    \vspace{2cm}
    \begin{tabbing}
        1-2-3-4-5-6-7-8-9-10-11-12\=-13-14-15-16-17-18-19-20-21-22-23-24-25\=-26-27-28-29 \kill
        {\fontsize{10pt}{10pt} \selectfont Выполнил:}                 \>                      \>                 \\
        {\fontsize{10pt}{10pt} \selectfont студент гр.~86--М1}        \>                      \> {\fontsize{10pt}{10pt} \selectfont Архипов С.\,В.}  \\
        \\
        {\fontsize{10pt}{10pt} \selectfont Научный руководитель:}     \>                      \>                 \\
        {\fontsize{10pt}{10pt} \selectfont профессор, д.\,ф.--м.\,н.} \>                      \> {\fontsize{10pt}{10pt} \selectfont Пакшин П.\,В.}   \\
        \\
        {\fontsize{10pt}{10pt} \selectfont Заведующий кафедрой:}     \>                      \>                 \\
        {\fontsize{10pt}{10pt} \selectfont профессор, д.\,ф.--м.\,н.} \>                      \> {\fontsize{10pt}{10pt} \selectfont Федоткин М.\,А.}
    \end{tabbing}
    \vspace{3cm}
    \begin{center} {\fontsize{10pt}{3mm} \fontshape{sc} \selectfont
        Нижний Новгород \\
        2011 год
    } \end{center}
\end{titlepage}



% краткое описание рассматриваемой темы
% coding: utf-8
% --------------------------------------------------------------------------------------------------
% "Синтез оптимального стохастического управления", 2011 год
% --------------------------------------------------------------------------------------------------



\begin{abstract}

В работе исследуются вопросы синтеза и конструирования оптимального регулятора, используя принцип комбинированного управления, для стохастических систем, описываемых системой линейных дифференциальных уравнений и заданным критерием качества. Дается математическая постановка задачи, приведены результаты исследования проблемы для детерминированных систем, установлена связь с принципами максимума и методами динамического программирования, в частности с уравнениями Беллмана. Полученные результаты обобщены на случай многомерных стохастических систем, введено описание случайных возмущений, изучен характер стохастической устойчивости.

Результаты теоретического исследования использованы при решении конкретной типовой задачи наведения баллистической ракеты на заданный курс в условиях турбулентной атмосферы. Были получены алгоритмы решения, приведено математическое обоснование того, что с их помощью можно синтезировать оптимальное управление. Алгоритмы были проанализированы, приведены результаты экспериментальных исследований.

Разработан программный комплекс, математический пакет, позволяющий программно моделировать регулятор и проводить компьютерные эксперименты. Этот пакет был использован для моделирования задачи наведения ракеты. Представлены результаты сравнения теоретических результатов с практическими. В работе также приведены способы получения программного пакета, который распространяется свободно; дано краткое руководство пользователя.

\end{abstract}


% содержание. генерируется автоматически
% coding: utf-8
% --------------------------------------------------------------------------------------------------
% "Численные методы аппроксимации гибридных диффузионных процессов", 2010
% : содержание
% --------------------------------------------------------------------------------------------------

\tableofcontents


% введение в тему
% coding: utf-8
% --------------------------------------------------------------------------------------------------
% "Синтез оптимального стохастического управления", 2011 год
% --------------------------------------------------------------------------------------------------



\chapter*{Введение}
\addcontentsline{toc}{chapter}{Введение}



Процессы управления происходят повсюду: и в живой, и в неживой природе: в биологических организмах, обществе, технике. Даже естественный отбор, являющийся основой учения Дарвина, благодаря которому одни особи исчезают, а другие выживают и воспроизводятся, также является своего рода процессом управления, протекающим в природе.

Общим для всех процессов управления, где бы они не протекали, является прием, хранение, преобразование информации и выработка, синтез, управления на ее основе. Осознание этой общности послужило предпосылкой к возникновению в конце сороковых годов XX века научного направления, названного Н.\,Винером \emph{кибернетикой}. Хотя управление человеческими коллективами с одной стороны, и техническими объектами~--- с другой, имеет много общего, коренные различия, которые существуют между этими <<объектами>>, делают необходимыми их раздельное рассмотрение.

В технике \emph{управлением} называют целенаправленное воздействие на какое-либо устройство или объект. Если управление осуществляет человек, то управление называют \emph{ручным}; \emph{автоматическое} управление осуществляется без непосредственного участия человека. Устройство (машина, агрегат, технологический процесс), состоянием которого нужно управлять, называется \emph{объектом управления} или \emph{управляемым объектом}. Целью управления таким объектом является поддержание заданного режима работы или вывод объекта на такой режим работы. Под \emph{заданным режимом} понимают изменение какого-либо параметра, характеризующего состояние объекта управления, по определенному закону.

Объект управления с взаимодействующим с ним управляющим устройством называют \emph{системой управления}. В простейших случаях систему автоматического управления называют \emph{системой автоматического регулирования} (САР), управляющее устройство~--- \emph{регулятором}, а объект управления~--- \emph{объектом регулирования} или \emph{регулируемым объектом}.

\newpage

Выделяют три основных принципа управления:

\bdescr
	\item[Принцип программного управления]
		Если об объекте точно известно, как зависит выходная переменная от управляющего воздействия, управление можно формировать как известную функцию времени. Такой принцип неприменим при управлении объектом, на который действуют заранее не известные возмущения, оказывающие существенное влияние на управляемую величину. Он также неприменим, если объект является нейтральным или неустойчивым, а система должна функционировать длительное время.
		
	\item[Принцип компенсации]
		При таком принципе управления определяются каким-либо образом действующие на систему управления возмущения и на их основе вырабатывается управляющее воздействие, которое полностью или частично компенсирует влияние возмущений на процесс управления. Также этот принцип называют \emph{способ управления по возмущению} или \emph{принципом Понселе}. Достоинством такого способа управления является принципиальная возмжность полной компенсации возмущающего воздействия.
		
	\item[Принцип обратной связи]
		Этот принцип также называется \emph{управлением по отклонению}. Это такой способ управления, при котором определяется отклонение текущего значения выходной переменной от требуемого значения, и на его основе формируется управляющее воздействие. Системы управления, основанные на этом способе, непременно содержат \emph{обратную связь}~--- канал связи, по которому информация об управляемой переменной поступает на управляющее устройство. Иногда этот принцип называют \emph{принципом Ползунова--Уатта}.
		
		Недостатком такого способа управления является принципиальная невозможность полной компенсации возмущающих воздействий. Это связано с тем, что при этом способе управляющее воздействие начинает вырабатываться и оказывать влияние на ход процесса только после того, как возмущение, начав действовать, вызывает отклонение управляемой величины от требуемого режима.
		
	\item[Принцип комбинированного управления]
		Этот принцип является попыткой совместить достоинства управлений по возмущению и по отклонению. Этот принцип используется в тех случаях, когда на систему действует много различных возмущений, но лишь немногие оказывают наибольшее влияние на работу системы уравления. В подобных случаях влияние превалирующего возмущения можно нейтрализовать, используя принцип компенсации, и нейтрализовать влияние остальных возмущений, используя принцип обратной связи.
\edescr

Задачей работы является изучение теоретической возможности управления объектами, используя принцип комбинированного управления, при учете множества различных случайных возмущений. Это могут быть как шум среды, так, например, механические воздействия друг на друга сразу нескольких объектов, принимающих участие в сложном движении. В результате работы выработаны некоторые алгоритмы, которые могут быть использованы при конструировании требуемого регулятора, отмечены как положительные стороны алгоритмов, так и отрицательные. Каждый алгоритм имеет под собой математическую основу, гарантирующую, что сконструированный таким образом регулятор будет обеспечивать требуемое в определенном смысле управление. Требуемое управление, иначе называемое \emph{оптимальным} говорит о том, что его использование позволит получить требуемый результат наилучшим среди прочих образом.

В процессе исследований был разработан некоторый математический пакет, позволяющий программно сконструировать подобный регулятор и проводить различные экспериментальные проверки с целью выявляения особенностей работы того или иного алгоритма. Пакет распространяется свободно, на принципах открытого сообщества.

\br

Работа устроена следующим образом:

\bdescr
	\item[Постановка задачи]
		Раздел посвящен математической формализации задачи синтеза оптимального регулятора. Дается как математическое описание системы (в случаях случайных воздействий и их отсутствия), так и принцип, по которому определяется оптимальность управления.
	
	\item[Синтез оптимального управления]
		Раздел посвящен методам синтеза оптимального управления в различных случаях без случайных воздействий. Дается математическое обоснование полученных результатов, как в общем случае, так и в частном, наиболее распространенном случае с квадратичным критерием качества. Результаты связываются с принципом Беллмана и принципом максимума.
		
	\item[Методы стохастического управления]
		Раздел посвящен обобщению результатов, полученных в разделе <<\emph{Синтез оптимального управления}>> на случаях стохастических систем, то есть систем, где учитываются случайные воздействия. Дополнительно раскрывается вопрос стохастической устойчивости полученных теоретических результатов.
		
	\item[Пример решения типовой задачи]
		Раздел посвящен решению конкретной типовой задачи, в которой используются результаты из раздела <<\emph{Методы стохастического управления}>>. Дается описание задачи (полет баллистической ракеты в турбулентной атмосфере), конструируются алгоритмы ее решения, дается описание созданного для ее исследования программного математического пакета, анализируются полученные результаты, дается их обоснование.
\edescr


% математическая постановка задачи.
% coding: utf-8
% --------------------------------------------------------------------------------------------------
% "Синтез оптимального стохастического управления", 2011 год
% --------------------------------------------------------------------------------------------------



\chapter{Постановка задачи}
% ==============================================================================================
\newcommand{\funcF}{ \calf{F}         } % функция качества
\newcommand{\optF}{  \optimum{\funcF} } % оптимальная функция F
% ==============================================================================================



Предположим, что имеется некоторый физический объект, чья динамическое состояние описывается с помощью вектора параметров, полностью определяющих его положение в некотором пространстве и достаточных для того, чтобы однозначно определить его положение в следующий момент времени. Динамику его перемещений можно описать с помощью системы дифференциальных уравнений. Не нарушая общности положим, что динамика описывается с помощью системы линейных дифференциальных уравнений

\beq{eq:1/1}
    \eqsystem{
        \dot{x}_1 = a_{11}(t)x_1(t) + a_{12}(t)x_2(t) + \cdots + a_{1n}(t)x_n(t) \text {,} \\
        \dot{x}_2 = a_{21}(t)x_1(t) + a_{22}(t)x_2(t) + \cdots + a_{2n}(t)x_n(t) \text {,} \\
        \vdots                                                                             \\
        \dot{x}_n = a_{n1}(t)x_1(t) + a_{n2}(t)x_2(t) + \cdots + a_{nn}(t)x_n(t) \text {.} \\
    }
\eeq

В матричном виде уравнение~\ref{eq:1/1} можно переписать следующим образом:

\beq{eq:1/2}
    \dot{\m{x}}(t) = \m{A}(t)\m{x}(t) \text{,}
\eeq

где $\m{x}(t) \eqdef (x_1(t), x_2(t), \ldots, x_n(t)) \in \Rv{n}$, а матрица $\m{A}(t) \in \Rs{n}{n}$ и представляется как

\beqn
    \m{A}(t) \eqdef \matr{
        a_{11}(t) & a_{12}(t) & \cdots & a_{1n}(t) \\
        a_{21}(t) & a_{22}(t) & \cdots & a_{2n}(t) \\
        \vdots    & \vdots    & \ddots & \vdots    \\
        a_{n1}(t) & a_{n2}(t) & \cdots & a_{nn}(t)
    } \text{.}
\eeqn

Предположим также, что имеется некоторый способ воздействовать на динамику системы, чтобы было возможным перевести ее из начального состояния $t_0$ в конечное состояние $T$ так, что $\m{x}(T)$ удовлетворяла бы ряду заранее заданных условий. Такие воздействия называются \emph{управлением} и моделируются с помощью вектора $\m{u}(t) \eqdef (u_1(t), u_2(t), \ldots u_m(t)) \in \Rv{m}$.

Вводя управление, систему~\ref{eq:1/2} можно переписать следующим образом:

\beq{eq:1/3}
    \dot{\m{x}}(t) = \m{A}(t)\m{x}(t) + \m{B}(t)\m{u}(t) \text{,}
\eeq

где матрица $\m{B}(t) \in \Rs{n}{m}$ и представляется как

\beqn
    \m{B}(t) \eqdef \matr{
        b_{11}(t) & b_{12}(t) & \cdots & b_{1m}(t) \\
        b_{21}(t) & b_{22}(t) & \cdots & b_{2m}(t) \\
        \vdots    & \vdots    & \ddots & \vdots    \\
        b_{n1}(t) & b_{n2}(t) & \cdots & b_{nm}(t)
    } \text{.}
\eeqn

В работе, не нарушая общности, будут рассматриваться матрицы $\m{B}(t)$ с постоянными значениями, то есть $\m{B}(t) \equiv \m{B}$. Такие предположения сделаны лишь для того, чтобы упростить выкладки.

Используя различное управление, система будет переходить в различные конечные состояния $\m{x}(T)$: используя управление $\m{u}^1(t)$, система будет переведена в конечное состояние $\m{x}^1(t)$, управление $\m{u}^2(t)$~--- в $\m{x}^2(T)$ и так далее. Как правило, имеет значение, в котором окажется система в конечный момент времени. Управление\footnote{Конечно же, оптимальное управление в общем случае не является единственным.}, которое переводит систему в требуемое состояние $\optimum{\m{x}}(t)$ будет называться \emph{оптимальным} и обозначаться соответственно $\optimum{\m{u}}(t)$. Траекторию изменения состояний от момента времени $t_0$ до $T$ также назовем \emph{оптимальной}.

\br

Зачастую имеет значение не только факт, что в определенный момент времени система перешла в нужное состояние, но и то, как был совершен этот переход. Иными словами, требуется учитывать предысторию перехода, всю динамику системы с использованием управлений. Чтобы учитывать такое <<качество>> перехода, вводятся различные \emph{критерии качества}~--- функции, выражающие качество перехода через некоторые количественные значения. Так как для вещественных чисел определены операции сравнения, то появляется возможность таким образом сравнивать траектории и выделять оптимальное уже в этом случае управление. Иногда даже не имеет значения, в каком конечном состоянии окажется система, важно, чтобы переход осуществлялся так, чтобы он удовлетворял некоторому критерию качества сам по себе.

Естественным образом возникает задача минимизации (или максимизации) такого критерия качества $\funcF\bigl(\m{x}(t), \m{u}(t)\bigr): \Rv{n} \times \Rv{m} \to \Rv{1}$. Для определенности рассмотрим задачу минимизации критерия. Минимальное значение $\min\funcF\bigl(\m{x}(t), \m{u}(t)\bigr)$ обозначим как $\optF\bigl(\m{x}(t), \m{u}(t)\bigr)$.

При этих предположениях, задачу можно сформулировать таким образом: требется найти управление $\optimum{\m{u}}(t)$ для системы~\ref{eq:1/2} и заданного критерия качества $\funcF$ при условии

\beq{eq:1/4}
    \funcF\bigl(\m{x}(t), \m{u}(t)\bigr) \to \min \text{.}
\eeq

Найденное таким образом управление также будет называться оптимальным.

\br

Однако система~\ref{eq:1/3} помимо воздействия управления может подвергаться и воздействиям различных внешних факторов, которые нельзя проконтроллировать. Воздействие этих факторов непредсказуемо, они оказывают влияние на поведение объекта на протяжение всего интервала наблюдений $[t_0, T]$. Такие случайные воздействия возникают, например, вследствие разнообразных шумов среды, столкновений с другими объектами или скачков энергии. Факторов может быть огромное количество, но при моделировании их можно объединять в группы. Эти воздействия описываются случайными величинами, в частности векторными. В работе будет исследоваться влияние векторной случайной величины $\m{\xi}(t) = \bigl( \xi_1(t), \xi_2(t), \ldots, \xi_k(t) \bigr) \in \Rv{k}$.

Таким образом, задачу~\ref{eq:1/3}--\ref{eq:1/4} можно переписать следующим образом:

\beq{eq:1/5}
    \dot{\m{x}}(t) = \m{A}(t)\m{x}(t) + \m{B}(t)\m{u}(t) + \m{\Sigma}(t)\m{\xi}(t) \text{,}
\eeq

где $\m{\Sigma}(t) \in \Rs{n}{k}$ и записывается как

\beqn
    \m{\Sigma}(t) \eqdef \matr{
        \sigma_{11}(t) & \sigma_{12}(t) & \cdots & \sigma_{1k}(t) \\
        \sigma_{21}(t) & \sigma_{22}(t) & \cdots & \sigma_{2k}(t) \\
        \vdots    & \vdots    & \ddots & \vdots    \\
        \sigma_{n1}(t) & \sigma_{n2}(t) & \cdots & \sigma_{nk}(t)
    } \text{.}
\eeqn

Впрочем, не нарушая общности, как и в случае для матрицы $\m{B}(t)$ будем полагать, что матрица $\m{\Sigma}(t)$ является постоянной, то есть $\m{\Sigma}(t) \equiv \m{\Sigma}$. Критерий качества, очевидно, остается прежним, поскольку факторы, которые нельзя проконтроллировать, влияют на саму траекторию системы.

Нахождение оптимальных управлений $\optimum{\m{u}}(t)$ для системы~\ref{eq:1/5} и задачи минимизации~\ref{eq:1/4} и является задачей этой работы.

% синтез оптимального управления
\chapter{Синтез оптимального управления}
% ==============================================================================================
\newcommand{\funcF}{ \calf{F}           } % функция качества
\newcommand{\optF}{  \optimum{\funcF}   } % оптимальная функция F
\newcommand{\optU}{  \optimum{u}        } % оптимальное управление u
\newcommand{\optX}{  \optimum{x}        } % оптимальная траектория x
\newcommand{\funcL}{ \sff{L}            } % функция Лагранжа
\newcommand{\funcT}{ \optimum{\sff{F}}  } % передаточная функция F
\newcommand{\funcV}{ \sff{V}            } % функция Ляпунова
\newcommand{\funcH}{ \optimum{H}        } % функция Гамильтона

\newcommand{\mA}{    \bff{A}            } % матрица A
\newcommand{\mB}{    \bff{B}            } % матрица B
\newcommand{\mM}{    \bff{M}            } % матрица M
\newcommand{\mQ}{    \bff{Q}            } % матрица Q
\newcommand{\mR}{    \bff{R}            } % матрица R
\newcommand{\mP}{    \bff{P}            } % матрица P
\newcommand{\mH}{    \bff{H}            } % матрица H
\newcommand{\mD}{    \bff{D}            } % матрица D
\newcommand{\mC}{    \bff{C}            } % матрица C
\newcommand{\mLa}{   \bff{\Lambda}      } % матрица Лямбда
\newcommand{\mI}{    \bff{I}            } % единичная матрица I
\newcommand{\mb}{    \bff{b}            } % матрица-вектор b
\newcommand{\mk}{    \bff{k}            } % матрица-вектор k
% ==============================================================================================




% **********************************************************************************************
\section{Принцип Беллмана}
% **********************************************************************************************



Рассмотрим следующую задачу Лагранжа. Для системы $\dot{x} = f(x, u, t)$, начинающей движение из начального состояния $x(t_1) = x_1$, найдем управление $u(t)$, ограниченное некоторым допустимым классом функций $\Omega(t)$ и определенное на интервале времени $[t_1, t_2]$, которое минимизирует показатель качества

\beq{eq:2/1/1}
    \funcF = \int\limits_{t_1}^{t_2} \funcL(x, u, t)\,dt \mbox{,}
\eeq

где функция $\funcL(x, u, t)$ предполагается непрерывной по $t$. Конечное состояние $x(t_2) = x_2$ не задано. Все функции, принадлежащие $\Omega(t)$ ограничены по амплитуде в любой момент времени $t$: $|u_j(t)| \leqslant V_j$. Множество точек $|u_j(t)| \leqslant V_j$ ($j = 1, 2, \ldots, n$) будем обозначать как $\Upsilon$.

Показатель качества $\funcF$ для фиксированного значения $t_2$ зависит от переменных $u(t)$, $x_1$ и $t_1$. Однако оптимальное значение функционала $\funcF$ зависит лишь от начального состояния $x_1$ и момента времени $t_1$. Обозначим оптимальное значение этого функционала как $\optF(x_1, t_1)$. Если поставлена задача оптимального управления, то для каждой точки $x_1$ ставится в соответствие некоторое значение критерия оптимальности. Функция $\optF(x_1, t_1)$ определяет гиперповерхность в $n+1$-мерном пространстве. Эта гиперповерхность, конечно, в явном виде не задана, но некоторые ее свойства позволяют вывести ряд необходимых условий, которым должна удовлетворять функция оптимального управления $\optU(t)$.

Обозначим через $x \bigl( u(t), t \bigr)$ траекторию, которая получается в результате воздействия управления $u(t)$ на систему $\dot{x} = f(x, u, t)$ с начальным состоянием $x_1$ в момент времени $t = t_1$. При этом $\optF$ определяется выражением

\beq{eq:2/1/2}
    \optF(x_1, t_1) = \underset{\substack{u(t) \in \Omega(t) \\ t_1 \leqslant t \leqslant t_2}}{\min} \biggl\{ \int\limits_{t_1}^{t_2} \funcL\bigl(x(u, t), u, t\bigr)\,dt \biggr\} \mbox{.}
\eeq

Для некоторого момента времени $t'$ из интервала между $t_1$ и $t_2$ это выражение можно написать в таком виде:

\beq{eq:2/1/3}
    \optF(x_1, t_1) = \underset{\substack{u(t) \in \Omega(t) \\ t_1 \leqslant t \leqslant t_2}}{\min} \biggl\{ \int\limits_{t_1}^{t'} \funcL\bigl(x(u, t), u, t\bigr)\,dt + \int\limits_{t'}^{t_2} \funcL\bigl(x(u, t), u, t\bigr)\,dt \biggr\} \mbox{.}
\eeq

Уравнение~\ref{eq:2/1/3} позволяет применить для его решения принцип оптимальности. Для случая непрерывных систем принцип оптимальности можно сформулировать следующим образом.

\begin{statement}\lbl{statement:1}
	Оптимальное управление $\optU(t)$ на интервале времени $[t_1, t_2]$ имеет следующее свойство: для любого $t'$, заключенного в интервале $t_1 < t' < t_2$ независимо от значений, которые управление $\optU(t)$ принимало на интервале времени $[t_1, t']$, и, следовательно, независимо от значения $\optX(t')$ оно должно оставаться оптимальным управлением относительно состояния $\optX(t')$ на интервале времени $(t', t_2]$.
\end{statement}

Применяя этот принцип, уравнение~\ref{eq:2/1/3} можно преобразовать к следующему виду

\beq{eq:2/1/4}
    \optF(x_1, t_1) = \underset{\substack{u(t) \in \Omega(t) \\ t_1 \leqslant t \leqslant t'}}{\min} \biggl\{ \int\limits_{t_1}^{t'} \funcL\bigl(x(u, t), u, t\bigr)\,dt + \optF\bigl(x(t'), t'\bigr) \biggr\} \mbox{,}
\eeq

где $x(t')$~--- конечное состояние, которое является результатом действия управления $u(t)$ на интервале времени $[t_1, t']$\footnote{ Если предположить, что функция $\funcF$ имеет непрерывные частные производные по $x$ и $t$, можно легко прийти к уравнению Беллмана (см TODO). Однако, как будет показано ниже, это предположение не выполняется для большого класса задач оптимального управления. В приводимом здесь доказательстве, основанном на работе TODO, это предположение не используется }.

При оптимальном управлении $u(t) = \optU(t)$ на интервале $[t_1, t']$ имеем

\beq{eq:2/1/5}
    \optF(x_1, t_1) = \int\limits_{t_1}^{t'} \funcL\bigl(x(\optU, t), \optU(t), t\bigr)\,dt + \optF\bigl(x(t'), t'\bigr) \mbox{.}
\eeq

Перенося члены и деля на $t' - t_1$ получим

\beq{eq:2/1/6}
    -\frac{\optF\bigl( x(t'), t' \bigr) - \optF( x_1, t_1)}{t' - t_1} = \frac{1}{t' - t_1} \int\limits_{t_1}^{t'} \funcL\bigl(x(\optU, t), \optU(t), t\bigr)\,dt \mbox{.}
\eeq

При $t' \to t_1$ уравнение~\ref{eq:2/1/6} принимает вид

\beq{eq:2/1/7}
    \lim_{t' \to t_1} \biggr[ -\frac{\optF\bigl( x(t'), t' \bigr) - \optF( x_1, t_1)}{t' - t_1} \biggr] = \funcL\bigl(x_1, \optU(t_1), t_1\bigr) \mbox{.}
\eeq

Заметим, что для получения правой части выражения~\ref{eq:2/1/7} как предела правой части уравнения~\ref{eq:2/1/6} используется теорема о среднем значении (TODO). Если предел левой части уравнения~\ref{eq:2/1/7} существует, то можно определить величину

\beq{eq:2/1/8}
    \biggl[ \genericdiff{\optF}{t} \biggr]_{\optU, t_1} \eqdef \lim_{t' \to t_1} \biggr[ -\frac{\optF\bigl( x(t'), t' \bigr) - \optF( x_1, t_1)}{t' - t_1} \biggr] \mbox{.}
\eeq

Величина $[ \genericlinediff{\optF}{t} ]_{\optU, t_1}$ есть производная функции $\optF$ по времени, вычисленная в момент времени $t_1$. Анализируя правую часть выражения~\ref{eq:2/1/8}, можно видеть, что производную следует вычислять вдоль траектории, обусловленной управлением $\optU$ и начинающейся в $x_1$. Иными словами, пусть для некоторой задачи функция $\optF(x, t)$ определена для любого начального состояния $x$ и начального момента времени $t$. Пусть $x\bigl( u(t), t \bigr)$ определяет траекторию, обусловленную управлением $u(t)$; тогда вдоль любой из этих траекторий величина $\optF(x, t)$ будет изменяться во времени со скоростью, определяемой величиной $[ \genericlinediff{\optF}{t} ]_{u, t}$. В частности, вдоль оптимальной траектории она будет изменяться со скоростью $[ \genericlinediff{\optF}{t} ]_{\optU, t}$.

Таким образом

\beq{eq:2/1/9}
    \biggl[ \genericdiff{\optF}{t} \biggr]_{\optU, t_1} + \funcL\bigl(x_1, \optU(t_1), t_1\bigr) = 0 \mbox{.}
\eeq

Заметим, что при $u(t) \neq \optU(t)$ для интеграла~\ref{eq:2/1/5} в соответствии с определением должно выполняться неравенство

\beq{eq:2/1/10}
    \optF(x_1, t_1) \leqslant \int\limits_{t_1}^{t'} \funcL\bigl(x(\optU, t), \optU(t), t\bigr)\,dt + \optF\bigl(x(t'), t'\bigr) \mbox{.}
\eeq

Откуда следует, что

\beq{eq:2/1/11}
    \biggl[ \genericdiff{\optF}{t} \biggr]_{u, t_1} + \funcL\bigl(x_1, u(t_1), t_1\bigr) \geqslant 0 \mbox{.}
\eeq

Используя выражения~\vref{eq:2/1/9} и~\ref{eq:2/1/11}, получим

\beq{eq:2/1/12}
    \biggl[ \genericdiff{\optF}{t} \biggr]_{u, t_1} + \funcL\bigl(x_1, u(t_1), t_1\bigr) \geqslant \biggl[ \genericdiff{\optF}{t} \biggr]_{\optU, t_1} + \funcL\bigl(x_1, \optU(t_1), t_1\bigr) \mbox{,}
\eeq

или

\beq{eq:2/1/13}
    \underset{u(t_1) \in \Upsilon}{\min} \biggl\{ \biggl[ \genericdiff{\optF}{t} \biggr]_{u, t_1} + \funcL\bigl(x_1, u(t_1), t_1\bigr) \biggr\} = 0 \mbox{.}
\eeq

Уравнение~\ref{eq:2/1/13} справедливо в любой момент времени из интервала $[t_1, t_2]$, так что для любого момента времени $t$ и любого состояния $x$ на траектории $\optX(t)$, принимая его за начальное, можно написать

\beq{eq:2/1/14}
    \underset{u(t) \in \Upsilon}{\min} \biggl\{ \biggl[ \genericdiff{\optF}{t} \biggr]_{u, t} + \funcL\bigl(x(u, t), u, t\bigr) \biggr\} = 0 \mbox{.}
\eeq

Уравнение~\ref{eq:2/1/14} и есть функциональное уравнение Беллмана в общей форме, выражающее необходимое условие оптимальности.

Для любого состояния $x$ и момента времени $t$, когда функция $\genericlinediff{\optF}{t}$, $\genericlinediff{\optF}{x}$ и $f(x, u, t)$ непрерывны по $x$ и $t$, полную производную $[\genericlinediff{\optF}{t}]_{u, t}$ можно записать так:

\beq{eq:2/1/15}
    \biggl[ \genericdiff{\optF}{t} \biggr]_{u, t} = \biggl(\partdiff{\optF}{x}\biggr)^T f(x, u, t) + \partdiff{\optF}{t} \mbox{.}
\eeq

Так как частная производная $\partlinediff{\optF}{t}$ не зависит от $u$, уравнение \ref{eq:2/1/14} можно представить в виде

\beq{eq:2/1/16}
    - \partdiff{\optF}{t} = \underset{u(t) \in \Upsilon}{\min} \Biggl\{ \funcL\bigl(x(u, t), u, t\bigr) + \biggl(\partdiff{\optF}{x}\biggr)^T f(x, u, t) \Biggr\} \mbox{.}
\eeq

В тех случаях, когда это уравнение применимо, оно обеспечивает необходимое условие оптимальности. Отметим, что оно является необычной формой дифференциального уравнения в частных производных, которое включает операцию минимизации. Это, в общем случае нелинейное, дифференциальное уравнение в частных производных первого порядка относительно одной переменной $\optF$. Это уравнение определяет одноточечную краевую задачу с граничным условием вида

\beq{eq:2/1/17}
    \lim_{t_1 \to t_2} \optF(x_1, t_1) = \lim_{t_1 \to t_2} \int\limits_{t_1}^{t_2} \funcL(\optX, \optU, t)\,dt = 0 \mbox{.}
\eeq

Очевидно, это условие соответствует двум.

\benum
    \item
        Оно указывает, как следует вести поиск оптимальной $u(t)$: в любой момент времени $t$ поиск должен быть таким, чтобы минимизировать величину, заключенную в квадратные скобки в уравнении~\vref{eq:2/1/16}.

    \item
        Если функция оптимального управления $\optU(t)$ найдена, то~\ref{eq:2/1/16} сводится к уравнению без операции минимизации
        \beq{eq:2/1/18}
            - \partdiff{\optF}{t} = \funcL\bigl(x(\optU(t), t), \optU, t\bigr) + \biggl(\partdiff{\optF}{x}\biggr)^T f(x(\optU(t), t), \optU, t) \mbox{,}
        \eeq 
        которому удовлетворяет функция $\optF(x, t)$ для всех значений $t$ в интервале $[t_1, t_2]$.
\eenum

При отсутствии ограничений на величину $u$ и при условии, что функции $\funcL$ и $f$ имеют частные производные по $u$, оптимальное управление $\optU(t)$ можно найти путем дифференцирования выражения, заключенного в квадратные скобки в уравнении~\ref{eq:2/1/16} и приравнивания полученного результата к нулю. Это дает условие

\beq{eq:2/1/19}
    \biggl[ \partdiff{\funcL}{u} + \sum\limits_{i=1}^n \partdiff{\optF}{x_i} \centerdot \partdiff{f_i}{u} \biggr] = 0 \mbox{ для всех } t \in [t_1, t_2] \mbox{.}
\eeq

Если показатель качества соответствует задаче Майера, а именно $\funcF = P\bigl( x(t_2), t_2 \bigr)$, то, придерживаясь тех же рассуждений, получим соответствующее уравнение Беллмана

\beq{eq:2/1/20}
    \underset{u(t) \in \Upsilon}{\min} \Biggl\{ \biggl[ \genericdiff{\optF}{t} \biggr]_{u, t} \Biggr\} = 0
\eeq

с граничным условием

\beq{eq:2/1/21}
    \lim_{t_1 \to t_2} \optF\bigl(x(t_1), t_1\bigr) = P\bigl(x(t_2), t_2\bigr) \mbox{.}
\eeq

В тех же случаях, когда функции $\partlinediff{\optF}{t}$, $\partlinediff{\optF}{x}$ и $f(x, u, t)$ непрерывны по $x$ и $t$, имеем

\beq{eq:2/1/22}
    - \partdiff{\optF}{t} = \underset{u(t) \in \Upsilon}{\min} \Biggl\{ \biggl(\partdiff{\optF}{x}\biggr)^T f(x, u, t) \Biggr\}
\eeq

с тем же граничным условием~\ref{eq:2/1/21}.

\br

Отметим, что тип граничного условия для конечного состояния не играет роли при выводе функционального уравнения. Таким образом, уравнение Беллмана~\ref{eq:2/1/14} или~\vref{eq:2/1/16} справедливо для задачи, даже если конечное состояние $x_2$ задано, а конечный момент времени $t_2$ нет. Условие~\vref{eq:2/1/17} по-прежнему справедливо, и нет необходимости вносить какие-либо изменения при выводе уравнения Беллмана.



% **********************************************************************************************
\section{Синтез оптимального управления в системах с квадратичным критерием качества}
% **********************************************************************************************



Как было показано в предыдущем разделе, динамическое программирование обладает тем свойством, что оптимальная функция управления $\optU$ обычно записывается в виде функции переменных состояния $x$, и, следовательно, синтезируется система с обратной связью. Однако использование динамического программирования требует решения уравнения Беллмана. Задачу можно решать двумя путями:

\benum
    \item
        Получить $2n$ дифференциальных уравнений с двухточечными граничными условиями с помощью вариационного исчисления или принципа максимума.
        
    \item
        Получить одно дифференциальное уравнение в частных производных с граничными условиями в одной точке
\eenum

Преимущество второго метода над первым не очевидно. Однако есть ряд задач о синтезе регулятора, где второй метод оказывается удобнее и практичнее. Сформулируем следующую задачу\footnote{ Более детально задачи о синтезе регуляторов такого рода рассматривали Калман и Мэрриэм (TODO) }.

Пусть дан линейный нестационарный объект управления, который описывается уравнением $\dot{x} = \mA(t)x + \mB(t)u$ с начальным условием $x(t_1) = x_1$ и с незаданным конечным состоянием $x(t_2)$.. Необходимо найти\footnote{ Необходимость использовать этот критерий возникает в тех случаях, когда требуется существование второй вариации от заданного функционала. } оптимальную функцию управления $\optU(t)$ на интервале $[t_1, t_2]$, минимизирующую функционал

\beq{eq:2/2/1}
    \funcF = \frac{1}{2} x_2^T \mM x_2 + \frac{1}{2} \int\limits_{t_1}^{t_2} \bigl[ x^T(\tau)\mQ(\tau)x(\tau) + u^T(\tau)\mR(\tau)u(\tau) \bigr]\,d\tau \mbox{,}
\eeq

где $\mM \in \Rs{n}{n}$, $\mQ(t) \in \Rs{m}{m}$ и $\mR(t) \in \Rs{n}{n}$~--- симметричные матрицы, причем $\mQ(t)$ и $\mR(t)$ являются положительно определенными и обладают непрерывными вторыми производными по $t$; $\mM$~--- знакоположительная постоянная матрица.

Сформулированная таким образом задача есть задача Больца (TODO). Уравнение Беллмана в этом случае имеет вид

\beq{eq:2/2/2}
\begin{split}
    - \partdiff{\optF}{t} =& \frac{1}{2} x^T(t)\mQ(t)x(t) + \frac{1}{2} \optU^T(t)\mR(t)\optU(t) + \\
    &+ \biggl(\partdiff{\optF}{x}\biggr)^T\bigl( \mA(t)x(t) + \mB(t)\optU(t) \bigr) \mbox{.}
\end{split}
\eeq

Граничное условие, вытекающее из уравнений~\vref{eq:2/1/17} и~\vref{eq:2/1/21}, сводится к

\beq{eq:2/2/3}
    \lim_{t \to t_2} \optF(x, t) = \frac{1}{2} x^T(t_2) \mM x(t_2) \mbox{.}
\eeq

Процесс минимизации приводит к условию (TODO на книжку умную)

\beq{eq:2/2/4}
    \partdiff{\calf{I}}{u} \biggm|_{u = \optU} + \frac{\partial}{\partial u} \Biggl[ \biggl(\partdiff{\optF}{x}\biggr)^T\bigl( \mA(t)x + \mB(t)u \bigr) \Biggr] \Biggm|_{u = \optU} = 0 \mbox{,}
\eeq

где $\calf{I}$~--- подынтегральное выражение в уравнении~\ref{eq:2/2/1}. Из последнего условия следует, что

\beq{eq:2/2/5}
    \optU(t) = -\mR^{-1}(t)\mB^T(t) \partdiff{\optF}{x} \mbox{.}
\eeq

Если хотим синтезировать линейное управление как функцию координат, то следует как критерий качества $\optF$ принять квадратичную форму

\beq{eq:2/2/6}
    \optF(x, t) = \frac{1}{2} x^T\mP(t)x \mbox{,}
\eeq

где $\mP(t) \in \Rs{n}{n}$~--- симметричная матрица.

Подставив выражения~\ref{eq:2/2/5} и~\ref{eq:2/2/6} в уравнение Беллмана~\ref{eq:2/2/2}, получим матричное уравнение Риккати

\beq{eq:2/2/7}
\begin{split}
    - \genericdiff{\mP(t)}{t} =& -\mP(t)\mB(t)\mR^{-1}(t)\mB^T(t)\mP(t) + \mP(t)\mA(t) + \\
    &+ \mA^T(t)\mP(t) + \mQ(t)
\end{split}
\eeq

с граничным условием

\beq{eq:2/2/8}
    \mP(t_2) = \mM \mbox{.}
\eeq

Матрицу $\mP(t)$ теоретически можно найти, интегрируя уравнение~\ref{eq:2/2/7} в обратном времени с начальным условием~\ref{eq:2/2/8}. После нахожденя матрицы $\mP(t)$, так как она симметрична, с помощью выражений~\vref{eq:2/2/5} и~\ref{eq:2/2/6} получим

\beq{eq:2/2/9}
    \optU(x, t) = -\mR^{-1}(t)\mB^T(t)\mP(t)x(t) \mbox{.}
\eeq

Таким образом, единственный абсолютный минимум функционала~\vref{eq:2/2/1} получается только в том случае, если матричное уравнение Риккати имеет одно решение. В отношении последнего Калман (TODO) доказал следующую теорему.

\begin{teo}\lbl{teo:1}
    В соответствии с принятыми предположениями, уравнение~\ref{eq:2/2/7} с граничным условием~\ref{eq:2/2/8} имеет единственное решение для матрицы $\mP(t)$, при этом управление~\ref{eq:2/2/9} является оптимальным по отношению к критерию оптимальности~\vref{eq:2/2/6}.
\end{teo}

Иными словами, уравнение Беллмана~\vref{eq:2/2/2} обеспечивает для этого случая достаточное условие оптимальности.

% TODO то, что дальше, засунуть в описание к задаче про ракету

\br

Калман (TODO) показал, что при $T \to \infty$ для полностью управляемой линейной стационарной системы и показателя качества

\beqn
    \funcF = \frac{1}{2} \int\limits_0^\infty \bigl[ x^T(\tau)\mQ x(\tau) + u^T(t)\mR u(\tau) \bigr]\,d\tau \mbox{,}
\eeqn

где $\mQ$ и $\mR$~--- симметричные, положительно определенные постоянные матрицы, справедливо условие $\lim\limits_{T \to \infty}\dot{\mP}(t) = 0$ и, таким образом, матричное уравнение Риккати~\ref{eq:2/2/7} сводится к нелинейному матричному алгебраическому уравнению

\beq{eq:2/2/10}
    -\mP\mB\mR^{-1}\mB^T\mP + \mP\mA + \mA^T\mP + \mQ = 0 \mbox{.}
\eeq

Решение этого уравнения определяет постоянную матрицу $\mP$. Оптимальное управлерние имеет вид $\optU(x, t) = -\mR^{-1}\mB^T\mP x$. Это уравнение означает, что все переменные состояния должны быть известны, то есть нужно точно измерять выходной сигнал и $n-1$ его производную. Это очень жесткое требование. Применительно к случаю одного управляющего воздействия, Калман (TODO) показал, что если в приведенной выше задаче уравнение~\ref{eq:2/2/10} определяет постоянную матрицу $\mP$, получается строгое решение в частотной области.

\br

Пусть объект, определяемый уравнением $\dot{x} = \mAx + \mb u$, полностью управляем, а показатель качества имеет вид

\beq{eq:2/2/11}
    \funcF = \frac{1}{2} \int\limits_0^\infty \bigl[ x^T(\tau)\mQ x(\tau) + ru^2(\tau) \bigr]\,d\tau \mbox{.}
\eeq

Полагаем $\mQ$ положительно определенной и $r > 0$. Тогда оптимальное управление~--- $\optU = -(1/r) \mb^T\mP x = -\mk^Tx$, где $\mk^T = \mb^T\mP / r$. Тогда передаточная функция разомкнутой оптимальной системы управления равна (TODO)

\beq{eq:2/2/12}
    \funcT(s) = k^T(s\mI - \mA)^{-1}\mb \mbox{.}
\eeq

\begin{teo}\lbl{teo:2}
    Функция $\funcF(s)$, определяемая уравнением~\ref{eq:2/2/12} обладает тем свойством, что при всех частотах $\omega$ справедливо, что
	\beq{eq:2/2/13}
    	\bigl| 1 + \funcT(j\omega) \bigr| > 1 \mbox{.}
	\eeq
\end{teo}

\strong{Доказательство.}

Заметим, что показатель качества~\ref{eq:2/2/11} можно нормировать так, что $r=1$. Тогда уравнение~\ref{eq:2/2/10} принимает вид\footnote{ В случае одного управляющего воздействия вектор $\mR$ в уравнении~\ref{eq:2/2/10} переходит в скаляр $r$. }

\beq{eq:2/2/14}
    - \mP\mA - \mA^T\mP = \mQ - \mP\mb \mb^T\mP \mbox{.}
\eeq

Прибавляя и вычитая $s\mP$, получим

\beq{eq:2/2/15}
    \mP(s\mI - \mA) + (-s\mI - \mA^T)\mP = \mQ - \mP\mb\mb^T\mP \mbox{.}
\eeq

Передаточная функция объекта от $u$ до $x$ определяется выражением $\mH(s) = (s\mI-\mA)^{-1}\mb$. Если обе стороны уравнения~\vref{eq:2/2/15} предварительно умножить на $\mH^T(-s)$, а затем на $\mH(s)$, то найдем

\beqn
    \mH^T(-s)\mP \mb + \mb^T\mP\mH(s) = \mH^T(-s)[\mQ - \mP\mb\mb^T\mP]\mH(s)
\eeqn

или

\beq{eq:2/2/16}
    \mH^T(-s)\mP\mb + \mH^T(-s)\mP\mb\mb^T\mP\mH(s) + \mb^T\mP\mH(s) = \mH^T(-s)\mQ\mH(s) \mbox{,}
\eeq

так как положительно определенную матрицу $Q$ можно представить в виде (TODO)

\beqn
    \mQ = \mD^T\mLa\mD \mbox{,}
\eeqn

где матрица $\mD$~--- действительная, а матрица $\mLa$~--- действительно, положительно определенная и диагональная.

Уравнение~\ref{eq:2/2/16} с учетом того, что $\mk = \mP\mb$, можно записать следующим образом:

\beq{eq:2/2/17}
    \bigl( 1+\mH^T(-s)\mk \bigr)\bigl( 1+\mk^T\mH(s) \bigr) = 1 + \mH^T(-s)\mD\mLa\mD\mH(s) \mbox{.}
\eeq

Ввиду того, что $\mk^T\mH(s) = \funcT(s)$, при $s = j\omega$ имеем

\beq{eq:2/2/18}
    \bigl| 1 + \funcT(j\omega) \bigr|^2 = 1 + \mH^T(-j\omega)\mD^T\mLa\mD\mH(j\omega) \mbox{.}
\eeq

Так как последний член уравнения~\vref{eq:2/2/15} всегда положителен, то из~\ref{eq:2/2/18} следует, что условие теоремы~\ref{eq:2/2/13} выполнено.\qed

Условие~\ref{eq:2/2/13} можно представить в виде единичной окружности с центром в точке $-1 + j0$, которую годограф $\funcT(j\omega)$ не должен касаться или входить в нее. Это значит, кто какой бы ни была первоначальная передаточная функция объекта $\sff{G}(s)$, оптимальный линейный регулятор путем компенсации будет превращать ее в $\funcT(j\omega)$ с указанными выше свойствами. Другими словами, вследствие строгости условия~\vref{eq:2/2/13} оптимальное уравнение трудно будет осуществить в случае нетривиальных управляемых объектов.

\br

Хотя приведенные выше результаты применимы лишь к системам без элемента сравнения, их можно очень легко распространить на системы с элементами сравнения и входными сигналами, принадлежащими определенному классу. Однако такое обобщение остается за рамками этой работы.



% **********************************************************************************************
\section{Связь с принципом максимума}
% **********************************************************************************************




Введем понятие принципа максимума. Пускай задана система $\dot{x} = f(x, u, t)$, с начальным условием $x(t_1) = x_1$. Необходимо выбрать такую функцию управления $u(t)$ на интервале $t \in [t_1, t_2]$, чтобы

\benum
	\item
		Изображающая точка была переведена из состояния $x_1$ в момент времени $t_1$ в такое состояни в момент времени $t_2$, чтобы первые $m$ ($m \leqslant n$) составляющий текущего вектора состояния $x$ ($x_1, x_2, \ldots, x_m$) совпадали с $m$ составляющих заданного конечного состояния $s_1, s_2, \ldots, s_m$ и чтобы функционал
		\beq{eq:2/3/1}
			\funcF = \int\limits_{t_1}^{t_2} \funcL(x,u,t)\,dt
		\eeq
		достигал бы минимума.
		
	\item
		Были учтены все ограничения, наложенные на $u(t)$; без потери общности можно считать, что $u(t)$ принимает значения внутри $r$-мерного единичного куба, определяемого условием $|u_i(t)| \leqslant 1$ ($i = 1,2,\ldots,r$).
\eenum

В приведенной формулировке некоторые переменные состояния остались свободными в конечный момент времени $t_2$. Полагаем, что функция $f(x, u, t)$ является непрерывной функцией от $u$, кусочно-непрерывной по $t$ и непрерывно дифференцируемой по $x$. Полагаем, что функция $\funcL$ обладает теми же свойствами.

Учтем формулу~\ref{eq:2/3/1} непосредственно в формулировке задачи. Это достигается путем введения дополнительной переменной состояния $x_0$, где

\beq{eq:2/3/2}
    \dot{x}_0 \equiv \funcL(x, u, t) \mbox{.}
\eeq

Определим два $(n+1)$-мерных вектора $x'$ и $f'$ соответственно с составляющими $x_0, x_1, \ldots, x_n$ и $\funcL, f_1, f_2, \ldots, f_n$. Уравнения для расширенной системы можно записать в таком виде

\beq{eq:2/3/3}
    \dot{x} = f'(x', u, t) \mbox{.}
\eeq

Причем $x_0(t_1) = 0$, $x_0(t_2) = \funcF$.

Задачу оптимизации, рассматриваемую в расширенном пространстве размерности $n+1$, можно теперь сформулировать таким образом: в $(n+1)$-мерном пространстве состояний с координатами $x_0, x_1, \ldots, x_n$ найти допустимое управление $\optU(t)$ ($t_1 \leqslant t \leqslant t_2$), которое переводит систему из начального положения $x_0 = 0$, $x = x_1$ в конечное таким образом, что $x_i(t_2) = s_i$ при $i = 1, 2, \ldots, m$, а $x_0(t_2)$ принимает наименьшее значение.

Введем $(n+1)$-мерный вектор $\psi(t)$ и образуем следующую функцию гамильтона от четырех переменных $x'$, $u$, $t$, $\psi'$

\beq{eq:2/3/4}
    H(x', u, t, \psi') \eqdef \psi'^T(t)f'(x', u, t) \mbox{.}
\eeq

Сформулируем теперь \emph{принцип максимума} Понтрягина.

\begin{teo}\lbl{teo:3}
	Если функция $\optU(t)$ определяет оптимальное управление, а $\optX'(t)$~--- оптимальная траектория, соответствующая $\optU(t)$, согласно уравнению~\ref{eq:2/3/3}, то имеется ненулевая непрерывная векторная функция $\psi'(t)$ такая, что
	\benum
		\item
			В любой момент времени $t$ (за исключением, возможно, точек, в которых $f'$ или $u$ являются разрывными) функция Гамильтона вдоль оптимальной траектории $\optX(t)$ достигает максимума. Иначе говоря,
			\beqn
				H(\optX', \optU, t, \psi') \geqslant H(\optX', u, t, \psi') \mbox{.}
			\eeqn
		
		\item
			В любой момент времени $t$ (за исключением, возможно, точек, в которых $f'$ или $u$ являются разрывными) вектор $\psi(t)$ удовлетворяет системе
			\beq{eq:2/3/5}
				\dot{\psi}'(t) = -\partdiff{H(x', \optU, t, \psi')}{x'} \biggm|_{x' = \optX(t)} \mbox{.}
			\eeq
		
		\item
			$\psi_i(t_2) = 0$ при $i= m+1, m+2, \ldots, n$.
		
		\item
			$\psi_0(t_2) \leqslant 0$
	\eenum
\end{teo}

\br

Выражения~\ref{eq:2/1/14} и~\vref{eq:2/1/16} очень похожи на выражения, связанные с максимизацией функции Гамильтона в формулировке задачи оптимального управления на основе принципа максимума (TODO). Так как для любой функции $F$ $\max(-F) = \min F$, то выражение~\ref{eq:2/1/14} можно записать в виде

\beq{eq:2/3/6}
    \partdiff{\optF}{t} = \underset{u(t) \in \Upsilon}{\max} \Bigl\{ -\funcL\bigl( x(u, t), u, t \bigr) - \biggl(\partdiff{\optF}{x}\biggr)^Tf \Bigr\} \mbox{.}
\eeq

Если вдоль оптимальной траектории вектор $-\partlinediff{\optF}{x}$ отождествить с сопряженным вектором $\psi$, то величина, заключенная в фигурных скобках в уравнении~\ref{eq:2/3/6}, является гамильтонианом $H$ для принципа максимума. В этом случае величина $\partlinediff{\optF}{t}$ представляет собой величину $H(\optX, \optU, t, \psi)$ и выполняется 2 условие теоремы~\vref{teo:3}. При этих условиях имеем

\beqarr
    \lbl{eq:2/3/7}
    	\partdiff{\optF}{x} = \psi(t) \mbox{;} \\
    \lbl{eq:2/3/8}
    	\partdiff{\optF}{x} = \funcH \eqdef H(\optX, \optU, t, \psi)
\eeqarr

Таким образом, сопряженный вектор $\psi'$, соответствующий оптимальной траектории, представляет собой вектор, направленный в сторону, противоположную направлению градиент критерия оптимальности $\optF$. Так как вектор градиента направлен в сторону быстрейшего изменения функции, то вектор $\psi$ располагается в каждый момент времени в направлении быстрейшего изменения критерия оптимальности.

Для расширенной системы имеем $H = \psi'^T\dot{x}'$. При этом принцип оптимальности дает $\partlinediff{\optF}{x'} = \psi'$. Итак, условие $\max\limits_{u \in \Omega(t)} H$ в принципе максимума означает, что в расширенном $(n+1)$-мерном пространстве управлений $\optU \in \Upsilon$ следует выбирать таким образом, чтобы в любой момент времени составляющая вектора $\dot{x}$ в направлении наискорейшего убывания величины $\optF$ была максимально большой.

Из выражения~\ref{eq:2/3/8} видно, что оптимальная функция Гамильтона $\funcH$ равна скорости изменения критерия оптимальности при отсчете времени от выбранной начальной точки. Поэтому ясно, почему в случае оптимального по быстродействию управления стационарным объектом функция $\funcH$ должна равняться нулю. Последнее видно из рассмотрения величины $\partlinediff{\optF(x, t)}{t}$. Напомним, что $t$ в $\optF(x, t)$ представляет собой начальное время. При фиксированной начальной точке оптимальное время достижения любой данной точки не зависит от начального времени для стационарного объекта управления, а следовательно $\funcH = 0$.

Иными словами, всякий раз, когда управляемый объект и функция стоимости $\funcL$ не зависит явно от времени, а конечное время $t_2$ не фиксировано, то $\partlinediff{\optF}{t}$ будет всегда равняться нулю, следовательно во всех этих случаях $\funcH = 0$.

Выражая уравнение Беллмана через гамильтониан $H$, можно записать

\beq{eq:2/3/9}
    \partdiff{\optF}{t} - H\biggl( \optX, \optU, t, \partdiff{\optF}{x} \biggr) = 0 \mbox{.}
\eeq

Это уравнение совпадает с уравнением Гамильтона--Якоби (TODO).

\br

Предположим, требуется найти оптимальное линейное управление с обратной связью для системы с точно определенным входным сигналом $z(t)$. Цель управления состоит в том, чтобы удерживать выходной сигнал близким к входному, не затрачивая при этом чрезмерно много энергии. Задачу можно сформулировать следующим образом.

Пусть выходной сигнал объекта $\dot{x} = \mA(t)x + \mB(t)u$ определяется как $y(t) = \mC(t)x(t)$, где векторы $x$ и $y$ соответственно $n$- и $m$-мерны. Пусть входной сигнал есть $m$-мерный вектор $z(t)$, а ошибка системы равна $e(t) = z(t) - y(t)$. Соответствующий показатель качества имеет вид

\beq{eq:2/3/10}
    \funcF = e^T(t_2) \mM e(t_2) + \frac{1}{2} \int\limits_{t_1}^{t_2} \bigl( e^T(\tau) \mQ(\tau) e(\tau) + u^T(\tau)\mR(\tau)u(\tau) \bigr)\,d\tau \mbox{;}
\eeq

его необходимо минимизировать, найдя оптимальное управление $\optU(x, t)$.

Определим гамильтониан

\beq{eq:2/3/10}
\begin{split}
    H & = -\frac{1}{2} \bigl( z(t) - \mC(t)x(t) \bigr)^T\mQ(t)\bigl( z(t) - \mC(t)x(t) \bigr) - \\
    &- \frac{1}{2} \bigl( u^T(t)\mR(t)u(t) \bigr) + \biggl( \bigl(\mA(t)x(t) \bigr)^T + \bigl(\mB(t)x(t) \bigr)^T \biggr)\psi(t)
\end{split}
\eeq

Оптимальную функцию управления $\optU(t)$ найдем из уравнения

\beqn
	\partdiff{H}{u}\biggm|_{u = \optU} = 0 \mbox{,}
\eeqn

откуда

\beq{eq:2/3/11}
    \optU(t) = -\mR^{-1}\mB^T(t)\psi(t) = -\mR^{-1}(t)\mB^T(t)\partdiff{\optF(x, t)}{x} \mbox{.}
\eeq

Это имеет смысл при условии существования обратной матрицы $\mR^{-1}(t)$ для всех $t$. Таким образом, удалось получить результат, аналогичный~\vref{eq:2/2/5}, но из других соображений.



% **********************************************************************************************
\section{Достаточное условие оптимальности}
% **********************************************************************************************



Докажем теорему о том, что уравнение Беллмана представляет собой условие оптимальности (TODO).

\begin{teo}\lbl{teo:4}
	Рассмотрим систему $\dot{x} = f(x, u, t)$, начинающую движение из начального состояния $x(t_1) = x_1$ на интервале времени $t \in [t_1, t_2]$. Требуется найти управление $u(t) = \optU(t)$, доставляющее минимум функционалу~\vref{eq:2/1/1}.
	
	Пусть область цели представляет собой открытое множество $\calf{S}$, а конечный момент времени $t_2$ не задан. Обозначим через $\calf{V}$ открытую область, в которой функция $\funcV(x, t)$ определяется следующим образом
	
	\benum
		\item
			$\partlinediff{\funcV}{t}$ непрерывная по $x$ и $t$, а $\partlinediff{\funcV}{x}$ или непрерывная по $x$ и $t$, или удовлетворяет условию
			\beq{eq:2/4/1}
    			\lim\limits_{t \to \tau-0} \Bigl[ \biggl(\partdiff{\funcV}{x}\biggr)^Tf \Bigr] = \lim\limits_{t \to \tau+0} \Bigl[ \biggl(\partdiff{\funcV}{x}\biggr)^Tf \Bigr]
			\eeq
			в любой момент времени $t = \tau$, когда $\partlinediff{\funcV}{x}$ или $f(x, u, t)$ разрывны.
			
		\item
			Для каждого $x \in \calf{V}$ в каждый момент времени $t$ функция Гамильтона $H(x, u, t, \partlinediff{\funcV}{x})$ имеет абсолютный минимум при $u = \optU$ из множества допустимых функций управления. Кроме того, управление $\optU(t)$ определяет единственную траекторию системы $\optX(t)$.
			
		\item
			На конечном множестве $\calf{S}$ $\funcV(x, t) = 0$.
	\eenum
	
	Тогда $\funcV(x, t) = \optF(x, t)$ при всех допустимых функциях управления, которые переводят изображающую точку из состояния $x_1$ в $\calf{S}$, не покидая области $\calf{V}$. Далее оптимальное управление с обратной связью $\optU(x, t)$ можно получить, решая соответствующее уравнение Беллмана.
\end{teo}


\strong{Доказательство.}

В соответствии с условиями 1 и 2 теоремы можно написать

\beq{eq:2/4/2}
\begin{split}
    0 =& \partdiff{\funcV}{t} + \Bigl[ \biggl(\partdiff{\funcV}{x}\biggr)^T f(x, \optU, t) + \funcL(x, \optU, t) \Bigr] < \\
       & < \partdiff{\funcV}{t} + \Bigl[ \biggl(\partdiff{\funcV}{x}\biggr)^T f(x, u, t) + \funcL(x, u, t) \Bigr]
\end{split}
\eeq

Проинтегрировав левую часть~\ref{eq:2/4/2} вдоль траектории $\optX(t)$ от $t_1$ до оптимального конечного момента $t_2^{*}$, получим

\beq{eq:2/4/3}
\begin{split}
    \int\limits_{t_1}^{t_2} \Bigl[ \partdiff{\funcV(\optX, t)}{t} +& \biggl(\partdiff{\funcV(\optX, t)}{x}\biggr)^T f(\optX, \optU, t) \Bigr]\,dt + \int\limits_{t_1}^{t_2} \funcL(\optX, \optU, t)\,dt = \\
    &= \int\limits_{t_1}^{t_2^{*}} \biggl[ \genericdiff{\funcV}{t} \biggr]_{\optU, t}\,dt + \int\limits_{t_1}^{t_2^{*}} \funcL(\optX, \optU, t)\,dt = 0 \mbox{.}
\end{split}
\eeq

Рассмотрим теперь интеграл

\beq{eq:2/4/4}
    W \eqdef \int\limits_{t_1}^{t_2^{*}} \Bigl[ \partdiff{\funcV}{t} + \biggl(\partdiff{\funcV}{x}\biggr)^T f(x, u, t) + \funcL(x, u, t) \Bigr]\,dt \mbox{.}
\eeq

Далее допустим, что сочетание $\optU$ и $\optX$ является единственным, которое доставляет абсолютный минимум величине $W$, равный в соответствии с условием~\ref{eq:2/4/3} нулю. Если это неверно, то найдутся другие $u(t) \neq \optU(t)$ и $x(t) \neq \optX(t)$, которые сделают выражение~\ref{eq:2/4/4} равным нулю. Однако из условия~\ref{eq:2/4/2} следует, что подынтегральное выражение~\ref{eq:2/4/4} всегда положительно. Таким образом, чтобы обратить $W$ в нуль, указанное подынтегральное выражение должно равняться нулю на всем интервале $[t_1, t_2]$. Из условий~\ref{eq:2/4/2} следует, что сделать это может лишь пара $\optU(t)$ и $\optX(t)$.

Таким образом, теорема~\vref{teo:4} доказана. \qed



% **********************************************************************************************
\section{Область применимости результатов}
% **********************************************************************************************



Сделаем ряд замечаний относительно оптимальных линейных законов управления. Неверным является суждение, что во всех конкретных случаях удается свести системы к стандартной форме и применить формулы, приведенные в этом разделе. С этим классом задач связан ряд серьезных трудностей.

Во-первых, критерий~\vref{eq:2/2/1} зачастую вызывает сомнения в целесообразности применения. Не существует обоснованного способа выбора нужных весовых матриц $\mM$, $\mQ(t)$ и $\mR(t)$. Очень часто соответствующие матрицы находят из эвристических соображений, методом проб и ошибок, причем наиболее удовлетворительную переходную характеристику определяют на глаз.

Во-вторых, закон оптимального управления с обратной связью использует все переменные состояния, а это обычно означает, что выход и $n-1$ его производных нужно измерять без ошибок, что в большинстве реальных систем попросту невозможно.

В-третьих, при отсутствии входного сигнала матрицу обратной связи $\mP(t)$ необходимо получить путем решения нелинейной системы уравнений~\vref{eq:2/2/7}. Обычно это выполняется численными методами. С инженерных позиций этот метод малоудовлетворителен не только потому, что для систем высокого порядка трудоемкость и объем вычисления существенно возрастают, но также потому, что полученных результат представляет собой совокупность $n(n+1)/2$ функций времени, которые являются коэффициентами усиления в канае обратной связи. Применительно к нестационарным системам правильность получаемых результатов вызывает сомнение, и любая попытка проверить полученный результат путем использования уравнения Беллмана может оказаться в высшей степени трудоемкой.

В-четвертых, условие~\vref{eq:2/2/13}, являющееся по существу условием оптимальности для стационарных систем, накладывает исключительно жесткие ограничения, практически делая невозможным расчет сколько-нибудь сложных систем.


% методы стохастического управления
% coding: utf-8
% --------------------------------------------------------------------------------------------------
% "Синтез оптимального стохастического управления", 2011 год
% --------------------------------------------------------------------------------------------------



\chapter{Методы стохастического управления}
% ==============================================================================================
\newcommand{\mSig}{     \sff{\Sigma}                     } % матрица ковариаций

\newcommand{\setTime}{  [0, T]                           } % временной промежуток, который достало вводить
\newcommand{\setX}{     \cale{X}                         } % множество X
\newcommand{\setT}{     \cale{T}                         } % множество T
\newcommand{\setF}{     \cale{F}                         } % множество F
\newcommand{\setM}{     \cale{M}_2\bigl( \setTime \bigr) } % множество M2
\newcommand{\setK}{     \cale{K}_2\bigl( \setTime \bigr) } % множество K2
\newcommand{\setU}{     \cale{U}                         } % множество U
\newcommand{\setC}{     \cale{C}                         } % множество U

\newcommand{\argEmpty}{ \{\emptyarg\}                    } % пустой аргумент

\renewcommand{\funcF}{  \calf{F}                         } % функция качества
\renewcommand{\funcH}{  \calf{H}                         } % функция качества H
\renewcommand{\funcV}{  \sff{V}                          } % функция Ляпунова

\newcommand{\idxB}[1]{  ^{(#1)}                          } % индекс в скобках
% ==============================================================================================



% **********************************************************************************************
\section{Некоторые сведения из теории случайных процессов}
% **********************************************************************************************



Перед рассмотрением методов стохастического управления, дадим некоторые сведения из теории вероятностей, в частности из теории случайных процессов.

\begin{df}\lbl{df:1}
    Семейство случайных величин, параметризованное параметров $t$, интерпретируемым как время, будем называть случайным процессом и обозначать как

    \beqn
        \xi_t \eqdef \xi(t, \omega) \in \setX \subset \Rv{1} \text{,}
    \eeqn

    где $t \in \setT \subset \Rv{1}$~--- временной параметр; $\setT$~--- множество определения, а $\setX$~--- множество значений случайного процесса $\xi_t$.
\end{df}

Область определения $\setT$ случайного процесса может быть конечным, полубесконечным или бесконечным интервалом числовой прямой. В этих случаях $\xi(t, \omega)$ называется \emph{случайным процессом с непрерывным временем}. Если $\setT$ имеет конечное или счетное множество элементов $t_k \in \setT \colon t_k < t_{k+1}$ ($k = 0, 1, 2, \ldots$), то $\xi_t$ называется \emph{случайным процессом с дискретным временем} или \emph{случайной последовательностью}.

Случайный процесс считается полностью заданным, если заданы его конечномерные распределения~--- набор функций распределения.

Случайный процесс называется \emph{гауссовским}, если все его совместные плотности распределения являются гауссовскими:

\beqn
    \prob_\xi(x_1, x_2, \ldots, x_k, t_1, t_2, \ldots, t_k) = \frac{1}{\sqrt{(2\pi)^n \det\mSig}}\exp\biggl( -\frac{(x-m)^T\mSig^{-1}(x-m)}{2} \biggr) \text{,}
\eeqn

где $x = (x_1, x_2, \ldots, x_k)^T$; $m = (m_1, m_2, \ldots, m_k)^T$; $m_i = \E x_i$; $\mSig = \mSig^T > 0$~--- матрица ковариаций.

Случайный процесс $\xi_t$, заданный на $\setT$, называется \emph{стационарным в узком смысле}, если для любых $n \geqslant 1$ и любых моментов $t_1, t_2, \ldots, t_n, t$, таких, что $t_i + t \in \setT$ ($i = 1, 2, \ldots, n$), совместные функции распределения совокупности случайных величин $\xi_{t_1+t}, \xi_{t_2+t}, \ldots, \xi_{t_n+t}$ не зависят от $t$.

\begin{df}\lbl{df:2}
    Случайный процесс $\xi_t$, определенный при всех $t \in \setTime$, называется непрерывным в момент времени $t^{*} \in \setTime$ в вероятностном смысле, если для любой последовательности $\{t_k\}_{k=1}^\infty$, принадлежащей $[0, T]$ и сходящейся к $t^{*}$ при $k \to \infty$, справедливо предельное соотношение

    \beqn
        \xi_{t_k} \to \xi_{t^{*}} \text{ при } t_k \to t^{*}  \text{,}
    \eeqn

    понимаемое в соответствующем вероятностном смысле.
\end{df}

\begin{df}\lbl{df:3}
    Случайный процесс $\xi_t$ называется непрерывным на промежутке $[t_1, t_2]$ в данном вероятностном смысле, если он непрерывен в этом смысле при $\forall t \in [t_1, t_2]$. Непрерывность в среднем степени 2 случайного процесса $\xi_t$ будем называть среднеквадратичной непрерывностью.
\end{df}

\begin{df}\lbl{df:4}
    Случайный процесс $\xi_t$, определенный для всех $t \in \setTime$, называется дифференцируемым в момент $t \in \setTime$ в одном из вероятностных смыслов, если существует левая часть следующего предельного соотношения:

    \beqn
        \lim\limits_{\delta \to 0} \frac{\xi_{t+\delta} - \xi_t}{\delta} = \genericdiff{\xi_t}{t} \text{,}
    \eeqn

    понимаемого в этом вероятностном смысле. Процесс $\xi_t$ будем называть дифференцируемым на промежутке $[t_1, t_2] \subset \setTime$ в данном вероятностном смысле, если он дифференцируем в этом смысле при всех $t \in [t_1, t_2]$.
\end{df}

\begin{df}\lbl{df:5}
    Случайный процесс $\xi_t$, определенный для всех $t \in \setTime$, называется процессом с независимыми приращениями, если для любых $t_0, t_1, \ldots, t_k \in \setTime$ ($0 \leqslant t_0 < t_1 < \ldots < t_k \leqslant T$), случайные величины $\xi_{t_0}, \xi_{t_1}-\xi_{t_0}, \ldots, \xi_{t_k}-\xi_{t_{k-1}}$ стохастически независимы между собой.
\end{df}

Особую роль в дальнейшем изложении будут играть так называемые винеровские случайные процессы, относящиеся к классу гауссовских процессов с независимыми приращениями\footnote{ Подробное изложение теории случайных процессов с независимыми приращениями можно найти в монографии А.\,В.\,Скорохода\cite{SKOROHOD}.}.

\begin{df}\lbl{df:6}
    Случайный процесс $\xi_t \in \Rv{1}$, определенный для всех $t \in \setTime$, называется винеровским, если он является процессом с независимыми приращениями и удовлетворяет следующим условиям

    \benum
        \item
            $\E \xi_t = 0$,

        \item
            $\E (\xi_{t_2} - \xi_{t_1})^2 - \sigma_\xi^2 |t_2 - t_1| \text{, где } \sigma_\xi^2 > 0$,

        \item
            $\xi_{t_1}$ и $\xi_{t_2} - \xi_{t_1}$ ($t_2 > t_1$) имеют нормальное распределение при всех $t_1, t_2 \in \setTime$.
    \eenum
\end{df}

Если $\sigma_\xi^2$ и $\xi_0 = 0$, то винеровский процесс называется \emph{стандартным}. Поскольку приращения $\xi_{t_2} - \xi_{t_1}$ ($t_2 > t_1$) винеровского процесса имеют нормальное распределение, то их функция распределения определяется соотношением

\beqn
    \prob(\xi_{t_2} - \xi_{t_1} < x) = \frac{1}{\sigma_\xi \sqrt{2\pi (t_2 - t_1)}} \int\limits_\infty^x e^{-\frac{u^2}{2\sigma_xi^2(t_2-t_1)}}\,du \text{.}
\eeqn

Известно, что винеровский случайный процесс является непрерывным с вероятностью 1 случайным процессом.

\br

Определим теперь понятия интегралов и процессов Ито.

Пусть задано вероятностное пространство $(\Omega, \setF, \prob)$ и винеровский случайный процесс $f_t \in \Rv{1}$. Рассмотрим совокупность $\sigma$-алгебр $\{ \setF_t, t \in \setTime \}$, определенную на вероятностном пространстве $(\Omega, \setF, \prob)$ и связанную с винеровским процессом $f_t$ так, что

\benum
    \item
        $\setF_s \subset \setF_t \subset \setF$ при $s<t$;

    \item
        $f_t$ измерим относительно $\setF_t$ при каждом $t \in \setTime$;

    \item
        Процесс $f_{t+\Delta} - f_\Delta$ при всех $\Delta \geqslant 0$ и $t > 0$ не зависит от событий $\sigma$-алгебры $\setF_\Delta$.
\eenum

В дальнейшем рассматриваются только такие винеровские процессы $f_t$, которые удовлетворяют условиям 2 и 3.

Введем в рассмотрение класс $\setM$ функций $\xi \colon \setTime \times \Omega \to \Rv{1}$, которые удовлетворяют следующим условиям:

\benum
    \item
        Функции $\xi(t, \omega)$ измеримы по совокупности переменных $(t, \omega)$;

    \item
        Функция $\xi(t, \omega) \in \setF_t$ измерима при всех $t \in \setTime$, и ее значения $\xi(\tau, \omega)$ не зависят от приращений $f_{t+\Delta} - f_\Delta$ винеровского процесса при $\Delta \geqslant \tau$, $t > 0$;

    \item
        $\int_0^T \E \Bigl\{ \bigl( \xi(t, \omega) \bigr)^2 \Bigr\}\,dt < \infty$;

    \item
        $\E \Bigl\{ \bigl( \xi(t, \omega) \bigr)^2 \Bigr\} < \infty$ для всех $t \in \setTime$.
\eenum

Введем на классе $\setM$ норму вида

\beqn
    \| \xi \|_{2,T} = \sqrt{ \int_0^T \E \Bigl\{ \bigl( \xi(t, \omega) \bigr)^2 \Bigr\}\,dt } \text{.}
\eeqn

Считаем две функции $\xi$ и $\eta$ из класса $\setM$ эквивалентными, если $\| \xi - \eta \|_{2,T} = 0$.

Для произвольного разбиения $\{ \tau_j \}_{j=0}^N$ промежутка $\setTime$, такого что $0 = \tau_0 < \tau_1 < \ldots < \tau_N = T$, и произвольной последовательности среднеквадратично интегрируемых, $\setF_{\tau_j}$-измеримых и независящим от приращений $f_{s+t} - f_t$ при $t \geqslant \tau_j$, $s > 0$ случайных величин $\xi_j(\omega)$ ($j = 0, 1, \ldots, N-1$), определим \emph{ступенчатую функцию} $\xi \in \setM$ следующим образом: $\xi(t, \omega) = \xi_j(\omega)$ с вероятностью 1 при $\tau_j \leqslant t < \tau_{j+1}$ ($j = 0, 1, \ldots, N-1$).

При этом интеграл в требовании 3 понимается как

\beqn
    \int\limits_0^T \E \Bigl\{ \bigl( \xi(t, \omega) \bigr)^2 \Bigr\}\,dt = \sum\limits_{j=0}^{N-1} \E \bigl\{ (\xi_j)^2 \bigr\}(\tau_{j+1} - \tau{j}) \text{.}
\eeqn

Обозначим через $\setK$ множество всех ступенчатых функций в $\setM$. Для произвольной ступенчатой функции $\xi$ из класса $\setK$ определим \emph{стохастический интеграл Ито} следующим образом:

\beqn
    \int\limits_0^T \xi_\tau\,df_\tau \eqdef \sum\limits_{j=0}^{N-1} \xi_j(\omega)\bigl( f(\tau_{j+1}, \omega) - f(\tau_j, \omega) \bigr) \text{.}
\eeqn

\begin{df}\lbl{df:7}
    Стохастическим интегралом Ито от функции $\xi \in \setM$ назовем среднеквадратичный предел последовательности ступенчатых функций:

    \beqn
        \int\limits_0^T \xi_\tau\,df_\tau = \lim\limits_{N \to \infty} \sum\limits_{j=0}^{N-1} \xi\idxB{N}\bigl( f(\tau\idxB{N}_{j+1}, \omega) - f(\tau\idxB{N}_j, \omega) \bigr) \text{,}
    \eeqn

    где $\xi\idxB{N}(t, \omega)$~--- произвольная ступенчатая функция из $\setK$, сходящаяся по норме $\| \emptyarg \|_{2,T}$ к функции $\xi(t, \omega)$.
\end{df}

Известно, что для $\xi \in \setM$ такой интеграл существует\cite{KLOEDEN}.

\br

Винеровские случайные процессы порождают широкий класс случайных процессов, называемых процессами Ито.

Пусть задано фиксированное вероятностное пространство $(\Omega, \setF, \prob)$ и $\setF_t$-измеримый при всех $t \in \setTime$ векторный винеровский процесс $\m{f}_t \in \Rv{m}$ с независимыми компонентами $\m{f}_t\idxB{i}$ ($i = 1, 2, \ldots, m$).

Рассмотрим случайные процессы $a_t \in \Rv{1}$ и $\m{b}_t\idxB{i} \in \Rv{1}$ ($i = 1, 2, \ldots, m$) из класса $\setM$. Определим случайный процесс $\eta_t$ при всех $t \in \setTime$ в виде:

\beq{eq:3/1/1}
    \eta_t = \eta_0 + \int\limits_0^t a_\tau\,d\tau + \sum\limits_{i=1}^m \int\limits_0^t \m{b}_\tau\idxB{i}\,d\m{f}_\tau\idxB{i} \text{,}
\eeq

где $\eta_0$~--- случайная величина, стохастически независимая от приращений $\m{f}_\tau-\m{f}_0$ винеровского процесса $\m{f}_t$ при всех $\tau \in \setTime$.

\begin{df}\lbl{df:8}
    Процесс $\eta_t$, определенный на промежутке $\setTime$ в виде~\ref{eq:3/1/1}, называется процессом Ито, если правая часть~\ref{eq:3/1/1} существует при всех $t \in \setTime$. При этом процесс $a_t$ называется коэффициентом сноса, а процессы $\m{b}_t\idxB{i}$ ($i = 1, 2, \ldots, m$)~--- коэффициентами диффузии процесса $\eta_t$.
\end{df}

Из свойства аддитивности стохастических интегралов вытекает следующее представление для процесса Ито:

\beq{eq:3/1/2}
    \eta_s = \eta_t + \int\limits_t^s a_\tau\,d\tau + \sum\limits_{i=1}^m \int\limits_0^t \m{b}_\tau\idxB{i}\,d\m{f}_\tau\idxB{i} \text{ с вероятностью 1,}
\eeq

справедливое для всех $s, t \in \setTime$ и $s \geqslant t$.

Рассмотрим многомерный случай. Пусть существуют случайные процессы $\m{a}_t \in \Rv{n}$ и $\m{B}_t \in \Rs{n}{m}$, каждая компонента которых должна принадлежать классу $\setM$. Определим векторный случайный процесс $\m{\eta}_t \in \Rv{n}$ при всех $t \in \setTime$ в виде:

\beq{eq:3/1/3}
    \m{\eta}_t = \m{\eta}_0 + \int\limits_0^t \m{a}_\tau\,d\tau + \int\limits_0^t \m{B}_\tau\,d\m{f}_\tau \text{,}
\eeq

где случайная векторная величина $\m{\eta}_0 \in \Rv{n}$ является стохастически независимой от приращений $\m{f}_\tau - \m{f}_0$ винеровского процесса $\m{f}_t$ при всех $\tau \in \setTime$.

\begin{df}\lbl{df:9}
    Процесс $\m{\eta}_t \in \Rv{n}$ вида~\ref{eq:3/1/2}, определенный на промежутке $\setTime$ называется векторным процессом Ито, если правая часть~\ref{eq:3/1/2} существует при всех $t \in \setTime$. При этом процессы $\m{a}_t\idxB{i}$ называются коэффициентами снова, а процессы $\m{B}_t\idxB{i}$ называются коэффициентами диффузии процесса $\m{\eta}_t$ ($i = 1, 2, \ldots, n$, $j = 1, 2, \ldots, m$).
\end{df}

Соотношение~\ref{eq:3/1/2} обобщается и на случай векторного процесса Ито:

\beq{eq:3/1/4}
    \eta_s = \eta_t + \int\limits_t^s \m{a}_\tau\,d\tau + \int\limits_0^t \m{B}_\tau\,d\m{f}_\tau \text{ с вероятностью 1,}
\eeq

справедливое для всех $s, t \in \setTime$ и $s \geqslant t$.

\br

Процессы Ито тесно связаны со стохастическими дифференциальными уравнениями Ито. Действительно, пусть задано дифференциальное уравнение

\beqarr
    \lbl{eq:3/1/5}
        dx_t = a(x_t, t)dt + \sigma(x_t, t)df_t \text{;} \\
    \lbl{eq:3/1/6}
        x_0 = x(0, \omega) \text{,}
\eeqarr

где $x_t \in \Rv{1}$~--- случайный процесс, являющийся решением уравнения~\ref{eq:3/1/5}; $a(x, t)$ и $\sigma(x, t) \in Rv{1}$~--- измеримые при всех $(x, t) \in \Rv{1} \times \setTime$ функции. Тогда это уравнение может быть переписано в интегральном виде

\beqarr
    \lbl{eq:3/1/7}
        x_t = x_0 + \int\limits_0^t a(x_\tau, \tau)\,d\tau + \int\limits_0^t \sigma(x_\tau, \tau)\,df_\tau \text{;} \\
    \lbl{eq:3/1/8}
        x_0 = x(0, \omega) \text{.}
\eeqarr

Второй интеграл в правой части~\ref{eq:3/1/7} может пониматься в смысле Ито, Стратоновича или как $\theta$-интеграл\cite{KUZNETCOV}. Пусть интеграл $\int_0^t \sigma(x_\tau, \tau)\,df_\tau$ является стохастическим интегралом Ито, тогда уравнение~\ref{eq:3/1/5} и соответствующее ему уравнение~\ref{eq:3/1/7} называются стохастическими дифференциальными уравнениями Ито.



% **********************************************************************************************
\section{Оптимальное стохастическое управление}
% **********************************************************************************************



Пусть уравнение состояния динамической системы представляет собой систему стохастических дифференциальных уравнений Ито

\beq{eq:3/2/1}
    d\m{x}_t = \m{a}(\m{x}_t, \m{u}, t)dt + \m{\Sigma}(\m{x}_t, \m{u}, t)d\m{f}_t \text{,}
\eeq

где $\m{x}_t \in \Rv{n}$~--- случайный процесс, являющийся решением уравнения~\ref{eq:3/2/1}; $\m{a} \in \Rv{n}$, $\m{\Sigma} \in \Rv{n}$ представляют собой векторные функции сноса и диффузии соответственно; $\m{f}_t$~--- стандартный винеровский процесс, моделирующий белый шум; $\m{u} \in \Rv{k}$~--- управляющее воздействие, которое выбирается из условия минимизации функционала вида

\beq{eq:3/2/2}
    \funcF(\m{x}, \m{u}, s) = \E \biggl\{ Q(\m{x}_\tau, \tau) + \int\limits_s^\tau F(\m{x}_t, \m{u}, t)\,dt~\biggm|~ \m{x}_s = \m{x} \biggr\} \text{,}
\eeq

где $Q$, $F$~--- известные функции, а $\tau$, например, постоянный момент времени.

Известны различные типы управляющих воздействий, которые могут использовать или не использовать информацию о векторе состояния системы в предшествующие моменты времени (некоторые из них были рассмотрены в предыдущем разделе). Рассмотрим управление $\m{u}(\m{x}_t, t)$ марковского типа. Тогда для него минимальная по $u$ величина функционала $\funcF$, которую обозначим через $\funcH(\m{x}, s)$, удовлетворяет уравнению Гамильтона--Якоби--Беллмана вида

\beqarr
    \lbl{eq:3/2/3}
        \min\limits_{\m{u} \in \Rv{k}} \bigl\{ F(\m{x}, \m{u}, s) + L_u \funcH(\m{x}, s) \bigr\} = 0 \text{;} \\
    \lbl{eq:3/2/4}
        \funcH(\m{x}, T) = Q(\m{x}, T) \text{,}
\eeqarr

где предполагается, что $\tau = T$, а оператор $L_u\argEmpty$ имеет вид

\beq{eq:3/2/5}
\begin{split}
    L_u\argEmpty = \partdiff{\argEmpty}{s} &+ \sum\limits_{i=1}^n\m{a}\idxB{i}(\m{x}, \m{u}, s)\partdiff{\argEmpty}{\m{x}\idxB{i}}~+ \\
    &+ \frac{1}{2} \overset{n}{\sum\limits_{i=1} \sum\limits_{j=1}} \m{D}\idxB{ij}(\m{x}, \m{u}, s)\partdiffsecond{\argEmpty}{\m{x}\idxB{i}}{\m{x}\idxB{j}} \text{,}
\end{split}
\eeq

где $\m{D} = \m{\Sigma}\m{\Sigma}^T$.

Отметим, что приведенные выше условия являются необходимыми условиями, которым удовлетворяет минимум функционала, если он существует.

\br

Рассмотрим линейно-квадратичную проблему стохастического управления, то есть проблему нахождения управления $\m{u}(\m{x}_t, t)$ марковского типа, которое доставляет минимум следующему функционалу:

\beq{eq:3/2/6}
    \funcF(\m{x}, \m{u}, s) = \E \biggl\{ \m{x}_T\m{R}\m{x}_T^T + \int\limits_s^T \bigr( \m{x}_t^T\m{C}(t)\m{x}_t + \m{u}^T\m{P}(t)\m{u} \bigl)\,dt~\biggm|~ \m{x}_s = \m{x} \biggr\} \text{,}
\eeq

где $\m{C}(t), \m{R} \in \Rs{n}{n}$, $\m{P}(t) \in \Rs{k}{k}$, а $\m{x}_t \in \Rv{n}$~--- решение следующей системы стохастических дифференциальных уравнений.

\beq{eq:3/2/7}
    d\m{x}_t = \bigl( \m{A}(t)\m{x}_t + \m{B}(t)\m{u}(\m{x}_t, t) \bigr)dt + \m{\Sigma}(t)d\m{w}_t \text{,}
\eeq

где $\m{A}(t) \in \Rs{n}{n}$, $\m{B}(t) \in \Rs{n}{k}$, $\m{\Sigma} \in \Rs{n}{m}$, а $\m{w}_t \in \Rv{m}$~--- стандартный винеровский процесс с независимыми компонентами $\m{w}\idxB{i}_t$ ($i = 1, 2, \ldots, m$). Кроме этого, предполагается, что все зависящие от $t$ матрицы непрерывны. $\m{C}(t)$ и $\m{R}$~--- симметричные неотрицательно определенные матрицы, а матрица $\m{P}(t)$~--- симметричная положительно определенная.

Запишем уравнение Гамильтона-Якоби-Беллмана для данной задачи:

\beq{eq:3/2/8}
\begin{split}
    \partdiff{\funcH}{s} + \m{x}^T\m{C}(s)\m{x} &+ \sum\limits_{j=1}^n \bigl( \m{A}(s)\m{x} \bigr)\idxB{j}\partdiff{\funcH}{\m{x}\idxB{j}} + \frac{1}{2} \partdiffsecond{\funcH}{\m{x}\idxB{i}}{\m{x}\idxB{j}}~+ \\
    &+ \min\limits_{\m{u} \in \Rv{k}} \Bigl\{ \m{u}^T\m{P}(t)\m{u} + \sum\limits_{j=1}^n \bigl( \m{B}(s)\m{u} \bigr)\idxB{j}\partdiff{\funcH}{\m{x}\idxB{j}} \Bigr\} = 0 \text{,}
\end{split}
\eeq

где $\m{D} = \m{\Sigma}(t)\m{\Sigma}^T(t)$, а граничное условие имеет вид:

\beq{eq:3/2/9}
    \funcH(T, \m{x}) = \m{x}^T\m{R}\m{x} \text{.}
\eeq

В\cite{KLOEDEN}, в частности, показано, что оптимальное управление и минимальное значение функционала $\funcF(\m{x}, \m{u}, s)$ имеют в этой задаче следующий вид:

\beqarr
    \lbl{eq:3/2/10}
        \m{u}(\m{x}, s) = \m{P}^{-1}(s)\m{B}(s)\m{x} \text{;} \\
    \lbl{eq:3/2/11}
        \funcH(\m{x}, s) = \m{x}^T\m{S}\m{x} + \int\limits_s^T \tr \bigl\{ \m{\Sigma}(t)\m{\Sigma}^T(t)\m{S}(t) \bigr\}\,dt \text{,}
\eeqarr

где $\tr$~--- след\footnote{ Сумма диагональных элементов. } матрицы; $\m{S}(s) \in \Rs{n}{n}$~--- симметричная положительно определенная и непрерывно дифференцируемая матрица, которая удовлетворяет следующему матричному уравнению Риккати:

\beq{eq:3/2/12}
\begin{split}
    \genericdiff{\m{S}(s)}{s} &= -\m{A}(S)^T\m{S}(s) - \m{S}(s)\m{A}(s)~+ \\
    &+ \m{S}(s)\m{B}(s)\m{P}^{-1}(s)\m{B}(s)\m{S}(s) - \m{C}(s)
\end{split}
\eeq

с начальным условием

\beq{eq:3/2/13}
    \m{S}(T) = \m{R} \text{.}
\eeq

Таким образом, если удается решить (хотя бы численно) уравнение Риккати~\ref{eq:3/2/12}, то по формулам~\ref{eq:3/2/10} и~\ref{eq:3/2/11} можно найти оптимальное управление и минимальное значение функционала.

\br

Предположим, что измерение вектора $\m{x}_t$ невозможно, но допустимо измерять наблюдаемый процесс $\m{y}_t \in \Rv{k}$ ($k = 1, 2, \ldots, n$), который удовлетворяет следующей системе дифференциальных уравнений Ито:

\beq{eq:3/2/14}
    d\m{y}_t = \m{H}\m{x}_tdt + \Xi d\m{f}_t
\eeq

с начальным условием $\m{y}_0 = \m{y}(0) = 0$.

В системе~\ref{eq:3/2/14} $\m{H} \in \Rs{k}{n}$, $\Xi \in \Rs{k}{l}$~--- числовые матрицы; $\m{f}_t \in \Rv{l}$~--- стандартный винеровский процесс с независимыми компонентами $\m{f}\idxB{i}_t$ ($i = 1, 2, \ldots, l$), который не зависит от процесса $\m{w}_t$. В этом случае, с помощью фильтра Калмана-Бьюси можно посмтроить оценку $\hat{\m{x}}_t$ и использовать ее в оптимальном управлении $\m{u}(\hat{\m{x}}_t, t)$ для линейно-квадратичной задачи.

Такие исследования остаются за пределами рассмотрения этой работы, подробно они излагаются в соответствующей литературе\cite{KLOEDEN}.

\br

Отметим, что полученные результаты практически полностью, вплоть до переобозначений, совпадают с результатами~\ref{eq:2/2/7},~\ref{eq:2/2/8} и~\vref{eq:2/2/9}, что позволяет говорить о \emph{принципе полной эквивалентности в стохастическом управлении}. Иными словами, построение регулятора, синтезирующего оптимальное управление с обратной связью в случае процессов Ито ничем не отличается от классической модели.



% **********************************************************************************************
\section{Стохастическая устойчивость}
% **********************************************************************************************



Рассмотрим стохастическое дифференциальное уравнение Ито вида~\vref{eq:3/2/1}. Будем считать, что $\m{x}_t \equiv \m{0}$\footnote{ Здесь и далее полагаем <<$\m{0}$>> вектором соответствующей размерности, состоящим из нулей. } является невозмущенным движением динамической системы, а возмущенное движение описывается~\ref{eq:3/2/1}. В соответствии с этим положим

\beq{eq:3/3/1}
    \eqsystem{
        \m{a}(\m{0}, t) \equiv 0 \text{,} \\
        \m{\Sigma}(\m{0}, t) \equiv 0
    }
\eeq

при $t \geqslant 0$.

\br

Приведем некоторые определения основных типов устойчивости для стохастических систем.

\begin{df}\lbl{df:10}
    Невозмущенное движение системы $\m{x}_t = \m{0}$ будем называть устойчивым по вероятности или стохастически устойчивым при $t \geqslant t_0$, если для любых $\eps > 0$ и $t_0 \geqslant 0$

    \beqn
        \lim\limits_{\m{x}_0 \to \m{0}} \prob \Bigl( \sup\limits_{t \geqslant t_0} \bigl| \m{x}^{\m{x}_0, t_0}_t \bigr| \geqslant \eps \Bigr) = 0 \text{,}
    \eeqn

    где $\m{x}^{\m{x}_0, t_0}_t$~--- такое решение уравнения~\ref{eq:3/2/1}, что $\m{x}_{t_0} = \m{x}_0$.
\end{df}

\begin{df}\lbl{df:11}
    Невозмущенное движение системы $\m{x}_t \equiv \m{0}$ называется асимптотически устойчивым по вероятности или асимптотически стохастически устойчивым, если оно стохастически устойчиво и, кроме того,

    \beqn
        \lim\limits_{\m{x}_0 \to \m{0}} \prob \Bigl( \lim\limits_{t \to \infty} \bigl| \m{x}^{\m{x}_0, t_0}_t \bigr| = 0 \Bigr) = 1 \text{.}
    \eeqn
\end{df}

Приведенные определения являются обобщением классических определений устойчивости в малом по Ляпунову на случай стохастических систем.

\begin{df}\lbl{df:12}
    Невозмущенное движение системы $\m{x}_t \equiv \m{0}$ называется асимптотически стохастически устойчивым в целом, если оно стохастически устойчио и, кроме того, для всех $|\m{x}_0| < \infty$

    \beqn
        \prob \Bigl( \lim\limits_{t \to \infty} \bigl| \m{x}^{\m{x}_0, t_0}_t \bigr| = 0 \Bigr) = 1 \text{.}
    \eeqn
\end{df}

Приведем также некоторые определения устойчивости, которые связаны с моментальными характеристиками процесса $\m{x}_t$.

\begin{df}\lbl{df:13}
    Невозмущенное движение системы $\m{x}_t \equiv \m{0}$ называется устойчивым в среднем степени $p$ ($p > 0$) или $p$-устойчивым, если для любых $\eps > 0$ и $t_0 \geqslant 0$ найдется $\delta = \delta(t_0, \eps) > 0$ такое, что

    \beqn
        \E \Bigl\{ \bigl| \m{x}^{\m{x}_0, t_0}_t \bigr|^p \Bigr\} < \eps \text{ для всех } t \geqslant t_0 \text{ и } |\m{x}_0| < \delta \text{.}
    \eeqn
\end{df}

\begin{df}\lbl{df:14}
    Невозмущенное движение системы $\m{x}_t \equiv \m{0}$ называется асимптотически устойчивым в среднем степени $p$ ($p > 0$) или асимптотически $p$-устойчивым, если оно $p$-устойчиво и существует $\delta_0 = \delta_0(t_0) > 0$, такое, что

    \beqn
        \lim\limits_{t \to \infty} \E \Bigl\{ \bigl| \m{x}^{\m{x}_0, t_0}_t \bigr|^p \Bigr\} = 0 \text{ для всех } |\m{x}_0| < \delta_0 \text{.}
    \eeqn
\end{df}

\begin{df}\lbl{df:15}
    Невозмущенное движение системы $\m{x}_t \equiv \m{0}$ называется экспоненциально $p$-устойчивым, если существуют постоянные $\gamma > 0$ и $\alpha > 0$, такие, что

    \beqn
        \E \Bigl\{ \bigl| \m{x}^{\m{x}_0, t_0}_t \bigr|^p \Bigr\} < \gamma|\m{x}_0|^pe^{-\alpha(t-t_0)} \text{ для всех } t \geqslant t_0 \text{.}
    \eeqn
\end{df}

Следует заметить, что существуют и другие определения стохастической устойчивости\cite{KUSHNER}.

\br

Как известно, одним из методов исследования устойчивости детерминированных систем является метод функций Ляпунова. Достоинство этого метода заключается в том, что он позволяет (в том случае, если соответствующая функция Ляпунова найдена) делать заключения о характере поведения решения дифференциального уравнения, не требуя при этом знания точного решения этого дифференциального уравнения.

Рассмотрим суть метода стохастических функций Ляпунова. Рассмотрим функцию $\funcV(\m{x}, t) \colon \Rv{n} \times \{t > 0\} \to \Rv{1}$, которая при всех $(\m{x}, t) \in \setU \times \{t > 0\}$ ($\setU \subset \Rv{n}$, $\m{0} \in \setU$) является непрерывной и положительно определенной. Пусть также $\funcV(\m{x}, t)$ дважды непрерывно дифференцируема по $\m{x}$ и один раз по $t$, за исключением может быть множества $\{ \m{x} = \m{0} \}$. Кроме этого, предположим, что $\funcV(\m{0}, t) = 0$.

Далее через $\setC_2^0(\setU)$ будем обозначать множество функций дважды непрерывно дифференцируемых по $\m{x} \in \setU$ и один раз по $t$, за исключением множества $\{ \m{x} = \m{0} \}$. В силу формулы Ито имеем

\beq{eq:3/3/2}
\begin{split}
    \funcV(\m{x}_t, t) &- \funcV(\m{x}_{t_0}, t_0) = \int\limits_{t_0}^t L\bigl\{ \funcV(\m{x}_\tau, \tau) \bigr\}\,d\tau~+ \\
    &+\sum\limits_{i=1}^m \int\limits_{t_0}^t \m{G}\idxB{i}_0\bigl\{ \funcV(\m{x}_\tau, \tau) \bigr\}\,d\m{f}\idxB{i}_\tau \text{ с вероятностью 1,}
\end{split}
\eeq

где

\beq{eq:3/3/3}
\begin{split}
    L\argEmpty = \partdiff{\argEmpty}{t} &+ \sum\limits_{i=1}^n \m{a}\idxB{i}(\m{x}, t) \partdiff{\argEmpty}{\m{x}\idxB{i}}~+ \\
    &+ \frac{1}{2}\sum_{j=1}^m \overset{n}{\sum\limits_{l=1}\sum\limits_{l=1}} \m{\Sigma}\idxB{lj}(\m{x}, t) \m{\Sigma}\idxB{ij}(\m{x}, t) \partdiffsecond{\argEmpty}{\m{x}\idxB{l}}{\m{x}\idxB{i}}
\end{split}
\eeq

и при $i = 1, 2, \ldots, m$

\beq{eq:3/3/4}
    G\idxB{i}\argEmpty = \sum_{j=1}^n \m{\Sigma}\idxB{ji}(\m{x}, t) \partdiff{\argEmpty}{\m{x}\idxB{j}} \text{.}
\eeq

Как можно видеть, в стохастическом случае роль первой производной $\funcV(\m{x}, t)$ по $\m{x}$ в силу уравнения движения играет $L\bigl\{ \funcV(\m{x}_s, s) \bigr\}$. Будем считать, что при $\m{x} \neq \m{0}$ и $t \geqslant 0$ выполняется неравенство

\beq{eq:3/3/5}
    L\bigl\{ \funcV(\m{x}_s, s) \bigr\} \leqslant 0 \text{.}
\eeq

Функцию $\funcV(\m{x}, t)$, обладающую перечисленными выше свойствами, будем называть \emph{стохастической функцией Ляпунова}.

На стохастическую функцию Ляпунова могут также накладываться и некоторые дополнительные ограничения. Далее, используя~\ref{eq:3/3/2} и~\ref{eq:3/3/5} и свойства интеграла Ито, получаем

\beq{eq:3/3/6}
    \E \bigl\{ \funcV(\m{x}_t, t) \bigm| \funcF_{t_0} \bigr\} \leqslant \funcV(\m{x}_{t_0}, t_0) \text{ с вероятностью 1.}
\eeq

Таким образом, $\funcV(\m{x}_t, t)$~--- супермартингал. Если $\m{y}_s$ ($s \geqslant 0$)~--- мартингал и $\E \bigl\{ |\m{y}_s|^p \bigr\} < \infty$, то справедливо хорошо известное неравенство:

\beq{eq:3/3/7}
    \forall \alpha:~\prob\Biggl(\bigl\{ \omega \in \Omega \colon \sup\limits_{0 \leqslant s \leqslant t} |\m{y}(s, \omega)| \geqslant \alpha \bigr\}\Biggr) \leqslant \frac{1}{\alpha^p} \E \bigl\{ |\m{y}_t|^p \bigr\}
\eeq

и, следовательно, для всех $T > t_0$ и $\delta > 0$

\beq{eq:3/3/8}
    \prob\Bigl( \sup\limits_{t_0 \leqslant t \leqslant T} \funcV(\m{x}_t, t) \geqslant \delta \Bigr) \leqslant \frac{1}{\delta} \funcV(\m{x}_{t_0}, t_0) \text{.}
\eeq

Из последнего неравенства, непрерывности $\funcV(\m{x}, t)$ и того, что $\funcV(\m{0}, t) = 0$, получаем:

\beq{eq:3/3/9}
    \lim\limits_{\m{x}_0 \to \m{0}} \prob\Bigl( \sup\limits_{t_0 \leqslant t \leqslant T} \funcV(\m{x}_t, t) \geqslant \delta \Bigr) = 0 \text{.}
\eeq

Далее, если предположить, что при некоторых постоянных $p$, $k_1$ и $k_2$

\beq{eq:3/3/10}
    k_1 |\m{x}|^p \leqslant \funcV(\m{x}, t) \leqslant k_2 |\m{x}|^p \text{,}
\eeq

то согласно~\ref{eq:3/3/9} получим, что невозмущенное движение системы $\m{x}_t~\equiv~\m{0}$ стохастически устойчиво.

\br

Приведем еще ряд фактов\cite{HASMINSKI}, касающихся стохастической устойчивости

\bteo{teo:5}
    Пусть в области $\{ t > 0 \} \times \setU$, включающей множество $\{ \m{x} \equiv \m{0} \}$, существует непрерывная, положительно определенная функция $\funcV(\m{x}, t) \in \setC_2^0(\setU)$ и пусть при $t > 0$

        \beqn
            \limsup\limits_{\m{x} \to \m{0}} \funcV(\m{x}, t) = 0 \text{,}
        \eeqn

        а $L\bigl\{ \funcV(\m{x}, t) \bigr\} < 0$ в этой области. Тогда невозмущенное движение $\m{x}_t \equiv \m{0}$ асимптотически стохастически устойчиво.
\eteo

\bteo{teo:6}
    Пусть для всех $\m{x} \in \Rv{n}$ существует положительно определенная функция $\funcV(\m{x}, t) \in \setC_2^0(\Rv{n})$, для которой при $t > 0$

    \beqn
        \eqsystem{
            \limsup\limits_{\m{x} \to \m{0}} \funcV(\m{x}, t) = 0 \text{,} \\
            \liminf\limits_{\m{x} \to \m{\infty}} \funcV(\m{x}, t) = \infty \text{,} \\
            L\bigl\{ \funcV(\m{x}, t) \bigr\} < 0 \text{.}
        }
    \eeqn

    Тогда невозмущенное движение $\m{x}_t \equiv \m{0}$ асимптотически стохастически устойчиво в целом.
\eteo

\bteo{teo:7}
    Для экспоненциальной $p$-устойчивости невозмущенного движения $\m{x}_t \equiv \m{0}$ при $t > 0$ достаточно, чтобы при всех $\m{x} \in \Rv{n}$ существовала функция $\funcV(\m{x}, t) \in \setC_2^0(\Rv{n})$, удовлетворяющая при некоторых положительных постоянных $k_1$, $k_2$ и  $k_3$ неравенствам

    \beqn
        \eqsystem{
            k_1 |\m{x}|^p \leqslant \funcV(\m{x}, t) \leqslant k_2 |\m{x}|^p \text{,} \\
            L\bigl\{ \funcV(\m{x}, t) \bigr\} < -k_3 |\m{x}|^p \text{.}
        }
    \eeqn
\eteo

Другой метод детерминистской теории устойчивости заключается в линеаризации дифференциального уравнения в окрестности невозмущенного движения и анализе устойчивости нулевого решения линеаризованного дифференциального уравнения. Известно\cite{ARNOLD}, что в ряде случаев устойчивость нулевого решения линеаризованной системы влечет устойчивость невозмущенного решения соответствующей нелинейной системы. Особую роль при анализе устойчивости с помощью этого подхода играют так называемые ляпуновские экспоненты или ляпуновские показатели. Оказывается, что метод линеаризации может быть обобщен и на случай стохастических систем\cite{AUSLENDER}.


% пример типовой задачи
% coding: utf-8
% --------------------------------------------------------------------------------------------------
% "Синтез оптимального стохастического управления", 2011 год
% --------------------------------------------------------------------------------------------------



\chapter{Пример решения типовой задачи}
% ==============================================================================================
\renewcommand{\optU}{  \optimum{\m{u}} } % оптимальное управление
\renewcommand{\funcF}{ \calf{F}        } % множество F
% ==============================================================================================



% **********************************************************************************************
\section{Постановка задачи}
% **********************************************************************************************



Рассмотрим полученные в разделах 2 и 3 результаты, применим их к конкретной задаче.

Пускай стоит задача вывода на орбиту искусственного спутника. В подобных задачах ракета-носитель должна следовать по номинальной траектории. Но из-за нестационарности параметров, этого не будет. Для коррекции небольших отклонений относительно заданной траектории можно, например, линеаризовать динамику ракеты-носителя относительно заданной траектории. После выбора критерия, можно приступать к расчетам в соответствии с методиками этого раздела.

Динамику линеаризованного контура наведения ракеты можно выразить в следующей форме:

\beq{eq:4/1/1}
	\eqsystem{
		\dot{x}_1 &= x_2 \mbox{,} \\
		\dot{x}_2 &= \frac{k_1}{k_2 - t} x_3 \mbox{,} \\
		\dot{x}_3 &= u \mbox{.}
	}
\eeq

где

\begin{description}
	\item[$x_1$]~--- боковое отклонение от номинальной траектории;
	\item[$x_2$]~--- скорость этого отклонения;
	\item[$x_3$]~--- угол направления вектора тяги.
\end{description}

Зависимость между боковой тягой и боковым ускорением нестационарна и определяется коэффициентом $k_1 / (k_2 - t)$, который учитывает потерю массы при действии тяги. Интегрируемая связь между $x_3$ и $u$ представляет собой линеаризованное уравнение привода.

\addfigure{fig:1}{x_static_noise_high}{Функции матрицы $\m{P}$.}

Если исходить из предположения о возможности точного измерения $x_1(t)$, $x_2(t)$ и $x_3(t)$, то можно разработать такую систему управления с обратной связью, чтобы обеспечить оптимальное наведение ракеты, удовлетворив при этом заданному показателю качества. Предположим, что используется функционал вида

\beq{eq:4/1/2}
	\funcF = \frac{1}{2} \int\limits_0^T \bigl( \m{x}^T(\tau)\m{Q}(\tau)\m{x}(\tau) + ru(t)^2 \bigr)\,d\tau \mbox{,}
\eeq

где

\beq{eq:4/1/3}
	\m{Q}(t) = \frac{1}{(300-t)^2} \matr{
		5\cdot 10^{-7} & 0       & 0    \\
		0              & 10^{-3} & 0    \\
		0              & 0       & 10^3	
	}
\eeq

Систему~\ref{eq:4/1/1} можно представить в обозначениях раздела 2.2, а именно в виде системы $\m{x}(t) = \m{A}(t)\m{x}(t) + \m{B}(t)\m{u}(t)$. Это возможно, поскольку матрицы $\m{A}$ и $\m{B}$ выражаются как

\beqarr
	\lbl{eq:4/1/4}
		\m{A}(t) = \matr{
			0 & 1 & 0 \\
			0 & 0 & \frac{k_1}{k_2-t} \\
			0 & 0 & 0		
		} \mbox{;} \\
	\lbl{eq:4/1/5}
		\m{B}(t) = \m{b} = \matr{
			0 \\ 0 \\ 1		
		} \mbox{.}
\eeqarr

Иными словами, задача выражается как задача из раздела 2.2 с критерием качества~\vref{eq:2/2/11}. Как было показано, для ее решения, необходимо получить матрицу $\m{P}(t)$ с граничным условием $\m{P}(t) = \m{0}$. Как было показано ранее, оптимальное уравнение в данном случае равно

\beq{eq:4/1/6}
	\optU(t) = -\frac{1}{r}\m{B}(t)^T\m{P}(t)\m{x}(t) = -\frac{1}{r}\m{b}^T\m{P}(t)\m{x}(t) \mbox{.}
\eeq

С другой стороны, так как

\beq{eq:4/1/7}
	\m{P}(t) = \matr{
		p_{11}(t) & p_{21}(t) & p_{31}(t) \\
		p_{12}(t) & p_{22}(t) & p_{32}(t) \\
		p_{13}(t) & p_{23}(t) & p_{33}(t) \\
	} \mbox{,}
\eeq

то учитывая~\ref{eq:4/1/5} и~\ref{eq:4/1/6}, получаем, что

\beq{eq:4/1/8}
	\optU(t) = -\frac{1}{r} \matr{ p_{13}(t) & p_{23}(t) & p_{33}(t) } \m{x}(t) \mbox{.}
\eeq




% **********************************************************************************************
\section{Исследование и анализ алгоритмов решения поставленной задачи}
% **********************************************************************************************



Положим для определенности $r = 10$, $T = 250$ (c).

Для того, чтобы получить матрицу $\m{P}(t)$ требуется решить уравнение~\vref{eq:2/2/7}. Непосредственно решая его, в силу~\ref{eq:4/1/3},~\ref{eq:4/1/4} и~\ref{eq:4/1/5} получаем следующую систему нелинейных дифференциальных уравнений с граничными условиями в одной точке:

\beq{eq:4/1/9}
	\eqsystem{
		\dot{p}_{11}(t) &= \frac{\mathstrut 1}{10}p^2_{13}(t) - \frac{5 \cdot 10^{-7}}{(300-t)^2}         \mbox{,} \\
		\dot{p}_{12}(t) &= \frac{1}{10}p_{13}(t)p_{23}(t) - p_{11}(t)                                     \mbox{,} \\
		\dot{p}_{13}(t) &= \frac{1}{10}p_{13}(t)p_{33}(t) - \frac{k_1}{k_2-t} p_{12}(t)                   \mbox{,} \\
		\dot{p}_{21}(t) &= 0                                                                              \mbox{,} \\
		\dot{p}_{22}(t) &= \frac{1}{10}p^2_{23}(t) - 2p_{12}(t) - \frac{10^{-3}}{(300-t)^2}               \mbox{,} \\
		\dot{p}_{23}(t) &= \frac{1}{10}p_{23}(t)p_{33}(t) - \frac{k_1}{k_2-t}p_{22}(t) - p_{13}(t)        \mbox{,} \\
		\dot{p}_{31}(t) &= 0                                                                              \mbox{,} \\
		\dot{p}_{32}(t) &= 0                                                                              \mbox{,} \\
		\dot{p}_{33}(t) &= \frac{1}{10}p^2_{33}(t) - 2\frac{k_1}{k_2-t}p_{23}(t) - \frac{10^3}{(300-t)^2} \mbox{,}
	}
\eeq

Эту задачу, вследствие нелинейности системы~\ref{eq:4/1/9} нужно решать численно, двигаясь в обратном времени от $t=T$ с граничным условием $\m{P}(T) = \m{0}$.

Таким образом, можно сформулировать первый алгоритм решения задачи:

\balgo{alg:1}
	\benum
		\item
			Для заданных $T$, $r$, $\m{Q}(е)$ требуется решить матричное дифференциальное уравнение~\ref{eq:2/2/7} в обратном времени с граничным условием $\m{P}(T) = \m{0}$. Таким образом, получим траекторию $\m{P}(t)$, где $0 \leqslant t \leqslant T$;
		\item
			Пользуясь формулой~\ref{eq:4/1/8} получаем оптимальное управление $\optU(t)$ на интервале $0 \leqslant t \leqslant T$ в каждый момент времени $t$. Обратим внимание, что таким образом синтезируется управление с обратной связью, следовательно необходимо знать точные значения каждой переменной состояния из вектора $\m{x}(t)$ в рассматриваемый момент времени.
	\eenum
\ealgo

% TODO: пример решения

Рассмотрим теперь случай, когда система стационарна. Как было показано ранее, при $T \to \infty$, уравнение~\vref{eq:2/2/7} сводится к нелинейному алгебраическому уравнению Риккати~\vref{eq:2/2/10}. Остальные действия остаются прежними. Следовательно, можно построить новый алгоритм, берущий за основу алгоритм~\ref{alg:1}.

Формула~\ref{eq:4/1/8} также преобразуется с учетом этой особенности:

\beq{eq:4/1/10}
	\optU(t) = -\frac{1}{r} \matr{ p_{13} & p_{23} & p_{33} } \m{x}(t) \mbox{.}
\eeq

Таким образом, можно сформулировать новый алгоритм.

\balgo{alg:2}
	\benum
		\item
			Для заданных $T$, $r$, $\m{Q}$ требуется решить матричное алгебраическое уравнение~\ref{eq:2/2/10}. Таким образом, получим матрицу с постоянными коэффициентами $\m{P}$ для всех $0 \leqslant t \leqslant T$;
		\item
			Пользуясь формулой~\ref{eq:4/1/10} получаем оптимальное управление $\optU(t)$ на интервале $0 \leqslant t \leqslant T$ в каждый момент времени $t$. Опять же, стоит обратить внимание на то, что таким образом синтезируется управление с обратной связью, следовательно необходимо знать точные значения каждой переменной состояния из вектора $\m{x}(t)$ в рассматриваемый момент времени.
	\eenum
\ealgo


% **********************************************************************************************
\section{Программная реализация решения}
% **********************************************************************************************



% **********************************************************************************************
\section{Результаты решения}
% **********************************************************************************************


% заключение
% coding: utf-8
% --------------------------------------------------------------------------------------------------
% "Синтез оптимального стохастического управления", 2011 год
% --------------------------------------------------------------------------------------------------



\chapter*{Заключение}
\addcontentsline{toc}{chapter}{Заключение}



В работе были исследованы вопросы синтеза оптимального регулятора с использованием принципа комбинированного управления, когда нейтрализовывалось влияние превалирующего возмущения (принцип компенсации), а побочные возмущения устранялись с использованием принципа обратной связи. Подробно исследовались вопросы синтеза оптимального регулятора с привлечением аппаратов динамического программирования; была установлена связь с принципами максимума; было показано, как свести методы к уравнению Беллмана. Полученные результаты удалось обобщить на случай многомерных стохастических систем и доказать принцип полной эквивалентности при синтезе оптимального регулятора для этого случая, был исследован вопрос стохастической устойчивости.

Результаты были применены к задаче синтеза управления для стабилизации полета баллистической ракеты в турбулентной атмосфере. Исследовались как вывод на <<номинальную>> траекторию, так и удержание курса на ней. Для данной задачи, используя теоретические результаты, были сконструированы алгоритмы нахождения оптимального управления, дано их математическое обоснование. Был также проведен анализ этих алгоритмов и установлены условия их применимости.

Кроме того, был разработан математический программный пакет, позволяющий программно смоделировать регулятор и использовать алгоритмы для проведения компьютерных экспериментов. Эти эксперименты не только подтвердили теоретические результаты, но и позволили обратить внимание на проблему, связанную с тем, что для наведения с использованием алгоритмов достаточно большую роль играет время наведения. Если время наведения оказывается недостаточным, то алгоритмы не только не позволяют решить поставленную задачу, но даже ухудшают результаты. Были даны численные оценки порогового значения времени наведения для конкретной задачи.


% список литературы
\addcontentsline{toc}{chapter}{Список использованной литературы}

\begin{thebibliography}{wlab}

% 2 раздел

\bibitem{YES}
\emph{Ширяев А.\,Н.}
\newblock Вероятность.
\newblock М.: Наука, 1980, 576 с.

\bibitem{BELLMAN1}
\emph{Беллман Р.}
\newblock Динамическое программирование.
\newblock М.: Изд-во иностранной лит., 1960, 472 с.

\bibitem{TCHAMRAN}
\emph{Tchamran A.}
\newblock On Bellman's Functional Equation and a Class of Time-Optimal Control Systems.
\newblock Journal of the Franklin Inst., 1965, vol.\,280, no.\,6, 493--505 pp.

\bibitem{FICHTENGOLZ}
\emph{Фихтенгольц Г.\,М.}
\newblock Основы математического анализа.
\newblock М.: Физматлит, 2008, т.\,2, 728 c.

\bibitem{OXENDAL}
\emph{Оксендаль Б.}
\newblock Стохастические дифференциальные уравнения
\newblock М.: АСТ, 2003, 408 с.

\bibitem{MERRIAM}
\emph{Мэрриэм Ч.\,В.}
\newblock Теория оптимизации и расчет систем управления с обратной связью
\newblock М.: Мир, 1967, 491 с.

\bibitem{KALMAN1}
\emph{Калман Р.\,Е.}
\newblock Когда линейная система управления является оптимальной?
\newblock Теоретические основы инженерных расчетов, 1964, т.\,86, №\,1, 69--84 с.

\newpage

\bibitem{XU}
\emph{Сю Д., Мейер А.}
\newblock Современная теория автоматического управления и её применение.
\newblock М.: Машиностроение, 1972, 544 с.

\bibitem{KRASOVSKY}
\emph{Красовский А.\,А.}
\newblock Системы автоматического управления полетом и их аналитическое конструирование.
\newblock М.: Наука, 1973, 560 с.

\bibitem{KALMAN2}
\emph{Kalman R.\,E.}
\newblock New methods and Results in Linear Prediction and Filterig Theory.
\newblock NY: John Wiley and Sons, 1963, 51 p.

\bibitem{BELLMAN2}
\emph{Беллман Р.}
\newblock Введение в теорию матриц.
\newblock М.: Наука, 1969, 702 с.

% 3 раздел

\bibitem{SKOROHOD}
\emph{Скороход А.\,В.}
\newblock Случайные процессы с независимыми приращениями.
\newblock М.: Наука, 1964, 280 с.

\bibitem{KLOEDEN}
\emph{Kloeden P.\,E., Platen E.}
\newblock Numerical solution of stochastic differential equations.
\newblock Berlin: Springer-Verlag, 1992, 632 p.

\bibitem{KUZNETCOV}
\emph{Кузнецов Д.\,Ф.}
\newblock Численное моделирование стохастических дифференциальных уравнений и стохастических интегралов.
\newblock СПб.: Наука, 1999, 459 с.

\bibitem{KUSHNER}
\emph{Кушнер Г.\,Дж.}
\newblock Стохастическая устойчивость и управление.
\newblock М.: Мир, 1969, 200 с.

\bibitem{HASMINSKI}
\emph{Hasminski R.\,Z.}
\newblock Stochastic Stability of Differential Equations.
\newblock Sijthoff \& Noordhoff, Alphen naan den Rijn, 1980, 527 p.

\bibitem{ARNOLD}
\emph{Arnold L., Wihstutz V.}
\newblock Lyapunov exponents.
\newblock Springer Lecture Notes in Mathematics, 1986, Vol. 1186, 561--588 pp.

\bibitem{AUSLENDER}
\emph{Ауслендер Э.\,И., Мильштейн Г.\,Н.}
\newblock Асимптотические разложения показателя Ляпунова для линейных стохастических систем с малыми шумами.
\newblock Прикладная математика и механика, 1982, т.\,46, 3 вып., 358--365~c.

\end{thebibliography}




\end{document}
