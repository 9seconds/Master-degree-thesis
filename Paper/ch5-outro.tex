% coding: utf-8
% --------------------------------------------------------------------------------------------------
% "Синтез оптимального стохастического управления", 2011 год
% --------------------------------------------------------------------------------------------------



\chapter*{Заключение}
\addcontentsline{toc}{chapter}{Заключение}



В работе были исследованы вопросы синтеза оптимального регулятора с использованием принципа комбинированного управления, когда нейтрализовывалось влияние превалирующего возмущения (принцип компенсации), а побочные возмущения устранялись с использованием принципа обратной связи. Подробно исследовались вопросы синтеза оптимального регулятора с привлечением аппаратов динамического программирования; была установлена связь с принципами максимума; было показано, как свести методы к уравнению Беллмана. Полученные результаты удалось обобщить на случай многомерных стохастических систем и доказать принцип полной эквивалентности при синтезе оптимального регулятора для этого случая, был исследован вопрос стохастической устойчивости.

Результаты были применены к задаче синтеза управления для стабилизации полета баллистической ракеты в турбулентной атмосфере. Исследовались как вывод на <<номинальную>> траекторию, так и удержание курса на ней. Для данной задачи, используя теоретические результаты, были сконструированы алгоритмы нахождения оптимального управления, дано их математическое обоснование. Был также проведен анализ этих алгоритмов и установлены условия их применимости.

Кроме того, был разработан математический программный пакет, позволяющий программно смоделировать регулятор и использовать алгоритмы для проведения компьютерных экспериментов. Эти эксперименты не только подтвердили теоретические результаты, но и позволили обратить внимание на проблему, связанную с тем, что для наведения с использованием алгоритмов достаточно большую роль играет время наведения. Если время наведения окзывается недостаточным, то алгоритмы не только не позволяют решить поставленную задачу, но даже ухудшают результаты. Были даны численные оценки порогового значения времени наведения для конкретной задачи.
