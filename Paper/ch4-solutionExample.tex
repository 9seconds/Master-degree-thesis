% coding: utf-8
% --------------------------------------------------------------------------------------------------
% "Синтез оптимального стохастического управления", 2011 год
% --------------------------------------------------------------------------------------------------



\chapter{Пример решения типовой задачи}
% ==============================================================================================
\renewcommand{\optU}{  \optimum{\m{u}} } % оптимальное управление
\renewcommand{\funcF}{ \calf{F}        } % множество F
% ==============================================================================================



% **********************************************************************************************
\section{Постановка задачи}
% **********************************************************************************************



Рассмотрим полученные в разделах 2 и 3 результаты, применим их к конкретной задаче.

Пускай стоит задача вывода на орбиту искусственного спутника. В подобных задачах ракета-носитель должна следовать по номинальной траектории. Но из-за нестационарности параметров, этого не будет. Для коррекции небольших отклонений относительно заданной траектории можно, например, линеаризовать динамику ракеты-носителя относительно заданной траектории. После выбора критерия, можно приступать к расчетам в соответствии с методиками этого раздела.

Динамику линеаризованного контура наведения ракеты можно выразить в следующей форме:

\beq{eq:4/1/1}
	\eqsystem{
		\dot{x}_1 &= x_2 \mbox{,} \\
		\dot{x}_2 &= \frac{k_1}{k_2 - t} x_3 \mbox{,} \\
		\dot{x}_3 &= u \mbox{.}
	}
\eeq

где

\begin{description}
	\item[$x_1$]~--- боковое отклонение от номинальной траектории;
	\item[$x_2$]~--- скорость этого отклонения;
	\item[$x_3$]~--- угол направления вектора тяги.
\end{description}

Зависимость между боковой тягой и боковым ускорением нестационарна и определяется коэффициентом $k_1 / (k_2 - t)$, который учитывает потерю массы при действии тяги. Интегрируемая связь между $x_3$ и $u$ представляет собой линеаризованное уравнение привода.

Если исходить из предположения о возможности точного измерения $x_1(t)$, $x_2(t)$ и $x_3(t)$, то можно разработать такую систему управления с обратной связью, чтобы обеспечить оптимальное наведение ракеты, удовлетворив при этом заданному показателю качества. Предположим, что используется функционал вида

\beq{eq:4/1/2}
	\funcF = \frac{1}{2} \int\limits_0^T \bigl( \m{x}^T(\tau)\m{Q}(\tau)\m{x}(\tau) + ru(t)^2 \bigr)\,d\tau \mbox{,}
\eeq

где

\beq{eq:4/1/3}
	\m{Q}(t) = \frac{1}{(300-t)^2} \matr{
		5\cdot 10^{-7} & 0       & 0    \\
		0              & 10^{-3} & 0    \\
		0              & 0       & 10^3	
	}
\eeq

Систему~\ref{eq:4/1/1} можно представить в обозначениях раздела 2.2, а именно в виде системы $\m{x}(t) = \m{A}(t)\m{x}(t) + \m{B}(t)\m{u}(t)$. Это возможно, поскольку матрицы $\m{A}$ и $\m{B}$ выражаются как

\beqarr
	\lbl{eq:4/1/4}
		\m{A}(t) = \matr{
			0 & 1 & 0 \\
			0 & 0 & \frac{k_1}{k_2-t} \\
			0 & 0 & 0		
		} \mbox{;} \\
	\lbl{eq:4/1/5}
		\m{B}(t) = \m{b} = \matr{
			0 \\ 0 \\ 1		
		} \mbox{.}
\eeqarr

Иными словами, задача выражается как задача из раздела 2.2 с критерием качества~\vref{eq:2/2/11}. Как было показано, для ее решения, необходимо получить матрицу $\m{P}(t)$ с граничным условием $\m{P}(t) = \m{0}$. Как было показано ранее, оптимальное уравнение в данном случае равно

\beq{eq:4/1/6}
	\optU(t) = -\frac{1}{r}\m{B}(t)^T\m{P}(t)\m{x}(t) = -\frac{1}{r}\m{b}^T\m{P}(t)\m{x}(t) \mbox{.}
\eeq

С другой стороны, так как

\beq{eq:4/1/7}
	\m{P}(t) = \matr{
		p_{11}(t) & p_{21}(t) & p_{31}(t) \\
		p_{12}(t) & p_{22}(t) & p_{32}(t) \\
		p_{13}(t) & p_{23}(t) & p_{33}(t) \\
	} \mbox{,}
\eeq

то учитывая~\ref{eq:4/1/5} и~\ref{eq:4/1/6}, получаем, что

\beq{eq:4/1/8}
	\optU(t) = -\frac{1}{r} \matr{ p_{13}(t) & p_{23}(t) & p_{33}(t) } \m{x}(t) \mbox{.}
\eeq

\input{fig_x_adaptive_noise_high}


% **********************************************************************************************
\section{Исследование и анализ алгоритмов решения поставленной задачи}
% **********************************************************************************************



Положим для определенности $r = 10$, $T = 250$ (c).

Для того, чтобы получить матрицу $\m{P}(t)$ требуется решить уравнение~\vref{eq:2/2/7}. Непосредственно решая его, в силу~\ref{eq:4/1/3},~\ref{eq:4/1/4} и~\ref{eq:4/1/5} получаем следующую систему нелинейных дифференциальных уравнений с граничными условиями в одной точке:

\beq{eq:4/1/9}
	\eqsystem{
		\dot{p}_{11}(t) &= \frac{\mathstrut 1}{10}p^2_{13}(t) - \frac{5 \cdot 10^{-7}}{(300-t)^2}         \mbox{,} \\
		\dot{p}_{12}(t) &= \frac{1}{10}p_{13}(t)p_{23}(t) - p_{11}(t)                                     \mbox{,} \\
		\dot{p}_{13}(t) &= \frac{1}{10}p_{13}(t)p_{33}(t) - \frac{k_1}{k_2-t} p_{12}(t)                   \mbox{,} \\
		\dot{p}_{21}(t) &= 0                                                                              \mbox{,} \\
		\dot{p}_{22}(t) &= \frac{1}{10}p^2_{23}(t) - 2p_{12}(t) - \frac{10^{-3}}{(300-t)^2}               \mbox{,} \\
		\dot{p}_{23}(t) &= \frac{1}{10}p_{23}(t)p_{33}(t) - \frac{k_1}{k_2-t}p_{22}(t) - p_{13}(t)        \mbox{,} \\
		\dot{p}_{31}(t) &= 0                                                                              \mbox{,} \\
		\dot{p}_{32}(t) &= 0                                                                              \mbox{,} \\
		\dot{p}_{33}(t) &= \frac{1}{10}p^2_{33}(t) - 2\frac{k_1}{k_2-t}p_{23}(t) - \frac{10^3}{(300-t)^2} \mbox{,}
	}
\eeq

Эту задачу, вследствие нелинейности системы~\ref{eq:4/1/9} нужно решать численно, двигаясь в обратном времени от $t=T$ с граничным условием $\m{P}(T) = \m{0}$.

Таким образом, можно сформулировать первый алгоритм решения задачи:

\balgo{alg:1}
	\benum
		\item
			Для заданных $T$, $r$, $\m{Q}(е)$ требуется решить матричное дифференциальное уравнение~\ref{eq:2/2/7} в обратном времени с граничным условием $\m{P}(T) = \m{0}$. Таким образом, получим траекторию $\m{P}(t)$, где $0 \leqslant t \leqslant T$;
		\item
			Пользуясь формулой~\ref{eq:4/1/8} получаем оптимальное управление $\optU(t)$ на интервале $0 \leqslant t \leqslant T$ в каждый момент времени $t$. Обратим внимание, что таким образом синтезируется управление с обратной связью, следовательно необходимо знать точные значения каждой переменной состояния из вектора $\m{x}(t)$ в рассматриваемый момент времени.
	\eenum
\ealgo

% TODO: пример решения

Рассмотрим теперь случай, когда система стационарна. Как было показано ранее, при $T \to \infty$, уравнение~\vref{eq:2/2/7} сводится к нелинейному алгебраическому уравнению Риккати~\vref{eq:2/2/10}. Остальные действия остаются прежними. Следовательно, можно построить новый алгоритм, берущий за основу алгоритм~\ref{alg:1}.

Формула~\ref{eq:4/1/8} также преобразуется с учетом этой особенности:

\beq{eq:4/1/10}
	\optU(t) = -\frac{1}{r} \matr{ p_{13} & p_{23} & p_{33} } \m{x}(t) \mbox{.}
\eeq

Таким образом, можно сформулировать новый алгоритм.

\balgo{alg:2}
	\benum
		\item
			Для заданных $T$, $r$, $\m{Q}$ требуется решить матричное алгебраическое уравнение~\ref{eq:2/2/10}. Таким образом, получим матрицу с постоянными коэффициентами $\m{P}$ для всех $0 \leqslant t \leqslant T$;
		\item
			Пользуясь формулой~\ref{eq:4/1/10} получаем оптимальное управление $\optU(t)$ на интервале $0 \leqslant t \leqslant T$ в каждый момент времени $t$. Опять же, стоит обратить внимание на то, что таким образом синтезируется управление с обратной связью, следовательно необходимо знать точные значения каждой переменной состояния из вектора $\m{x}(t)$ в рассматриваемый момент времени.
	\eenum
\ealgo


% **********************************************************************************************
\section{Программная реализация решения}
% **********************************************************************************************



% **********************************************************************************************
\section{Результаты решения}
% **********************************************************************************************
