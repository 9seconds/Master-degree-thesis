% coding: utf-8
% --------------------------------------------------------------------------------------------------
% "Синтез оптимального стохастического управления", 2011 год
% --------------------------------------------------------------------------------------------------
\chapter{Постановка задачи}



Предположим, что имеется некоторый физический объект, чья динамическое состояние описывается с помощью вектора параметров, полностью определяющих его положение в некотором пространстве и достаточных для того, чтобы однозначно определить его положение в следующий момент времени. Динамику его перемещений можно описать с помощью системы дифференциальных уравнений. Не нарушая общности положим, что динамика описывается с помощью системы линейных дифференциальных уравнений

\beq{eq:1/1}
	\eqsystem{
		\dot{x}_1 = a_{11}(t)x_1(t) + a_{12}(t)x_2(t) + \cdots + a_{1n}(t)x_n(t) \text {,} \\
		\dot{x}_2 = a_{21}(t)x_1(t) + a_{22}(t)x_2(t) + \cdots + a_{2n}(t)x_n(t) \text {,} \\
		\cdots \\
		\dot{x}_n = a_{n1}(t)x_1(t) + a_{n2}(t)x_2(t) + \cdots + a_{nn}(t)x_n(t) \text {.} \\
	}
\eeq