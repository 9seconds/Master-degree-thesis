% coding: utf-8
% --------------------------------------------------------------------------------------------------
% "Синтез оптимального стохастического управления", 2011 год
% --------------------------------------------------------------------------------------------------



\chapter{Постановка задачи}
% ==============================================================================================
\newcommand{\funcF}{ \calf{F}         } % функция качества
\newcommand{\optF}{  \optimum{\funcF} } % оптимальная функция F
% ==============================================================================================



Предположим, что имеется некоторый физический объект, чья динамическое состояние описывается с помощью вектора параметров, полностью определяющих его положение в некотором пространстве и достаточных для того, чтобы однозначно определить его положение в следующий момент времени. Динамику его перемещений можно описать с помощью системы дифференциальных уравнений. Не нарушая общности положим, что динамика описывается с помощью системы линейных дифференциальных уравнений

\beq{eq:1/1}
    \eqsystem{
        \dot{x}_1 = a_{11}(t)x_1(t) + a_{12}(t)x_2(t) + \cdots + a_{1n}(t)x_n(t) \text {,} \\
        \dot{x}_2 = a_{21}(t)x_1(t) + a_{22}(t)x_2(t) + \cdots + a_{2n}(t)x_n(t) \text {,} \\
        \vdots                                                                             \\
        \dot{x}_n = a_{n1}(t)x_1(t) + a_{n2}(t)x_2(t) + \cdots + a_{nn}(t)x_n(t) \text {.} \\
    }
\eeq

В матричном виде уравнение~\ref{eq:1/1} можно переписать следующим образом:

\beq{eq:1/2}
    \dot{\m{x}}(t) = \m{A}(t)\m{x}(t) \text{,}
\eeq

где $\m{x}(t) \eqdef \bigl(x_1(t), x_2(t), \ldots, x_n(t)\bigr) \in \Rv{n}$, а матрица $\m{A}(t) \in \Rs{n}{n}$ и представляется как

\beqn
    \m{A}(t) \eqdef \matr{
        a_{11}(t) & a_{12}(t) & \cdots & a_{1n}(t) \\
        a_{21}(t) & a_{22}(t) & \cdots & a_{2n}(t) \\
        \vdots    & \vdots    & \ddots & \vdots    \\
        a_{n1}(t) & a_{n2}(t) & \cdots & a_{nn}(t)
    } \text{.}
\eeqn

Предположим также, что имеется некоторый способ воздействовать на динамику системы, чтобы было возможным перевести ее из начального состояния $t_0$ в конечное состояние $T$ так, что $\m{x}(T)$ удовлетворяла бы ряду заранее заданных условий. Такие воздействия называются \emph{управлением} и моделируются с помощью вектора $\m{u}(t) \eqdef \bigl(u_1(t), u_2(t), \ldots u_m(t) \bigr) \in \Rv{m}$.

Вводя управление, систему~\ref{eq:1/2} можно переписать следующим образом:

\beq{eq:1/3}
    \dot{\m{x}}(t) = \m{A}(t)\m{x}(t) + \m{B}(t)\m{u}(t) \text{,}
\eeq

где матрица $\m{B}(t) \in \Rs{n}{m}$ и представляется как

\beqn
    \m{B}(t) \eqdef \matr{
        b_{11}(t) & b_{12}(t) & \cdots & b_{1m}(t) \\
        b_{21}(t) & b_{22}(t) & \cdots & b_{2m}(t) \\
        \vdots    & \vdots    & \ddots & \vdots    \\
        b_{n1}(t) & b_{n2}(t) & \cdots & b_{nm}(t)
    } \text{.}
\eeqn

В работе, не нарушая общности, будут рассматриваться матрицы $\m{B}(t)$ с постоянными значениями, то есть $\m{B}(t) \equiv \m{B}$. Такие предположения сделаны лишь для того, чтобы упростить выкладки.

Используя различное управление, система будет переходить в различные конечные состояния $\m{x}(T)$: используя управление $\m{u}_1(t)$, система будет переведена в конечное состояние $\m{x}_1(t)$, управление $\m{u}_2(t)$~--- в $\m{x}_2(T)$ и так далее. Как правило, имеет значение, в котором окажется система в конечный момент времени. Управление\footnote{Конечно же, оптимальное управление в общем случае не является единственным.}, которое переводит систему в требуемое состояние $\optimum{\m{x}}(t)$ будет называться \emph{оптимальным} и обозначаться соответственно $\optimum{\m{u}}(t)$. Траекторию изменения состояний от момента времени $t_0$ до $T$ также назовем \emph{оптимальной}.

\br

Зачастую имеет значение не только факт, что в определенный момент времени система перешла в нужное состояние, но и то, как был совершен этот переход. Иными словами, требуется учитывать предысторию перехода, всю динамику системы с использованием управлений. Чтобы учитывать такое <<качество>> перехода, вводятся различные \emph{критерии качества}~--- функции, выражающие качество перехода через некоторые количественные значения. Так как для вещественных чисел определены операции сравнения, то появляется возможность таким образом сравнивать траектории и выделять оптимальное уже в этом случае управление. Иногда даже не имеет значения, в каком конечном состоянии окажется система, важно, чтобы переход осуществлялся так, чтобы он удовлетворял некоторому критерию качества сам по себе.

Естественным образом возникает задача минимизации (или максимизации) такого критерия качества $\funcF\bigl(\m{x}(t), \m{u}(t)\bigr) \colon \Rv{n} \times \Rv{m} \to \Rv{1}$. Для определенности рассмотрим задачу минимизации критерия. Минимальное значение $\funcF\bigl(\m{x}(t), \m{u}(t)\bigr)$ обозначим как $\optF\bigl(\m{x}(t), \m{u}(t)\bigr)$.

При этих предположениях, задачу можно сформулировать таким образом: требуется найти управление $\optimum{\m{u}}(t)$ для системы~\ref{eq:1/2} и заданного критерия качества $\funcF$ при условии

\beq{eq:1/4}
    \funcF\bigl(\m{x}(t), \m{u}(t)\bigr) \to \min \text{.}
\eeq

Найденное подобным образом управление, очевидно, будет являться оптимальным.

\br

Однако система~\ref{eq:1/3} помимо воздействия управления может подвергаться и воздействиям различных внешних факторов, которые нельзя проконтролировать. Воздействие этих факторов непредсказуемо, они оказывают влияние на поведение объекта на протяжение всего интервала наблюдений $[t_0, T]$. Такие случайные воздействия возникают, например, вследствие разнообразных шумов среды, столкновений с другими объектами или скачков энергии. Факторов может быть огромное количество, но при моделировании их можно объединять в группы. Эти воздействия описываются случайными величинами, в частности векторными. В работе будет исследоваться влияние векторной случайной величины $\m{\xi}(t) = \bigl( \xi_1(t), \xi_2(t), \ldots, \xi_k(t) \bigr) \in \Rv{k}$.

Таким образом, задачу~\ref{eq:1/3}--\ref{eq:1/4} можно переписать следующим образом:

\beq{eq:1/5}
    \dot{\m{x}}(t) = \m{A}(t)\m{x}(t) + \m{B}(t)\m{u}(t) + \m{\Sigma}(t)\m{\xi}(t) \text{,}
\eeq

где $\m{\Sigma}(t) \in \Rs{n}{k}$ и записывается как

\beqn
    \m{\Sigma}(t) \eqdef \matr{
        \sigma_{11}(t) & \sigma_{12}(t) & \cdots & \sigma_{1k}(t) \\
        \sigma_{21}(t) & \sigma_{22}(t) & \cdots & \sigma_{2k}(t) \\
        \vdots    & \vdots    & \ddots & \vdots    \\
        \sigma_{n1}(t) & \sigma_{n2}(t) & \cdots & \sigma_{nk}(t)
    } \text{.}
\eeqn

Впрочем, не нарушая общности, как и в случае для матрицы $\m{B}(t)$ будем полагать, что матрица $\m{\Sigma}(t)$ является постоянной, то есть $\m{\Sigma}(t) \equiv \m{\Sigma}$. Критерий качества, очевидно, остается прежним, поскольку факторы, которые нельзя проконтролировать, влияют на саму траекторию системы.

Нахождение оптимальных управлений $\optimum{\m{u}}(t)$ для системы~\ref{eq:1/5} и задачи минимизации~\ref{eq:1/4} и является задачей этой работы.
