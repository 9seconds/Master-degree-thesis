\chapter{Методы стохастического управления}
% ==============================================================================================
\newcommand{\mSig}{    \sff{\Sigma} } % матрица ковариаций

\newcommand{\setX}{    \calf{X}     } % множество X
\newcommand{\setT}{    \calf{T}     } % множество T
\newcommand{\setTime}{ [0, T]       } % временной промежуток, который достало вводить
% ==============================================================================================


% **********************************************************************************************
\section{Некоторые сведения из теории случайных процессов}
% **********************************************************************************************



Перед рассмотрением методов стохастического управления, дадим некоторые сведения из теории вероятностей, в частности из теории случайных процессов.

\begin{df}\lbl{df:1}
    Семейство случайных величин, параметризованное параметров $t$, интерпретируемым как время, будем называть случайным процессом и обозначать как
    
    \beqn
    	\xi_t \eqdef \ksi(t, \omega) \in \setX \subset \Rv{1}
    \eeqn

    где $t \in \setT \subset \Rv{1}$~--- временной параметр; $\setT$~--- множество определения, а $\setX$~--- множество значений случайного процесса $\xi_t$.
\end{df}

Область определения $\setT$ случайного процесса может быть конечным, полубесконечным или бесконечным интервалом числовой прямой. В этих случаях $\xi(t, \omega)$ называется \emph{случайным процессом с непрерывным временем}. Если $\setT$ имеет конечное или счетное множество элементов $t_k \in \setT: t_k < t_{k+1}$ ($k = 0, 1, 2, \ldots$), то $\xi_t$ называется \emph{случайным процессом с дискретным временем} или \emph{случайной последовательностью}.

Случайный процесс считается полностью заданным, если заданы его конечномерные распределения~--- набор функций распределения.

Случайный процесс называется \emph{гауссовским}, если все его совместные плотности распределения являются гауссовскими:

\beqn
	\prob_\xi(x1, x2, \ldots, x_k, t_1, t_2, \ldots, t_k) = \frac{1}{\sqrt{(2\pi)^n \det(\mSig)}}\exp\biggl( -\frac{(x-m)^T\mSig^{-1}(x-m)}{2} \biggr) \mbox{,}
\eeqn

где $x = (x_1, x_2, \ldots, x_k)^T$; $m = (m_1, m_2, \ldots, m_k)^T$; $m_i = \E x_i$; $\mSig = \mSig^T > 0$~--- матрица ковариаций.

Случайный процесс $\xi_t$, заданный на $\setT$, называется \emph{стационарным в узком смысле}, если для любых $n \geqslant 1$ и любых моментов $t_1, t_2, \ldots, t_n, t$, таких, что $t_i + t \in \setT$ ($i = 1, 2, \ldots, n$), совместные функции распределения совокупности случайных величин $\xi_{t_1+t}, \xi_{t_2+t}, \ldots, \xi_{t_n+1}$ не зависят от $t$.

\begin{df}\lbl{df:2}
	Случайный процесс $\xi_t$, определенный при всех $t \in \setTime$, называется непрерывным в момент времени $t^{*} \in \setTime$ в вероятностном смысле, если для любой последовательности $\{t_k\}_{k=1}^\infty$, принадлежащей $[0, T]$ и сходящейся к $t^{*}$ при $k \to \infty$, справедливо предельное соотношение
	
	\beqn
		\xi_{t_k} \to \xi_{t^{*}} \mbox{ при } t_k \to t^{*}  \mbox{,}
	\eeqn
	
	понимаемое в соответствующем вероятностном смысле.
\end{df}

\begin{df}\lbl{df:3}
	Случайный процесс $\xi_t$ называется непрерывным на промежутке $[t_1, t_2]$ в данном вероятностном смысле, если он непрерывен в этом смысле при $\forall t \in [t_1, t_2]$. Непрерывность в среднем степени 2 случайного процесса $\xi_t$ будем называть среднеквадратической непрерывностью.
\end{df}

\begin{df}\lbl{df:4}
	Случайный процесс $\xi_t$, определеный для всех $t \in \setTime$, называется дифференцируемым в момент $t \in \setTime$ в одном из вероятностных смыслов, если существует левая часть следующего предельного соотношения:
	
	\beqn
		\lim\limits_{\delta \to 0} \frac{\xi_{t+\delta} - \xi_t}{\delta} = \genericdiff{\xi_t}{t} \mbox{,}
	\eeqn
	
	понимаемого в этом вероятностном смысле. Процесс $\xi_t$ будем называть дифференцируемым на промежутке $[t_1, t_2] \subset \setTime$ в данном вероятностном смысле, если он дифференцируем в этом смысле при всех $t \in [t_1, t_2]$.
\end{df}

\begin{df}\lbl{df:5}
	Случайный процесс $\xi_t$, определеный для всех $t \in \setTime$, называется процессом с независимыми приращениями, если для любых $t_0, t_1, \ldots, t_k \in \setTime$ ($0 \leqslant t_0 < t_1 < \ldots < t_k \leqslant T$), случайные величины $\xi_{t_0}, \xi_{t_1}-\xi_{t_0}, \ldots, \xi_{t_k}-\xi_{t_{k-1}}$ стохастически независимы между собой.
\end{df}

Особую роль в дальнейшем изложении будут играть так называемые винеровские случайные процессы, относящиеся к классу гауссовских процессов с независимыми приращениями\footnote{ Подробное изложение теории случайных процессов с независимыми приращениями можно найти в монографии А.\,В.\,Скорохода (TODO) }.

\begin{df}\lbl{df:5}
	Случайный процесс $\xi_t \in \Rv{1}$, определеный для всех $t \in \setTime$, называется винеровским, если он является процессом с независимыми приращениями и удовлетворяет следующим условиям
	
	\benum
		\item
			$\E \xi_t = 0$,
		
		\item
			$\E (\xi_{t_2} - \xi_{t_1})^2 - \sigma_\xi^2 |t_2 - t_1| \mbox{, где } \sigma_\xi^2 > 0$,
		
		\item
			$\xi_{t_1}$ и $\xi_{t_2} - \xi_{t_1}$ ($t_2 > t_1$) имеют нормальное распредление при всех $t_1, t_2 \in \setTime$.
	\eenum
\end{df}

Если $\sigma_\xi^2$ и $\xi_0 = 0$, то винеровский процесс называется \emph{стандартным}. Поскольку приращения $\xi_{t_2} - \xi_{t_1}$ ($t_2 > t_1$) винеровского процесса имеют нормальное распределение, то их функция распределения определяется соотношением

\beqn
	\prob(\xi_{t_2} - \xi_{t_1} < x) = \frac{1}{\sigma_\xi \sqrt{2\pi (t_2 - t_1)}} \int\limits_\infty^x e^{-\frac{u^2}{2\sigma_xi^2(t_2-t_1)}}\,du \mbox{.}
\eeqn

Известно, что винеровский случайный процесс является непрерывным с вероятностью 1 случайным процессом.














% **********************************************************************************************
\section{Оптимальное стохастическое управление}
% **********************************************************************************************



% **********************************************************************************************
\section{Стохастическая устойчивость}
% **********************************************************************************************