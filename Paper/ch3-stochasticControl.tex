% coding: utf-8
% --------------------------------------------------------------------------------------------------
% "Синтез оптимального стохастического управления", 2011 год
% --------------------------------------------------------------------------------------------------



\chapter{Методы стохастического управления}
% ==============================================================================================
\newcommand{\mSig}{     \sff{\Sigma}                     } % матрица ковариаций

\newcommand{\setTime}{  [0, T]                           } % временной промежуток, который достало вводить
\newcommand{\setX}{     \calf{X}                         } % множество X
\newcommand{\setT}{     \calf{T}                         } % множество T
\newcommand{\setF}{     \calf{F}                         } % множество F
\newcommand{\setM}{     \calf{M}_2\bigl( \setTime \bigr) } % множество M2
\newcommand{\setK}{     \calf{K}_2\bigl( \setTime \bigr) } % множество K2
\newcommand{\setU}{     \calf{U}                         } % множество U
\newcommand{\setC}{     \calf{C}                         } % множество U

\newcommand{\argEmpty}{ \{\emptyarg\}                    } % пустой аргумент

\renewcommand{\funcF}{  \calf{F}                         } % функция качества
\renewcommand{\funcH}{  \calf{H}                         } % функция качества H
\renewcommand{\funcV}{  \sff{V}                          } % функция Ляпунова

\newcommand{\idxB}[1]{  ^{(#1)}                          } % индекс в скобках
% ==============================================================================================



% **********************************************************************************************
\section{Некоторые сведения из теории случайных процессов}
% **********************************************************************************************



Перед рассмотрением методов стохастического управления, дадим некоторые сведения из теории вероятностей, в частности из теории случайных процессов.

\begin{df}\lbl{df:1}
    Семейство случайных величин, параметризованное параметров $t$, интерпретируемым как время, будем называть случайным процессом и обозначать как
    
    \beqn
    	\xi_t \eqdef \xi(t, \omega) \in \setX \subset \Rv{1} \text{,}
    \eeqn

    где $t \in \setT \subset \Rv{1}$~--- временной параметр; $\setT$~--- множество определения, а $\setX$~--- множество значений случайного процесса $\xi_t$.
\end{df}

Область определения $\setT$ случайного процесса может быть конечным, полубесконечным или бесконечным интервалом числовой прямой. В этих случаях $\xi(t, \omega)$ называется \emph{случайным процессом с непрерывным временем}. Если $\setT$ имеет конечное или счетное множество элементов $t_k \in \setT: t_k < t_{k+1}$ ($k = 0, 1, 2, \ldots$), то $\xi_t$ называется \emph{случайным процессом с дискретным временем} или \emph{случайной последовательностью}.

Случайный процесс считается полностью заданным, если заданы его конечномерные распределения~--- набор функций распределения.

Случайный процесс называется \emph{гауссовским}, если все его совместные плотности распределения являются гауссовскими:

\beqn
	\prob_\xi(x1, x2, \ldots, x_k, t_1, t_2, \ldots, t_k) = \frac{1}{\sqrt{(2\pi)^n \det(\mSig)}}\exp\biggl( -\frac{(x-m)^T\mSig^{-1}(x-m)}{2} \biggr) \text{,}
\eeqn

где $x = (x_1, x_2, \ldots, x_k)^T$; $m = (m_1, m_2, \ldots, m_k)^T$; $m_i = \E x_i$; $\mSig = \mSig^T > 0$~--- матрица ковариаций.

Случайный процесс $\xi_t$, заданный на $\setT$, называется \emph{стационарным в узком смысле}, если для любых $n \geqslant 1$ и любых моментов $t_1, t_2, \ldots, t_n, t$, таких, что $t_i + t \in \setT$ ($i = 1, 2, \ldots, n$), совместные функции распределения совокупности случайных величин $\xi_{t_1+t}, \xi_{t_2+t}, \ldots, \xi_{t_n+1}$ не зависят от $t$.

\begin{df}\lbl{df:2}
	Случайный процесс $\xi_t$, определенный при всех $t \in \setTime$, называется непрерывным в момент времени $t^{*} \in \setTime$ в вероятностном смысле, если для любой последовательности $\{t_k\}_{k=1}^\infty$, принадлежащей $[0, T]$ и сходящейся к $t^{*}$ при $k \to \infty$, справедливо предельное соотношение
	
	\beqn
		\xi_{t_k} \to \xi_{t^{*}} \text{ при } t_k \to t^{*}  \text{,}
	\eeqn
	
	понимаемое в соответствующем вероятностном смысле.
\end{df}

\begin{df}\lbl{df:3}
	Случайный процесс $\xi_t$ называется непрерывным на промежутке $[t_1, t_2]$ в данном вероятностном смысле, если он непрерывен в этом смысле при $\forall t \in [t_1, t_2]$. Непрерывность в среднем степени 2 случайного процесса $\xi_t$ будем называть среднеквадратической непрерывностью.
\end{df}

\begin{df}\lbl{df:4}
	Случайный процесс $\xi_t$, определеный для всех $t \in \setTime$, называется дифференцируемым в момент $t \in \setTime$ в одном из вероятностных смыслов, если существует левая часть следующего предельного соотношения:
	
	\beqn
		\lim\limits_{\delta \to 0} \frac{\xi_{t+\delta} - \xi_t}{\delta} = \genericdiff{\xi_t}{t} \text{,}
	\eeqn
	
	понимаемого в этом вероятностном смысле. Процесс $\xi_t$ будем называть дифференцируемым на промежутке $[t_1, t_2] \subset \setTime$ в данном вероятностном смысле, если он дифференцируем в этом смысле при всех $t \in [t_1, t_2]$.
\end{df}

\begin{df}\lbl{df:5}
	Случайный процесс $\xi_t$, определеный для всех $t \in \setTime$, называется процессом с независимыми приращениями, если для любых $t_0, t_1, \ldots, t_k \in \setTime$ ($0 \leqslant t_0 < t_1 < \ldots < t_k \leqslant T$), случайные величины $\xi_{t_0}, \xi_{t_1}-\xi_{t_0}, \ldots, \xi_{t_k}-\xi_{t_{k-1}}$ стохастически независимы между собой.
\end{df}

Особую роль в дальнейшем изложении будут играть так называемые винеровские случайные процессы, относящиеся к классу гауссовских процессов с независимыми приращениями\footnote{ Подробное изложение теории случайных процессов с независимыми приращениями можно найти в монографии А.\,В.\,Скорохода\cite{SKOROHOD} }.

\begin{df}\lbl{df:6}
	Случайный процесс $\xi_t \in \Rv{1}$, определеный для всех $t \in \setTime$, называется винеровским, если он является процессом с независимыми приращениями и удовлетворяет следующим условиям
	
	\benum
		\item
			$\E \xi_t = 0$,
		
		\item
			$\E (\xi_{t_2} - \xi_{t_1})^2 - \sigma_\xi^2 |t_2 - t_1| \text{, где } \sigma_\xi^2 > 0$,
		
		\item
			$\xi_{t_1}$ и $\xi_{t_2} - \xi_{t_1}$ ($t_2 > t_1$) имеют нормальное распредление при всех $t_1, t_2 \in \setTime$.
	\eenum
\end{df}

Если $\sigma_\xi^2$ и $\xi_0 = 0$, то винеровский процесс называется \emph{стандартным}. Поскольку приращения $\xi_{t_2} - \xi_{t_1}$ ($t_2 > t_1$) винеровского процесса имеют нормальное распределение, то их функция распределения определяется соотношением

\beqn
	\prob(\xi_{t_2} - \xi_{t_1} < x) = \frac{1}{\sigma_\xi \sqrt{2\pi (t_2 - t_1)}} \int\limits_\infty^x e^{-\frac{u^2}{2\sigma_xi^2(t_2-t_1)}}\,du \text{.}
\eeqn

Известно, что винеровский случайный процесс является непрерывным с вероятностью 1 случайным процессом.

\br

Определим теперь понятия интегралов и процессов Ито.

Пусть задано вероятностное пространство $(\Omega, \setF, \prob)$ и винеровский случайный процесс $f_t \in \Rv{1}$. Рассмотрим совокупность $\sigma$-алгебр $\{ \setF_t, t \in \setTime \}$, определенную на вероятностном пространстве $(\Omega, \setF, \prob)$ и связанную с винеровским процессом $f_t$ так, что

\benum
	\item
		$\setF_s \subset \setF_t \subset \setF$ при $s<t$;
	
	\item
		$f_t$ измерим относительно $\setF_t$ при каждом $t \in \setTime$;
		
	\item
		Процесс $f_{t+\Delta} - f_\Delta$ при всех $\Delta \geqslant 0$ и $t > 0$ не зависит от событий $\sigma$-алгебры $\setF_\Delta$.
\eenum

В дальнейшем рассматриваются только такие винеровские процессы $f_t$, которые удовлетворяют условиям 2 и 3.

Введем в рассмотрение класс $\setM$ функций $\xi: \setTime \times \Omega \to \Rv{1}$, которые удовлетворяют следующим условиям:

\benum
	\item
		Функции $\xi(t, \omega)$ измеримы по совокупности переменных $(t, \omega)$;
	
	\item
		Функция $\xi(t, \omega) \in \setF_t$ измерима при всех $t \in \setTime$, и ее значения $\xi(\tau, \omega)$ не зависят от приращений $f_{t+\Delta} - f_\Delta$ винеровского процесса при $\Delta \geqslant \tau$, $t > 0$;
	
	\item
		$\int_0^T \E \Bigl\{ \bigl( \xi(t, \omega) \bigr)^2 \Bigr\}\,dt < \infty$;
	
	\item
		$\E \Bigl\{ \bigl( \xi(t, \omega) \bigr)^2 \Bigr\} < \infty$ для всех $t \in \setTime$.
\eenum

Введем на классе $\setM$ норму вида

\beqn
	\| \xi \|_{2,T} = \sqrt{ \int_0^T \E \Bigl\{ \bigl( \xi(t, \omega) \bigr)^2 \Bigr\}\,dt } \text{.}
\eeqn

Считаем две функции $\xi$ и $\eta$ из класса $\setM$ эквивалентными, если $\| \xi - \eta \|_{2,T} = 0$.

Для произвольного разбиения $\{ \tau_j \}_{j=0}^N$ промежутка $\setTime$, такого что $0 = \tau_0 < \tau_1 < \ldots < \tau_N = T$, и произвольной последовательности среднеквадратически интегрируемых, $\setF_{\tau_j}$-измеримых и независящим от приращений $f_{s+t} - f_t$ при $t \geqslant \tau_j$, $s > 0$ случайных величин $\xi_j(\omega)$ ($j = 0, 1, \ldots, N-1$), определим \emph{ступенчатую функцию} $\xi \in \setM$ следующим образом: $\xi(t, \omega) = \xi_j(\omega)$ с вероятностью 1 при $\tau_j \leqslant t < \tau_{j+1}$ ($j = 0, 1, \ldots, N-1$).

При этом интеграл в требовании 3 понимается как

\beqn
	\int\limits_0^T \E \Bigl\{ \bigl( \xi(t, \omega) \bigr)^2 \Bigr\}\,dt = \sum\limits_{j=0}^{N-1} \E \bigl\{ (\xi_j)^2 \bigr\}(\tau_{j+1} - \tau{j}) \text{.}
\eeqn

Обозначим через $\setK$ множество всех ступенчатых функций в $\setM$. Для произвольной ступенчатой функции $\xi$ из класса $\setK$ определим \emph{стохастический интеграл Ито} следующим образом:

\beqn
	\int\limits_0^T \xi_\tau\,df_\tau \eqdef \sum\limits_{j=0}^{N-1} \xi_j(\omega)\bigl( f(\tau_{j+1}, \omega) - f(\tau_j, \omega) \bigr) \text{.}
\eeqn

\begin{df}\lbl{df:7}
	Стохастическим интегралом Ито от функции $\xi \in \setM$ назовем среднеквадратический предел последовательности ступенчатых функций:
	
	\beqn
		\int\limits_0^T \xi_\tau\,df_\tau = \lim\limits_{N \to \infty} \sum\limits_{j=0}^{N-1} \xi\idxB{N}\bigl( f(\tau\idxB{N}_{j+1}, \omega) - f(\tau\idxB{N}_j, \omega) \bigr) \text{,} 
	\eeqn
	
	где $\xi\idxB{N}(t, \omega)$~--- произвольная ступенчатая функция из $\setK$, сходящаяся по норме $\| \emptyarg \|_{2,T}$ к функции $\xi(t, \omega)$.
\end{df}

Известно, что для $\xi \in \setM$ такой интеграл существует\cite{KLOEDEN}.

\br

Винеровские случайные процессы порождают широкий класс случайных процессов, называемых процессами Ито.

Пусть задано фиксированное вероятностное пространство $(\Omega, \setF, \prob)$ и $\setF_t$-измеримый при всех $t \in \setTime$ векторный винеровский процесс $\m{f}_t \in \Rv{m}$ с независимыми компонентами $\m{f}_t\idxB{i}$ ($i = 1, 2, \ldots, m$).

Рассмотрим случайные процессы $a_t \in \Rv{1}$ и $\m{b}_t\idxB{i} \in \Rv{1}$ ($i = 1, 2, \ldots, m$) из класса $\setM$. Определим случайный процесс $\eta_t$ при всех $t \in \setTime$ в виде:

\beq{eq:3/1/1}
	\eta_t = \eta_0 + \int\limits_0^t a_\tau\,d\tau + \sum\limits_{i=1}^m \int\limits_0^t \m{b}_\tau\idxB{i}\,d\m{f}_\tau\idxB{i} \text{,}
\eeq

где $\eta_0$~--- случайная величина, стохастически независимая от приращений $\m{f}_\tau-\m{f}_0$ винеровского процесса $\m{f}_t$ при всех $\tau \in \setTime$.

\begin{df}\lbl{df:8}
	Процесс $\eta_t$, определенный на промежутке $\setTime$ в виде~\ref{eq:3/1/1}, называется процессом Ито, если правая часть~\ref{eq:3/1/1} существует при всех $t \in \setTime$. При этом процесс $a_t$ называется коэффициентом сноса, а процессы $\m{b}_t\idxB{i}$ ($i = 1, 2, \ldots, m$)~--- коэффициентами диффузии процесса $\eta_t$.
\end{df}

Из свойства аддитивности стохастических интегралов вытекает следующее представление для процесса Ито:

\beq{eq:3/1/2}
	\eta_s = \eta_t + \int\limits_t^s a_\tau\,d\tau + \sum\limits_{i=1}^m \int\limits_0^t \m{b}_\tau\idxB{i}\,d\m{f}_\tau\idxB{i} \text{ с вероятностью 1,}
\eeq

справедливое для всех $s, t \in \setTime$ и $s \geqslant t$.

Рассмотрим многомерный случай. Пусть существуют случайные процессы $\m{a}_t \in \Rv{n}$ и $\m{B}_t \in \Rs{n}{m}$, каждая компонента которых должна принадлежать классу $\setM$. Определим векторный случайный процесс $\m{\eta}_t \in \Rv{n}$ при всех $t \in \setTime$ в виде:

\beq{eq:3/1/3}
	\m{\eta}_t = \m{\eta}_0 + \int\limits_0^t \m{a}_\tau\,d\tau + \int\limits_0^t \m{B}_\tau\,d\m{f}_\tau \text{,}
\eeq

где случайная векторная величина $\m{\eta}_0 \in \Rv{n}$ является стохастически независимой от приращений $\m{f}_\tau - \m{f}_0$ винеровского процесса $\m{f}_t$ при всех $\tau \in \setTime$.

\begin{df}\lbl{df:9}
	Процесс $\m{\eta}_t \in \Rv{n}$ вида~\ref{eq:3/1/2}, определенный на промежутке $\setTime$ называется векторным процессом Ито, если правая часть~\ref{eq:3/1/2} существует при всех $t \in \setTime$. При этом процессы $\m{a}_t\idxB{i}$ называются коэффициентами снова, а процессы $\m{B}_t\idxB{i}$ называются коэффициентами диффузии процесса $\m{\eta}_t$ ($i = 1, 2, \ldots, n$, $j = 1, 2, \ldots, m$).
\end{df}

Соотношение~\ref{eq:3/1/2} обобщается и на случай векторного процесса Ито:

\beq{eq:3/1/4}
	\eta_s = \eta_t + \int\limits_t^s \m{a}_\tau\,d\tau + \int\limits_0^t \m{B}_\tau\,d\m{f}_\tau \text{ с вероятностью 1,}
\eeq

справедливое для всех $s, t \in \setTime$ и $s \geqslant t$.

\br

Процессы Ито тесно связаны со стохастическими дифференциальными уравнениями Ито. Действительно, пусть задано дифференциальное уравнение

\beq{eq:3/1/5}
	dx_t = a(x_t, t)dt + \sigma(x_t, t)df_t \text{; } x_0 = x(0, \omega) \text{,}
\eeq

где $x_t \in \Rv{1}$~--- случайный процесс, являющийся решением уравнения~\ref{eq:3/1/5}; $a(x, t)$ и $\sigma(x, t) \in Rv{1}$~--- измеримые при всех $(x, t) \in \Rv{1} \times \setTime$ функции. Тогда это уравнение может быть переписано в интегральном виде

\beq{eq:3/1/6}
	x_t = x_0 + \int\limits_0^t a(x_\tau, \tau)\,d\tau + \int\limits_0^t \sigma(x_\tau, \tau)\,df_\tau \text{; } x_0 = x(0, \omega) \text{.}
\eeq

Второй интеграл в правой части~\ref{eq:3/1/6} может пониматься в смысле Ито, Стратоновича или как $\theta$-интеграл\cite{KUZNETCOV}. Пусть интеграл $\int_0^t \sigma(x_\tau, \tau)\,df_\tau$ является стохастическим интегралом Ито, тогда уравнение~\ref{eq:3/1/5} и соответствующее ему уравнение~\ref{eq:3/1/6} называются стохастическими дифференциальными уравнениями Ито.



% **********************************************************************************************
\section{Оптимальное стохастическое управление}
% **********************************************************************************************



Пусть уравнение состояния динамической системы представляет собой систему стохастических дифференциальных уравнений Ито

\beq{eq:3/2/1}
	d\m{x}_t = \m{a}(\m{x}_t, \m{u}, t)dt + \m{\Sigma}(\m{x}_t, \m{u}, t)d\m{f}_t \text{,}
\eeq

где $\m{x}_t \in \Rv{n}$~--- случайный процесс, являющийся решением уравнения~\ref{eq:3/2/1}; $\m{a} \in \Rv{n}$, $\m{\Sigma} \in \Rv{n}$ представляют собой векторные функции сноса и диффузии соответственно; $\m{f}_t$~--- стандартный винеровский процесс, моделирующий белый шум; $\m{u} \in \Rv{k}$~--- управляющее воздействие, которое выбирается из условия минимизации функционала вида

\beq{eq:3/2/2}
	\funcF(\m{x}, \m{u}, s) = \E \Biggl\{ Q(\m{x}_\tau, \tau) + \int\limits_s^\tau F(\m{x}_t, \m{u}, t)\,dt~\Biggm|~ \m{x}_s = \m{x} \Biggr\} \text{,}
\eeq

где $Q$, $F$~--- известные функции, а $\tau$, например, постоянный момент времени.

Известны различные типы управляющих воздействий, которые могут использовать или не использовать информацию о векторе состояния системы в предшествующие моменты времени (некоторые из них были рассмотрены в предыдущем разделе). Рассмотрим управление $\m{u}(\m{x}_t, t)$ марковского типа. Тогда для него минимальная по $u$ величина функционала $\funcF$, которую обозначим через $\funcH(\m{x}, s)$, удовлетворяет уравнению Гамильтона-Якоби-Беллмана вида

\beqarr
	\lbl{eq:3/2/3}
		\min\limits_{\m{u} \in \Rv{k}} \bigl\{ F(\m{x}, \m{u}, s) + L_u \funcH(\m{x}, s) \bigr\} = 0 \text{;} \\
	\lbl{eq:3/2/4}
		\funcH(\m{x}, T) = Q(\m{x}, T) \text{,}
\eeqarr

где предполагается, что $\tau = T$, а оператор $L_u\argEmpty$ имеет вид

\beq{eq:3/2/5}
\begin{split}
	L_u\argEmpty = \partdiff{\argEmpty}{s} &+ \sum\limits_{i=1}^n\m{a}\idxB{i}(\m{x}, \m{u}, s)\partdiff{\argEmpty}{\m{x}\idxB{i}} + \\
	&+ \frac{1}{2} \sum\limits_{i=1}^n \sum\limits_{j=1}^n \m{D}\idxB{ij}(\m{x}, \m{u}, s)\partdiffsecond{\argEmpty}{\m{x}\idxB{i}}{\m{x}\idxB{j}} \text{,}
\end{split}
\eeq

где $\m{D} = \m{\Sigma}\m{\Sigma}^T$.

Отметим, что приведенные выше условия являются необходимыми условиями, которым удовлетворяет минимум функционала, если он существует.

\br

Рассмотрим линейно-квадратичную проблему стохастического управления, то есть проблему нахождения управления $\m{u}(\m{x}_t, t)$ марковского типа, которое доставляет минимум следующему функционалу:

\beq{eq:3/2/6}
	\funcF(\m{x}, \m{u}, s) = \E \Biggl\{ \m{x}_T\m{R}\m{x}_T^T + \int\limits_s^T \bigr( \m{x}_t^T\m{C}(t)\m{x}_t + \m{u}^T\m{P}(t)\m{u} \bigl)\,dt~\Biggm|~ \m{x}_s = \m{x} \Biggr\} \text{,}
\eeq

где $\m{C}(t), \m{R} \in \Rs{n}{n}$, $\m{P}(t) \in \Rs{k}{k}$, а $\m{x}_t \in \Rv{n}$~--- решение следующей системы стохастических дифференциальных уравнений.

\beq{eq:3/2/7}
	d\m{x}_t = \bigl( \m{A}(t)\m{x}_t + \m{B}(t)\m{u}(\m{x}_t, t) \bigr)dt + \m{\Sigma}(t)d\m{w}_t \text{,}
\eeq

где $\m{A}(t) \in \Rs{n}{n}$, $\m{B}(t) \in \Rs{n}{k}$, $\m{\Sigma} \in \Rs{n}{m}$, а $\m{w}_t \in \Rv{m}$~--- стандартный винеровский процесс с независимыми компонентами $\m{w}\idxB{i}_t$ ($i = 1, 2, \ldots, m$). Кроме этого, предполагается, что все зависящие от $t$ матрицы непрерывны. $\m{C}(t)$ и $\m{R}$~--- симметричные неотрицательно определенные матрицы, а матрица $\m{P}(t)$~--- симметричная положительно определенная.

Запишем уравнение Гамильтона-Якоби-Беллмана для данной задачи:

\beq{eq:3/2/8}
\begin{split}
	\partdiff{\funcH}{s} + \m{x}^T\m{C}(s)\m{x} &+ \sum\limits_{j=1}^n \bigl( \m{A}(s)\m{x} \bigr)\idxB{j}\partdiff{\funcH}{\m{x}\idxB{j}} + \frac{1}{2} \partdiffsecond{\funcH}{\m{x}\idxB{i}}{\m{x}\idxB{j}} + \\
	&+ \min\limits_{\m{u} \in \Rv{k}} \Bigl\{ \m{u}^T\m{P}(t)\m{u} + \sum\limits_{j=1}^n \bigl( \m{B}(s)\m{u} \bigr)\idxB{j}\partdiff{\funcH}{\m{x}\idxB{j}} \Bigr\} = 0 \text{,}
\end{split}
\eeq

где $\m{D} = \m{\Sigma}(t)\m{\Sigma}^T(t)$, а граничное условие имеет вид:

\beq{eq:3/2/9}
	\funcH(T, \m{x}) = \m{x}^T\m{R}\m{x} \text{.}
\eeq

В\cite{KLOEDEN}, в частности, показано, что оптимальное управление и минимальное значение функционала $\funcF(\m{x}, \m{u}, s)$ имеют в этой задаче следующий вид:

\beqarr
	\lbl{eq:3/2/10}
		\m{u}(\m{x}, s) = \m{P}^{-1}(s)\m{B}(s)\m{x} \text{;} \\
	\lbl{eq:3/2/11}
		\funcH(\m{x}, s) = \m{x}^T\m{S}\m{x} + \int\limits_s^T \tr \bigl\{ \m{\Sigma}(t)\m{\Sigma}^T(t)\m{S}(t) \bigr\}\,dt \text{,}
\eeqarr

где $\tr$~--- след\footnote{ Сумма диагональных элементов. } матрицы; $\m{S}(s) \in \Rs{n}{n}$~--- симметричная положительно определенная и непрерывно дифференцируемая матрица, которая удовлетворяет следующему матричному уравнению Риккати:

\beq{eq:3/2/12}
	\genericdiff{S(s)}{s} = -\m{A}(S)^T\m{S}(s) - \m{S}(s)\m{A}(s) + \m{S}(s)\m{B}(s)\m{P}^{-1}(s)\m{B}(s)\m{S}(s) - \m{C}(s)
\eeq

с начальным условием

\beq{eq:3/2/13}
	\m{S}(T) = \m{R} \text{.}
\eeq

Таким образом, если удается решить (хотя бы численно) уравнение Риккати~\ref{eq:3/2/12}, то по формулам~\ref{eq:3/2/10} и~\ref{eq:3/2/11} можно найти оптимальное управление и минимальное значение функционала.

\br

Предположим, что измерение вектора $\m{x}_t$ невозможно, но допустимо измерять наблюдаемый процесс $\m{y}_t \in \Rv{k}$ ($k = 1, 2, \ldots, n$), который удовлетворяет следующей системе дифференциальных уравнений Ито:

\beq{eq:3/2/14}
	d\m{y}_t = \m{H}\m{x}_tdt + \Xi d\m{f}_t
\eeq

с начальным условием $\m{y}_0 = \m{y}(0) = 0$.

В системе~\ref{eq:3/2/14} $\m{H} \in \Rs{k}{n}$, $\Xi \in \Rs{k}{l}$~--- числовые матрицы; $\m{f}_t \in \Rv{l}$~--- стандартный винеровский процесс с независимыми компонентами $\m{f}\idxB{i}_t$ ($i = 1, 2, \ldots, l$), который не зависит от процесса $\m{w}_t$. В этом случае, с помощью фильтра Калмана-Бьюси можно посмтроить оценку $\hat{\m{x}}_t$ и использовать ее в оптимальном управлении $\m{u}(\hat{\m{x}}_t, t)$ для линейно-квадратичной задачи.

Такие исследования остаются за пределами рассмотрения этой работы, подробно они излагаются в соответствующей литературе\cite{KLOEDEN}.

\br

Отметим, что полученные результаты практически полностью, вплоть до переобозначений, совпадают с результатами~\ref{eq:2/2/7},~\ref{eq:2/2/8} и~\vref{eq:2/2/9}, что позволяет говорить о \emph{принципе полной эквивалентности в стохастическом управлении}. Иными словами, построение регулятора, синтезирующего оптимальное управление с обратной связью в случае процессов Ито ничем не отличается от классической модели.



% **********************************************************************************************
\section{Стохастическая устойчивость}
% **********************************************************************************************



Рассмотрим стохастическое дифференциальное уравнение Ито вида~\vref{eq:3/2/1}. Будем считать, что $\m{x}_t \equiv \m{0}$\footnote{ Здесь и далее полагаем <<$\m{0}$>> вектором соответствующей размерности, состоящим из нулей. } является невозмущенным движением динамической системы, а возмущенное движение описывается~\ref{eq:3/2/1}. В соответствии с этим положим

\beq{eq:3/3/1}
    \eqsystem{
		\m{a}(\m{0}, t) \equiv 0 \text{,} \\
		\m{\Sigma}(\m{0}, t) \equiv 0    
    }
\eeq

при $t \geqslant 0$.

\br

Приведем некоторые определения основных типов устойчивости для стохастических систем.

\begin{df}\lbl{df:10}
	Невозмущенное движение системы $\m{x}_t = \m{0}$ будем называть устойчивым по вероятности или стохастически устойчивым при $t \geqslant t_0$, если для любых $\eps > 0$ и $t_0 \geqslant 0$

	\beqn
		\lim\limits_{\m{x}_0 \to \m{0}} \prob \biggl( \sup\limits_{t \geqslant t_0} \bigl| \m{x}^{\m{x}_0, t_0}_t \bigr| \geqslant \eps \biggr) = 0 \text{,}
	\eeqn

	где $\m{x}^{\m{x}_0, t_0}_t$~--- такое решение уравнения~\ref{eq:3/2/1}, что $\m{x}_{t_0} = \m{x}_0$.
\end{df}

\begin{df}\lbl{df:11}
	Невозмущенное движение системы $\m{x}_t \equiv \m{0}$ называется асимптотически устойчивым по вероятности или асимптотически стохастически устойчивым, если оно стохастически устойчиво и, кроме того,
	
	\beqn
		\lim\limits_{\m{x}_0 \to \m{0}} \prob \biggl( \lim\limits_{t \to \infty} \bigl| \m{x}^{\m{x}_0, t_0}_t \bigr| = 0 \biggr) = 1 \text{.}
	\eeqn
\end{df}

Приведенные определения являются обобщением классических определений устойчивости в малом по Ляпунову на случай стохастических систем.

\begin{df}\lbl{df:12}
	Невозмущенное движение системы $\m{x}_t \equiv \m{0}$ называется асимптотически стохастически устойчивым в целом, если оно стохастически устойчио и, кроме того, для всех $|\m{x}_0| < \infty$
	
	\beqn
		\prob \biggl( \lim\limits_{t \to \infty} \bigl| \m{x}^{\m{x}_0, t_0}_t \bigr| = 0 \biggr) = 1 \text{.}
	\eeqn
\end{df}

Приведем также некоторые определения устойчивости, которые связаны с моментными характеристиками процесса $\m{x}_t$.

\begin{df}\lbl{df:13}
	Невозмущенное движение системы $\m{x}_t \equiv \m{0}$ называется устойчивым в среднем степени $p$ ($p > 0$) или $p$-устойчивым, если для любых $\eps > 0$ и $t_0 \geqslant 0$ найдется $\delta = \delta(t_0, \eps) > 0$ такое, что
	
	\beqn
		\E \Bigl\{ \bigl| \m{x}^{\m{x}_0, t_0}_t \bigr|^p \Bigr\} < \eps \text{ для всех } t \geqslant t_0 \text{ и } |\m{x}_0| < \delta \text{.}
	\eeqn
\end{df}

\begin{df}\lbl{df:14}
	Невозмущенное движение системы $\m{x}_t \equiv \m{0}$ называется асимптотически устойчивым в среднем степени $p$ ($p > 0$) или асимптотически $p$-устойчивым, если оно $p$-устойчиво и существует $\delta_0 = \delta_0(t_0) > 0$, такое, что
	
	\beqn
		\lim\limits_{t \to \infty} \E \Bigl\{ \bigl| \m{x}^{\m{x}_0, t_0}_t \bigr|^p \Bigr\} = 0 \text{ для всех } |\m{x}_0| < \delta_0 \text{.}
	\eeqn
\end{df}

\begin{df}\lbl{df:15}
	Невозмущенное движение системы $\m{x}_t \equiv \m{0}$ называется экспоненциально $p$-устойчивым, если существуют постоянные $\gamma > 0$ и $\alpha > 0$, такие, что
	
	\beqn
		\E \Bigl\{ \bigl| \m{x}^{\m{x}_0, t_0}_t \bigr|^p \Bigr\} < \gamma|\m{x}_0|^pe^{-\alpha(t-t_0)} \text{ для всех } t \geqslant t_0 \text{.}
	\eeqn
\end{df}

Следует заметить, что существуют и другие определения стохастической устойчивости\cite{KUSHNER}.

\br

Как известно, одним из методов исследования устойчивости детерменированых систем является метод функций Ляпунова. Достоинство этого метода заключается в том, что он позволяет (в том случае, если соответствующая функция Ляпунова найдена) делать заключения о характере поведения решения дифференциального уравнения, не требуя при этом знания точного решения этого дифференциального уравнения.

Рассмотрим суть метода стохастических функций Ляпунова. Рассмотрим функцию $\funcV(\m{x}, t): \Rv{n} \times \{t > 0\} \to \Rv{1}$, которая при всех $(\m{x}, t) \in \setU \times \{t > 0\}$ ($\setU \subset \Rv{n}$, $\m{0} \in \setU$) является непрерывной и положительно определенной. Пусть также $\funcV(\m{x}, t)$ дважды непрерывно дифференцируема по $\m{x}$ и один раз по $t$, за исключением может быть множества $\{ \m{x} = \m{0} \}$. Кроме этого, предположим, что $\funcV(\m{0}, t) = 0$.

Далее через $\setC_2^0(\setU)$ будем обозначать множество функций дважды непрерывно дифференцируемых по $\m{x} \in \setU$ и один раз по $t$, за исключением множества $\{ \m{x} = \m{0} \}$. В силу формулы Ито имеем

\beq{eq:3/3/2}
\begin{split}
	\funcV(\m{x}_t, t) - \funcV(\m{x}_{t_0}, t_0) &= \int\limits_{t_0}^t L\bigl\{ \funcV(\m{x}_\tau, \tau) \bigr\}\,d\tau + \\
	&+ \sum\limits_{i=1}^m \int\limits_{t_0}^t \m{G}\idxB{i}_0\bigl\{ \funcV(\m{x}_\tau, \tau) \bigr\}\,d\m{f}\idxB{i}_\tau \text{ с вероятностью 1,}
\end{split}
\eeq

где

\beq{eq:3/3/3}
\begin{split}
	L\argEmpty = \partdiff{\argEmpty}{t} &+ \sum\limits_{i=1}^n \m{a}\idxB{i}(\m{x}, t)\partdiff{\argEmpty}{\m{x}\idxB{i}} + \\
	&+ \frac{1}{2}\sum_{j=1}^m\underset{l, i = 1}{\overset{n}{\sum\sum}} \m{\Sigma}\idxB{lj}(\m{x}, t) \m{\Sigma}\idxB{ij}(\m{x}, t) \partdiffsecond{\argEmpty}{\m{x}\idxB{l}}{\m{x}\idxB{i}}
\end{split}
\eeq

и при $i = 1, 2, \ldots, m$

\beq{eq:3/3/4}
	G\idxB{i}\argEmpty = \sum_{j=1}^n \m{\Sigma}\idxB{ji}(\m{x}, t) \partdiff{\argEmpty}{\m{x}\idxB{j}} \text{.}
\eeq

Как можно видеть, в стохастическом случае роль первой производной $\funcV(\m{x}, t)$ по $\m{x}$ в силу уравнения движения играет $L\bigl\{ \funcV(\m{x}_s, s) \bigr\}$. Будем считать, что при $\m{x} \neq \m{0}$ и $t \geqslant 0$ выполняется неравенство

\beq{eq:3/3/5}
	L\bigl\{ \funcV(\m{x}_s, s) \bigr\} \leqslant 0 \text{.}
\eeq

Функцию $\funcV(\m{x}, t)$, обладающую перечисленными выше свойствами, будем называть \emph{стохастической функцией Ляпунова}.

На стохастическую функцию Ляпунова могут также накладываться и некоторые дополнительные ограничения. Далее, используя~\ref{eq:3/3/2} и~\ref{eq:3/3/5} и свойства интеграла Ито, получаем

\beq{eq:3/3/6}
	\E \bigl\{ \funcV(\m{x}_t, t) \bigm| \funcF_{t_0} \bigr\} \leqslant \funcV(\m{x}_{t_0}, t_0) \text{ с вероятностью 1.}
\eeq

Таким образом, $\funcV(\m{x}_t, t)$~--- супермартингал. Если $\m{y}_s$ ($s \geqslant 0$)~--- мартингал и $\E \bigl\{ |\m{y}_s|^p \bigr\} < \infty$, то справедливо хорошо известное неравенство:

\beq{eq:3/3/7}
	\forall \alpha:~\prob\Biggl(\bigl\{ \omega \in \Omega : \sup\limits_{0 \leqslant s \leqslant t} |\m{y}(s, \omega)| \geqslant \alpha \bigr\}\Biggr) \leqslant \frac{1}{\alpha^p} \E \bigl\{ |\m{y}_t|^p \bigr\}
\eeq

и, следовательно, для всех $T > t_0$ и $\delta > 0$

\beq{eq:3/3/8}
	\prob\Biggl( \sup\limits_{t_0 \leqslant t \leqslant T} \funcV(\m{x}_t, t) \geqslant \delta \Biggr) \leqslant \frac{1}{\delta} \funcV(\m{x}_{t_0}, t_0) \text{.}
\eeq

Из последнего неравенства, непрерывности $\funcV(\m{x}, t)$ и того, что $\funcV(\m{0}, t) = 0$, получаем:

\beq{eq:3/3/9}
	\lim\limits_{\m{x}_0 \to \m{0}} \prob\Biggl( \sup\limits_{t_0 \leqslant t \leqslant T} \funcV(\m{x}_t, t) \geqslant \delta \Biggr) = 0 \text{.}
\eeq

Далее, если предположить, что при некоторых постоянных $p$, $k_1$ и $k_2$

\beq{eq:3/3/10}
	k_1 |\m{x}|^p \leqslant \funcV(\m{x}, t) \leqslant k_2 |\m{x}|^p \text{,}
\eeq

то согласно~\ref{eq:3/3/9} получим, что невозмущенное движение системы $\m{x}_t~\equiv~\m{0}$ стохастически устойчиво.

\br

Приведем еще ряд фактов\cite{HASMINSKI}, касающихся стохастической устойчивости

\bteo{teo:5}
	Пусть в области $\{ t > 0 \} \times \setU$, включающей множество $\{ \m{x} \equiv \m{0} \}$, существует непрерывная, положительно определенная функция $\funcV(\m{x}, t) \in \setC_2^0(\setU)$ и пусть при $t > 0$
	
		\beqn
			\limsup\limits_{\m{x} \to \m{0}} \funcV(\m{x}, t) = 0 \text{,}
		\eeqn
		
		а $L\bigl\{ \funcV(\m{x}, t) \bigr\} < 0$ в этой области. Тогда невозмущенное движение $\m{x}_t \equiv \m{0}$ асимптотически стохастически устойчиво.
\eteo

\bteo{teo:6}
	Пусть для всех $\m{x} \in \Rv{n}$ существует положительно определенная функция $\funcV(\m{x}, t) \in \setC_2^0(\Rv{n})$, для которой при $t > 0$
	
	\beqn
		\eqsystem{
			\limsup\limits_{\m{x} \to \m{0}} \funcV(\m{x}, t) = 0 \text{,} \\
			\liminf\limits_{\m{x} \to \m{\infty}} \funcV(\m{x}, t) = \infty \text{,} \\
			L\bigl\{ \funcV(\m{x}, t) \bigr\} < 0 \text{.}
		}
	\eeqn
	
	Тогда невозмущенное движение $\m{x}_t \equiv \m{0}$ асимптотически стохастически устойчиво в целом. 
\eteo

\bteo{teo:7}
	Для экспоненциальной $p$-устойчивости невозмущенного движения $\m{x}_t \equiv \m{0}$ при $t > 0$ достаточно, чтобы при всех $\m{x} \in \Rv{n}$ существовала функция $\funcV(\m{x}, t) \in \setC_2^0(\Rv{n})$, удовлетворяющая при некоторых положительных постоянных $k_1$, $k_2$ и  $k_3$ неравенствам
	
	\beqn
		\eqsystem{
			k_1 |\m{x}|^p \leqslant \funcV(\m{x}, t) \leqslant k_2 |\m{x}|^p \text{,} \\
			L\bigl\{ \funcV(\m{x}, t) \bigr\} < -k_3 |\m{x}|^p \text{.}
		}
	\eeqn
\eteo

Другой метод детерминистской теории устойчивости заключается в линеаризации дифференциального уравнения в окрестности невозмущенного движения и анализе устойчивости нулевого решения линеаризованного дифференциального уравнения. Известно\cite{ARNOLD}, что в ряде случаев устойчивость нулевого решения линеаризованной системы влечет устойчивость невозмущенного решения соответствующей нелинейной системы. Особую роль при анализе устойчивости с помощью этого подхода играют так называемые ляпуновские экспоненты или ляпуновские показатели. Оказывается, что метод линеаризации может быть обобщен и на случай стохастических систем\cite{AUSLENDER}.
