% coding: utf-8
% --------------------------------------------------------------------------------------------------
% "Синтез оптимального стохастического управления", 2011 год
% --------------------------------------------------------------------------------------------------



\chapter*{Введение}
\addcontentsline{toc}{chapter}{Введение}



Процессы управления происходят повсюду: и в живой, и в неживой природе: в биологических организмах, обществе, технике. Даже естественный отбор, являющийся основой учения Дарвина, благодаря которому одни особи исчезают, а другие выживают и воспроизводятся, также является своего рода процессом управления, протекающим в природе.

Общим для всех процессов управления, где бы они не протекали, является прием, хранение, преобразование информации и выработка, синтез, управления на ее основе. Осознание этой общности послужило предпосылкой к возникновению в конце сороковых годов XX века научного направления, названного Н.\,Винером \emph{кибернетикой}. Хотя управление человеческими коллективами с одной стороны, и техническими объектами~--- с другой, имеет много общего, коренные различия, которые существуют между этими <<объектами>>, делают необходимыми их раздельное рассмотрение.

В технике \emph{управлением} называют целенаправленное воздействие на какое-либо устройство или объект. Если управление осуществляет человек, то управление называют \emph{ручным}; \emph{автоматическое} управление осуществляется без непосредственного участия человека. Устройство (машина, агрегат, технологический процесс), состоянием которого нужно управлять, называется \emph{объектом управления} или \emph{управляемым объектом}. Целью управления таким объектом является поддержание заданного режима работы или вывод объекта на такой режим работы. Под \emph{заданным режимом} понимают изменение какого-либо параметра, характеризующего состояние объекта управления, по определенному закону.

Объект управления с взаимодействующим с ним управляющим устройством называют \emph{системой управления}. В простейших случаях систему автоматического управления называют \emph{системой автоматического регулирования} (САР), управляющее устройство~--- \emph{регулятором}, а объект управления~--- \emph{объектом регулирования} или \emph{регулируемым объектом}.

\newpage

Выделяют три основных принципа управления:

\bdescr
    \item[Принцип программного управления]
        Если об объекте точно известно, как зависит выходная переменная от управляющего воздействия, управление можно формировать как известную функцию времени. Такой принцип неприменим при управлении объектом, на который действуют заранее не известные возмущения, оказывающие существенное влияние на управляемую величину. Он также неприменим, если объект является нейтральным или неустойчивым, а система должна функционировать длительное время.

    \item[Принцип компенсации]
        При таком принципе управления определяются каким-либо образом действующие на систему управления возмущения и на их основе вырабатывается управляющее воздействие, которое полностью или частично компенсирует влияние возмущений на процесс управления. Также этот принцип называют \emph{способ управления по возмущению} или \emph{принципом Понселе}. Достоинством такого способа управления является принципиальная возмжность полной компенсации возмущающего воздействия.

    \item[Принцип обратной связи]
        Этот принцип также называется \emph{управлением по отклонению}. Это такой способ управления, при котором определяется отклонение текущего значения выходной переменной от требуемого значения, и на его основе формируется управляющее воздействие. Системы управления, основанные на этом способе, непременно содержат \emph{обратную связь}~--- канал связи, по которому информация об управляемой переменной поступает на управляющее устройство. Иногда этот принцип называют \emph{принципом Ползунова--Уатта}.

        Недостатком такого способа управления является принципиальная невозможность полной компенсации возмущающих воздействий. Это связано с тем, что при этом способе управляющее воздействие начинает вырабатываться и оказывать влияние на ход процесса только после того, как возмущение, начав действовать, вызывает отклонение управляемой величины от требуемого режима.

    \item[Принцип комбинированного управления]
        Этот принцип является попыткой совместить достоинства управлений по возмущению и по отклонению. Этот принцип используется в тех случаях, когда на систему действует много различных возмущений, но лишь немногие оказывают наибольшее влияние на работу системы уравления. В подобных случаях влияние превалирующего возмущения можно нейтрализовать, используя принцип компенсации, и нейтрализовать влияние остальных возмущений, используя принцип обратной связи.
\edescr

Задачей работы является изучение теоретической возможности управления объектами, используя принцип комбинированного управления, при учете множества различных случайных возмущений. Это могут быть как шум среды, так, например, механические воздействия друг на друга сразу нескольких объектов, принимающих участие в сложном движении. В результате работы выработаны некоторые алгоритмы, которые могут быть использованы при конструировании требуемого регулятора, отмечены как положительные стороны алгоритмов, так и отрицательные. Каждый алгоритм имеет под собой математическую основу, гарантирующую, что сконструированный таким образом регулятор будет обеспечивать требуемое в определенном смысле управление. Требуемое управление, иначе называемое \emph{оптимальным} говорит о том, что его использование позволит получить требуемый результат наилучшим среди прочих образом.

В процессе исследований был разработан некоторый математический пакет, позволяющий программно сконструировать подобный регулятор и проводить различные экспериментальные проверки с целью выявляения особенностей работы того или иного алгоритма. Пакет распространяется свободно, на принципах открытого сообщества.

\br

Работа устроена следующим образом:

\bdescr
    \item[Постановка задачи]
        Раздел посвящен математической формализации задачи синтеза оптимального регулятора. Дается как математическое описание системы (в случаях случайных воздействий и их отсутствия), так и принцип, по которому определяется оптимальность управления.

    \item[Синтез оптимального управления]
        Раздел посвящен методам синтеза оптимального управления в различных случаях без случайных воздействий. Дается математическое обоснование полученных результатов, как в общем случае, так и в частном, наиболее распространенном случае с квадратичным критерием качества. Результаты связываются с принципом Беллмана и принципом максимума.

    \item[Методы стохастического управления]
        Раздел посвящен обобщению результатов, полученных в разделе <<\emph{Синтез оптимального управления}>> на случаях стохастических систем, то есть систем, где учитываются случайные воздействия. Дополнительно раскрывается вопрос стохастической устойчивости полученных теоретических результатов.

    \item[Пример решения типовой задачи]
        Раздел посвящен решению конкретной типовой задачи, в которой используются результаты из раздела <<\emph{Методы стохастического управления}>>. Дается описание задачи (полет баллистической ракеты в турбулентной атмосфере), конструируются алгоритмы ее решения, дается описание созданного для ее исследования программного математического пакета, анализируются полученные результаты, дается их обоснование.
\edescr
