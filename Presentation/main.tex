% coding: utf-8
% --------------------------------------------------------------------------------------------------
% "Синтез оптимального стохастического управления", 2011 год
% --------------------------------------------------------------------------------------------------



\documentclass[ignorenonframetext,hyperref={pdftex,unicode},compress]{beamer}
\mode<presentation>{}

\usepackage{cmap}
\usepackage[utf8]{inputenc}
\usepackage[russian]{babel}
\usepackage[T2A]{fontenc}
\usepackage{amssymb}
\usepackage{amsfonts}
\usepackage{amsmath}
\usepackage{mathtext}
\usepackage[mathscr]{eucal}
\usepackage[russian]{varioref}
\usepackage{listings}
\usepackage[font=it,labelsep=period,format=hang]{caption}



% for screen only
\usetheme{Singapore}

% for handout only
%\usetheme{Pittsburgh}
%\usecolortheme{dove}


\newcommand{\br}{\vspace{12pt}}
\newtheorem{teo}{Теорема}
\newtheorem{alg}{Алгоритм}



%-------------------------------------------------------------------------------



\title{О проблемах синтеза оптимального регулятора с обратной связью в многомерных стохастических системах}
\author{Архипов С.\,В.}
\institute{Научный руководитель: Пакшин П.\,В.}
\date{15 июня 2011 г.}



\begin{document}


% начало рисунка
\newcommand{\bfig}{ \begin{figure}\center }

% конец рисунка
\newcommand{\efig}{ \end{figure} }

% вставка рисунка
\newcommand{\addfigure}[3]{ \bfig \input{fig_#2} \caption{#3} \lbl{#1} \efig }



%-------------------------------------------------------------------------------
%---   слайд 0: титульная страница
%-------------------------------------------------------------------------------
\begin{frame}
    \titlepage
\end{frame}



%-------------------------------------------------------------------------------
%---   слайд 1: постановка основной задачи
%-------------------------------------------------------------------------------
\begin{frame}
    \frametitle{Гибридные диффузионные системы}
    \framesubtitle{Описание проблематики}

    \begin{description}
        \item[Составные системы]~--- системы, чей механизм работы качественно и существенно зависит от некоторого множества параметров.
    \end{description}
    \begin{itemize}
        \item Сложность в описании составных систем;
        \item Сложность при решении таких систем;
        \item Сложность учета новых условий
    \end{itemize}
    \par Необходимо введение нового подхода к описанию составных систем.
\end{frame}


\end{document}

