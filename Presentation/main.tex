% coding: utf-8
% --------------------------------------------------------------------------------------------------
% "Синтез оптимального стохастического управления", 2011 год
% --------------------------------------------------------------------------------------------------



\documentclass[ignorenonframetext,hyperref={pdftex,unicode},compress]{beamer}
\mode<presentation>{}

\usepackage{cmap}
\usepackage[utf8]{inputenc}
\usepackage[russian]{babel}
\usepackage[T2A]{fontenc}
\usepackage{amssymb}
\usepackage{amsfonts}
\usepackage{amsmath}
\usepackage{mathtext}
\usepackage[mathscr]{eucal}
\usepackage[russian]{varioref}
\usepackage{listings}
\usepackage[font=it,labelsep=period,format=hang]{caption}



% for screen only
\usetheme{Singapore}

% for handout only
%\usetheme{Pittsburgh}
%\usecolortheme{dove}


\newcommand{\br}{\vspace{12pt}}
\newtheorem{teo}{Теорема}
\newtheorem{alg}{Алгоритм}



%-------------------------------------------------------------------------------



\title{О проблемах синтеза оптимального регулятора с обратной связью в многомерных стохастических системах}
\author{Архипов С.\,В.}
\institute{Научный руководитель: Пакшин П.\,В.}
\date{15 июня 2011 г.}



\begin{document}



%-------------------------------------------------------------------------------
%---   слайд 0: титульная страница
%-------------------------------------------------------------------------------
\begin{frame}
    \titlepage
\end{frame}



%-------------------------------------------------------------------------------
%---   слайд 1: постановка основной задачи
%-------------------------------------------------------------------------------
\begin{frame}
	\frametitle{Постановка задачи}

    Пусть физический объект описывается системой~\ref{eq:1}

	\begin{equation}\label{eq:1}
	    \dot{\mathbf{x}}(t) = \mathbf{A}(t)\mathbf{x}(t) + \mathbf{B}(t)\mathbf{u}(t) \text{.}
	\end{equation}

	Требуется найти такое оптимальное управление $\mathbf{u}^*(t)$, при котором значение критерия качества было бы минимальным:
	
	\begin{equation}\label{eq:2}
	    \mathcal{F} \bigl(  \mathbf{x}(t), \mathbf{u}^*(t)  \bigr) = \min_{\mathbf{u}(t) \in \Upsilon} \mathcal{F} \bigl(  \mathbf{x}(t), \mathbf{u}(t)  \bigr)  \text{.}
	\end{equation}
\end{frame}



%-------------------------------------------------------------------------------
%---   слайд 2: Синтез оптимального управления -- уравнение Беллмана
%-------------------------------------------------------------------------------
\begin{frame}
	\frametitle{Синтез оптимального управления}
	\framesubtitle{Уравнение Беллмана}

    Пусть критерий качества задан в виде
    
    \begin{equation}\label{eq:3}
    	\mathcal{F} = \int\limits_{t_1}^{t_2} \mathsf{L}(\mathbf{x}, \mathbf{u}, t)\,dt \text{.}
    \end{equation}
    
    Тогда необходимое условие оптимальности запишется как
    
    \begin{equation}\label{eq:4}
    	\underset{\mathbf{u}(t) \in \Upsilon}{\min} \biggl\{ \biggl[ \frac{d \mathcal{F}^*}{d t} \biggr]_{\mathbf{u}, t} + \mathsf{L}\bigl(  \mathbf{x}(\mathbf{u}, t), \mathbf{u}, t  \bigr) \biggr\} = 0 \text{.}
    \end{equation}
\end{frame}



%-------------------------------------------------------------------------------
%---   слайд 3: Синтез оптимального управления -- уравнение Беллмана, квадратич-
%---            ный критерий качества
%-------------------------------------------------------------------------------
\begin{frame}
	\frametitle{Синтез оптимального управления}
	\framesubtitle{Уравнение Беллмана с квадратичным критерием качества}\small

    Пусть критерий качества задан в виде
    
    \begin{equation}\label{eq:5}
    	\mathcal{F} = \frac{1}{2} \mathbf{x}_2^T \mathbf{M} \mathbf{x}_2 + \frac{1}{2} \int\limits_{t_1}^{t_2} \bigl[ \mathbf{x}^T(\tau)\mathbf{Q}(\tau)\mathbf{x}(\tau) + \mathbf{u}^T(\tau)\mathbf{R}(\tau)\mathbf{u}(\tau) \bigr]\,d\tau \text{.}
    \end{equation}
    
    Тогда оптимальное управление $\mathbf{u}^*(t)$ может быть синтезировано следующим образом:
    
    \begin{equation}\label{eq:6}
    	\mathbf{u}^*(\mathbf{x}, t) = -\mathbf{R}^{-1}(t)\mathbf{B}^T(t)\mathbf{P}(t)\mathbf{x}(t) \text{,}
    \end{equation}
    
    где матрица $\mathbf{P}(t)$ может быть найдена из уравнения Риккати
    
    \begin{equation}\label{eq:7}
    	\begin{cases}
    		\begin{split}
    			- \frac{d \mathbf{P}(t)}{dt} =& -\mathbf{P}(t)\mathbf{B}(t)\mathbf{R}^{-1}(t)\mathbf{B}^T(t)\mathbf{P}(t) + \mathbf{P}(t)\mathbf{A}(t) + \\
    			&+ \mathbf{A}^T(t)\mathbf{P}(t) + \mathbf{Q}(t) \text{;}
			\end{split} \\
			\mathbf{P}(t_2) = \mathbf{M} \text{.}
		\end{cases}    
    \end{equation}
\end{frame}



%-------------------------------------------------------------------------------
%---   слайд 4: Синтез оптимального управления -- уравнение Беллмана, стационар-
%---            ный случай
%-------------------------------------------------------------------------------
\begin{frame}
	\frametitle{Синтез оптимального управления}
	\framesubtitle{Уравнение Беллмана при $T \to \infty$}

    Пусть критерий качества задан в виде
    
    \begin{equation}\label{eq:8}
    	\mathcal{F} = \frac{1}{2} \int\limits_0^\infty \bigl[ \mathbf{x}^T(\tau)\mathbf{Q} \mathbf{x}(\tau) + \mathbf{u}^T(t)\mathbf{R} \mathbf{u}(\tau) \bigr]\,d\tau \text{.}
    \end{equation}
    
    Тогда оптимальное управление $\mathbf{u}^*(t)$ может быть синтезировано следующим образом:
    
    \begin{equation}\label{eq:9}
    	\mathbf{u}^*(\mathbf{x}, t) = -\mathbf{R}^{-1}\mathbf{B}^T\mathbf{P}\mathbf{x}(t) \text{,}
    \end{equation}
    
    где матрица $\mathbf{P}$ может быть найдена из нелинейного алгебраического матричного уравнения Риккати
    
    \begin{equation}\label{eq:10}
    	-\mathbf{P}\mathbf{B}\mathbf{R}^{-1}\mathbf{B}^T\mathbf{P} + \mathbf{P}\mathbf{A} + \mathbf{A}^T\mathbf{P} + \mathbf{Q} = 0 \text{.}    
    \end{equation}
\end{frame}



%-------------------------------------------------------------------------------
%---   слайд 5: Оптимальное стохастическое управление - постановка задачи
%-------------------------------------------------------------------------------
\begin{frame}
	\frametitle{Оптимальное стохастическое управление}
	\framesubtitle{Постановка задачи}
    
    Пусть система задана в виде процесса Ито~\ref{eq:11}

    \begin{equation}\label{eq:11}
    	d\mathbf{x}_t = \bigl( \mathbf{A}(t)\mathbf{x}_t + \mathbf{B}(t)\mathbf{u}(\mathbf{x}_t, t) \bigr)dt + \mathbf{C}(t)d\mathbf{w}_t \text{.}
    \end{equation}
    
    Тогда оптимальное управление $\mathbf{u}^*(t)$ синтезируется из условия минимума функционала качества:
    
    \begin{equation}\label{eq:12}
    	\mathcal{F}(\mathbf{x}, \mathbf{u}, s) = \mathbb{E} \biggl\{ Q(\mathbf{x}_\tau, \tau) + \int\limits_s^\tau F(\mathbf{x}_t, \mathbf{u}, t)\,dt~\biggm|~ \mathbf{x}_s = \mathbf{x} \biggr\} \text{.}
    \end{equation}
\end{frame}



%-------------------------------------------------------------------------------
%---   слайд 6: Оптимальное стохастическое управление - квадратичный критерий
%---            качества
%-------------------------------------------------------------------------------
\begin{frame}
	\frametitle{Оптимальное стохастическое управление}
	\framesubtitle{Синтез оптимального управления}\small
    
    Пусть критерий качества является квадратичным

    \begin{equation}\label{eq:13}
    	\mathcal{F}(\mathbf{x}, \mathbf{u}, s) = \mathbb{E} \biggl\{ \mathbf{x}_T\mathbf{R}\mathbf{x}_T^T + \int\limits_s^T \bigr( \mathbf{x}_t^T\mathbf{C}(t)\mathbf{x}_t + \mathbf{u}^T\mathbf{P}(t)\mathbf{u} \bigl)\,dt~\biggm|~ \mathbf{x}_s = \mathbf{x} \biggr\} \text{.}
    \end{equation}
    
    Тогда оптимальное управление можно получить $\mathbf{u}^*(t)$ следующим образом:
    
    \begin{equation}\label{eq:14}
    	\mathbf{u}^*(\mathbf{x}, t) = -\mathbf{R}^{-1}\mathbf{B}^T(t)\mathbf{P}(t)\mathbf{x}(t) \text{,}
    \end{equation}

    где матрица $\mathbf{P}(t)$ находится из системы~\ref{eq:15}

    \begin{equation}\label{eq:15}
    	\begin{cases}
    		\begin{split}
    			-\frac{d\mathbf{P}(t)}{dt} =&~\mathbf{A}^T(t)\mathbf{P}(t) - \mathbf{P}(t)\mathbf{A}(t) +\\
                &+ \mathbf{P}(t)\mathbf{B}(t)\mathbf{R}^{-1}(t)\mathbf{B}(t)\mathbf{P}(t) - \mathbf{C}(t) \text{;}
			\end{split} \\
			\mathbf{P}(T) = \mathbf{R} \text{.}
		\end{cases}    
    \end{equation}
\end{frame}


\end{document}

